\section{Rationale}\label{rationale}

\subsection{100 SRFIs later\ldots{}}\label{srfis-later}

This SRFI is based on
\href{http://srfi.schemers.org/srfi-32/srfi-32.txt}{SRFI 32} by Olin
Shivers, which was withdrawn twelve years ago.

There are two backward incompatible changes to the API. The most
important one is that the comparison predicate now precedes the data
rather than following it. Most of the SRFI 32 commenters thought that
this is the better order, despite the then-widespread precedent. The
other change is that the algorithm-specific procedures have been removed
from the SRFI, though not necessarily from the sample implementation.
For documentation on them, see SRFI 32 or the source code.

Editorially, the text has been reorganized, removing the massive
redundancies. As in R5RS and R7RS, procedures that return nothing of
interest, and in SRFI 32 returned zero or more unspecified values, are
now specified to return one otherwise unspecified value. The multiple
packages, most with only a few procedures, into which SRFI 32 is divided
have been consolidated. References to
\href{http://srfi.schemers.org/srfi-95/srfi-95.html}{SRFI 95} and
\href{http://www.r6rs.org/final/html/r6rs-lib/r6rs-lib-Z-H-5.html\#node_chap_4}{R6RS}
equivalents have been added.

\subsection{What vs. how}\label{Whatvs.how}

In SRFI 32 there were two different interfaces: ``what'' (simple) and
``how'' (detailed).

\begin{itemize}
\tightlist
\item
  Simple: you specified semantics: datatype (list or vector),
  mutability, and stability.
\end{itemize}

\begin{itemize}
\tightlist
\item
  Detailed: you specified the datatype and the actual algorithm (quick,
  heap, insert, merge).
\end{itemize}

However, the difficulty with exposing specific algorithms by name is
that the local optima in the search space of algorithms changes over
time. For example, the ``qsort'' in the musl C library is actually
smooth sort, not quick sort; Python and Java have switched from quick
sort to timsort; some implementations of the C++ STL use introsort
rather than quick sort. Having procedure names that imply e.g. quick
sort, that do not actually implement quick sort, is confusing.

Therefore, only the ``what'' interface has been retained. The sample
implementation uses some of the original SRFI 32 algorithms, but this is
not a requirement of this SRFI.

\subsection{Consistency across procedure
signatures}\label{Consistencyacrossproceduresignatures}

Procedures share common signatures wherever possible, to facilitate
switching a given call from one procedure to another.

\subsection{Ordering parameter before data
parameter}\label{Orderingparameterbeforedataparameter}

These procedures uniformly observe the following parameter order: the
ordering, equality, or comparison argument precedes the argument(s)
denoting the data to be sorted. In SRFI 32, the data-first convention
was used, because it was consistent with all the implementations
examined in 2002, with the sole exception of Chez Scheme. However,
\href{http://www.r6rs.org/final/html/r6rs-lib/r6rs-lib-Z-H-5.html\#node_chap_4}{R6RS}
adopted the procedure-first convention, which is more consistent with
other Scheme libraries that put the procedure first --- the ``operation
currying'' convention, as in \texttt{for-each} or \texttt{map} or
\texttt{find}. This SRFI uses the R6RS convention throughout.

\subsection{Stability}\label{Stability}

A ``stable'' sort is one that preserves the pre-existing order of equal
elements. Suppose, for example, that we sort a list of numbers by
comparing their absolute values using \texttt{abs\textless{}} (defined
below). If we sort a list that contains both 3 and -3, then a stable
sort is an algorithm that will not swap the order of these two elements,
that is, the answer will look like \texttt{...\ 3\ -3\ ...}, not
\texttt{...\ -3\ 3\ ...}.

A ``stable'' merge is one that reliably favors one of its data sets when
equal items appear in both data sets. All merge operations specified in
this SRFI are stable, breaking ties between data sets in favor of the
first data set --- elements of the first list come before equal elements
in the second list.

So, if we are merging two lists of numbers ordered by absolute value
using the stable merge operation \texttt{list-merge} and
\texttt{abs\textless{}} compares numbers by their absolute values, then

\begin{verbatim}
(list-merge abs< '(0 -2 4 8 -10) '(-1 3 -4 7))
\end{verbatim}

reliably places the 4 of the first list before the equal-comparing -4 of
the second list: \texttt{(0\ -1\ -2\ 4\ -4\ 7\ 8\ -10)}.

Here's a definition of \texttt{abs\textless{}}, introduced just for the
examples as it is not part of this SRFI:

\begin{verbatim}
(define (abs< x y) (< (abs x) (abs y))))
\end{verbatim}

\subsection{All vector operations accept optional subrange
parameters}\label{Allvectoroperationsacceptoptionalsubrangeparameters}

The vector operations specified below all take optional \emph{start} and
\emph{end} arguments indicating a selected subrange of a vector's
elements. This compensates for the absence in most Schemes of shared
subvectors.

\subsection{Required vs. allowed
side-effects}\label{Requiredvs.allowedside-effects}

The \texttt{list-sort!} and \texttt{list-stable-sort!} procedures are
allowed, but not required, to alter their arguments' cons cells to
construct the result list.

The other procedures ending in \texttt{!}, on the other hand, explicitly
commit to the use of side-effects on their input lists in order to
guarantee their key algorithmic properties (e.g., linear-time operation,
constant-space stack use).

\subsection{Sorting lists}\label{Sortinglists}

Note that sorting lists involves chasing pointers through memory, which
can be a loser on modern machine architectures because of poor cache and
page locality. Pointer \emph{writing}, which is what the
\texttt{set-cdr!}s of a destructive list-sort algorithm do, is even
worse, especially if your Scheme has a generational GC --- the writes
will thrash the write-barrier. Sorting vectors has inherently better
locality.

In particular, all complexity guarantees assume that the basic accessors
and mutators of standard Scheme have O(1) space and time complexity.

The reference implementation's destructive list merge and merge sort
implementations are opportunistic --- they avoid redundant
\texttt{set-cdr!}s, and try to take long already-ordered runs of list
structure as-is when doing the merges.

