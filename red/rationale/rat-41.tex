\section{Streams: SRFI 41}

Harold Abelson and Gerald Jay Sussman discuss streams at length, giving
a strong justification for their use. The streams they provide are
represented as a cons pair with a promise to return a stream in its cdr;
for instance, a stream with elements the first three counting numbers is
represented conceptually as (cons 1 (delay (cons 2 (delay (cons 3 (delay
'())))))). Philip Wadler, Walid Taha and David MacQueen describe such
streams as \emph{odd} because, regardless of their length, the parity of
the number of constructors (delay, cons, '()) in the stream is odd.

The streams provided here differ from those of Abelson and Sussman,
being represented as promises that contain a cons pair with a stream in
its cdr; for instance, the stream with elements the first three counting
numbers is represented conceptually as (delay (cons 1 (delay (cons 2
(delay (cons 3 (delay '()))))))); this is an \emph{even} stream because
the parity of the number of constructors in the stream is even.

Even streams are more complex than odd streams in both definition and
usage, but they offer a strong benefit: they fix the off-by-one error of
odd streams. Wadler, Taha and MacQueen show, for instance, that an
expression like (stream-\textgreater{}list 4 (stream-map / (stream-from
4 -1))) evaluates to (1/4 1/3 1/2 1) using even streams but fails with a
divide-by-zero error using odd streams, because the next element in the
stream, which will be 1/0, is evaluated before it is accessed. This
extra bit of laziness is not just an interesting oddity; it is vitally
critical in many circumstances, as will become apparent below.

When used effectively, the primary benefit of streams is improved
modularity. Consider a process that takes a sequence of items, operating
on each in turn. If the operation is complex, it may be useful to split
it into two or more procedures in which the partially-processed sequence
is an intermediate result. If that sequence is stored as a list, the
entire intermediate result must reside in memory all at once; however,
if the intermediate result is stored as a stream, it can be generated
piecemeal, using only as much memory as required by a single item. This
leads to a programming style that uses many small operators, each
operating on the sequence of items as a whole, similar to a pipeline of
unix commands.

In addition to improved modularity, streams permit a clear exposition of
backtracking algorithms using the ``stream of successes'' technique, and
they can be used to model generators and co-routines. The implicit
memoization of streams makes them useful for building persistent data
structures, and the laziness of streams permits some multi-pass
algorithms to be executed in a single pass. Savvy programmers use
streams to enhance their programs in countless ways.

There is an obvious space/time trade-off between lists and streams;
lists take more space, but streams take more time (to see why, look at
all the type conversions in the implementation of the stream
primitives). Streams are appropriate when the sequence is truly
infinite, when the space savings are needed, or when they offer a
clearer exposition of the algorithms that operate on the sequence.
