\section{SRFI 124: Ephemerons}

Weak references are a mechanism for building data structures that point
at objects without protecting them from garbage collection. An example
of such a data structure might be an entry in a lookup table that should
be removed if the rest of the program does not reference its key. Such
an entry must still point at its key to carry out comparisons, but
should not in itself prevent its key from being garbage collected.

A weak reference is a reference that points at an object without
preventing it from being garbage collected. The term strong reference is
used to distinguish normal references from weak ones. If there is no
path of strong references to some object, the garbage collector will
reclaim that object and mark any weak references to it to indicate that
it has been reclaimed.

If there is a path of strong references from an object A to an object B,
A is said to hold B strongly. If there is a path of references from an
object A to an object B, but every such path traverses at least one weak
reference, A is said to hold B weakly.

Ephemerons, unlike simple weak pairs, allow the correct handling of data
structures in which the value points to the key. For example, a record
may have a single field holding the key; if it's desired to use these
records as values in a weak hash table using this internal key,
ephemerons are necessary or the table will not really be weak.

Ephemerons are considerably heavier-weight than simple weak pairs,
because garbage-collecting ephemerons is more complicated than
garbage-collecting weak pairs. In MIT Scheme, each ephemeron needs five
words of storage rather than the two words needed by a weak pair.
However, while the GC needs to spend more time on ephemerons than on
other objects, the amount of time it spends on ephemerons can be made to
scale linearly with the number of live ephemerons, which is how a
copying GC's running time scales with the total number of live objects
anyway.

