\section{Sets and Bags}

Sets are a standard part of the libraries of many high-level programming
languages, including Smalltalk,
\href{http://docs.oracle.com/javase/6/docs/api/java/util/Set.html}{Java},
and \href{http://www.cplusplus.com/reference/set/set/}{C++}.
\href{http://docs.racket-lang.org/reference/sets.html}{Racket} provides
general sets similar to those of this proposal, though with fewer
procedures, and there is a Chicken egg called \texttt{sets}
(unfortunately undocumented), which provides a minimal set of
procedures. \href{http://srfi.schemers.org/srfi-1/srfi-1.html}{SRFI 1}
also provides a list-based implementation of sets.

Bags are useful for counting anything from a fixed set of possibilities,
e.g. the number of each type of error in a log file or the number of
uses of each word in a lexicon drawn from a body of documents. Although
other data structures can serve the same purpose, using bags clearly
expresses the programmer's intent and allows for optimization.

Insofar as possible, the names in this SRFI are harmonized with the
names used for ordered collections (lists, strings, vectors, and
bytevectors) in Scheme. However, \texttt{size} is used instead of
\texttt{length} to express the number of elements in a collection,
because \texttt{length} implies order.

It's possible to use the general sets of this SRFI to contain
characters, but the use of
\href{http://srfi.schemers.org/srfi-14/srfi-14.html}{SRFI 14} is
recommended instead. The names and facilities in this SRFI are
harmonized with SRFI 14, except that SRFI 14 does not contain analogues
of the \texttt{set-search!}, \texttt{set\textgreater{}?},
\texttt{set\textless{}=?}, \texttt{set\textgreater{}=?},
\texttt{set-remove}, or \texttt{set-partition} procedures.

Sets and bags do not have a lexical syntax representation. It's possible
to use \href{http://srfi.schemers.org/srfi-108/srfi-108.html}{SRFI 108}
quasi-literal constructors to create them in code, but this SRFI does
not standardize how that is done.

The interface to general sets and bags depends on
\href{http://srfi.schemers.org/srfi-114/srfi-114.html}{SRFI 114}
comparators, despite that SRFI having a higher number than this one for
\href{http://www.catb.org/jargon/html/H/hysterical-reasons.html}{hysterical
raisins}. Comparators conveniently package the equality predicate of the
set with the hash function or comparison procedure needed to implement
the set efficiently.
