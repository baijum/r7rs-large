\section{Character sets}

The ability to efficiently manipulate sets of characters is quite useful
for text-processing code. Encapsulating this functionality in a general,
efficiently implemented library can assist all such code. This library
defines a new data structure to represent these sets, called a
``char-set.'' The char-set type is distinct from all other types.

This library is designed to be portable across implementations that use
different character types and representations, especially ASCII, Latin-1
and Unicode. Some effort has been made to preserve compatibility with
Java in the Unicode case (see the definition of
\texttt{char-set:whitespace} for the single real deviation).

\subsection{{Linear-update operations}}\label{linear-update-operations}

The procedures of this SRFI, by default, are ``pure functional'' -- they
do not alter their parameters. However, this SRFI defines a set of
``linear-update'' procedures which have a hybrid
pure-functional/side-effecting semantics: they are allowed, but not
required, to side-effect one of their parameters in order to construct
their result. An implementation may legally implement these procedures
as pure, side-effect-free functions, or it may implement them using side
effects, depending upon the details of what is the most efficient or
simple to implement in terms of the underlying representation.

The linear-update routines all have names ending with ``!''.

Clients of these procedures \emph{may not} rely upon these procedures
working by side effect. For example, this is not guaranteed to work:

\begin{verbatim}
(let* ((cs1 (char-set #\a #\b #\c))      ; cs1 = {a,b,c}.
       (cs2 (char-set-adjoin! cs1 #\d))) ; Add d to {a,b,c}.
  cs1) ; Could be either {a,b,c} or {a,b,c,d}.
\end{verbatim}

However, this is well-defined:

\begin{verbatim}
(let ((cs (char-set #\a #\b #\c)))
  (char-set-adjoin! cs #\d)) ; Add d to {a,b,c}.
\end{verbatim}

So clients of these procedures write in a functional style, but must
additionally be sure that, when the procedure is called, there are no
other live pointers to the potentially-modified character set (hence the
term ``linear update'').

There are two benefits to this convention:

\begin{itemize}
\tightlist
\item
  Implementations are free to provide the most efficient possible
  implementation, either functional or side-effecting.
\item
  Programmers may nonetheless continue to assume that character sets are
  purely functional data structures: they may be reliably shared without
  needing to be copied, uniquified, and so forth.
\end{itemize}

Note that pure functional representations are the right thing for ASCII-
or Latin-1-based Scheme implementations, since a char-set can be
represented in an ASCII Scheme with 4 32-bit words. Pure set-algebra
operations on such a representation are very fast and efficient.
Programmers who code using linear-update operations are guaranteed the
system will provide the best implementation across multiple platforms.

In practice, these procedures are most useful for efficiently
constructing character sets in a side-effecting manner, in some limited
local context, before passing the character set outside the local
construction scope to be used in a functional manner.

Scheme provides no assistance in checking the linearity of the
potentially side-effected parameters passed to these functions ---
there's no linear type checker or run-time mechanism for detecting
violations. (But sophisticated programming environments, such as
DrScheme, might help.)

\subsection{{Extra-SRFI recommendations}}\label{extra-srfi-recommendations}

Users are cautioned that the R5RS predicates

\texttt{\ char-alphabetic?\ \ char-numeric?\ \ char-whitespace?\ \ char-upper-case?\ \ char-lower-case?\ }

may or may not be in agreement with the SRFI 14 base character sets

\texttt{\ char-set:letter\ char-set:digit\ char-set:whitespace\ char-set:upper-case\ char-set:lower-case}

Implementors are strongly encouraged to bring these predicates into
agreement with the base character sets of this SRFI; not to do so risks
major confusion.

