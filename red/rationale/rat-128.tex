\section{SRFI 128: Comparators}

The four procedures above have complex dependencies on one another, and
it is inconvenient to have to pass them individually to other procedures
that might or might not make use of all of them. For example, a set
implementation by its nature requires only an equality predicate, but if
it is implemented using a hash table, an appropriate hash function is
also required if the implementation does not provide one; alternatively,
if it is implemented using a tree, procedures specifying a total order
are required. By passing a comparator rather than a bare equality
predicate, the set implementation can make use of whatever procedures
are available and useful to it.

This SRFI is a simplified and enhanced rewrite of
\href{http://srfi.schemers.org/srfi-114/srfi-114.html}{SRFI 114}, and
shares some of its design rationale and all of its acknowledgements. The
largest change is the replacement of the comparison procedure with the
ordering procedure. This allowed most of the special-purpose comparators
to be removed. In addition, many of the more specialized procedures, as
well as all but one of the syntax forms, have been removed as
unnecessary.

Special thanks to Taylan Ulrich Bayırlı/Kammer, whose insistence that
SRFI 114 was unacceptable inspired this redesign. Jörg Wittenberger
added Chicken-specific type declarations, which I have moved to
\texttt{comparators.scm}, as it is a Chicken-specific library. He also
provided Chicken-specific metadata and setup commands. Comments from
Shiro Kawai, Alex Shinn, and Kevin Wortman guided me to the current
design for bounds and salt.