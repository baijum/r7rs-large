\section{Boxes: SRFI 111}

\TODO{I've duplicated this paragraph in the Report, because it's more
  like an introduction than rationale.}
A box is a container for an object of any Scheme type, including another
box. It is like a single-element vector, or half of a pair, or a direct
representation of state. Boxes are normally used as minimal mutable
storage, and can inject a controlled amount of mutability into an
otherwise immutable data structure (or one that is conventionally
treated as immutable). They can be used to implement call-by-reference
semantics by passing a boxed value to a procedure and expecting the
procedure to mutate the box before returning.

Some Scheme systems use boxes to implement \texttt{set!}. In this
transformation, known as \emph{assignment conversion}, all variables
that are actually mutated are initialized to boxes, and all
\texttt{set!} syntax forms become calls on \texttt{set-box!}. Naturally,
all ordinary references to those variables must become calls on
\texttt{unbox}. By reducing all variable mutation to data-structure
mutation in this way, such Scheme systems are free to maintain variables
in multiple hardware locations, such as the stack and the heap or
registers and the stack, without worrying about exactly when and where
they are mutated.

Boxes are also useful for providing an extra level of indirection,
allowing more than one body of code or data structure to share a
reference, or pointer, to an object. In this way, if any procedure
mutates the box in any of the data structures, all procedures will
immediately ``see'' the new value in all data structures containing it.

Racket and Chicken provide \emph{immutable boxes}, which look like boxes
to \texttt{box?} and \texttt{unbox} but which cannot be mutated. They
are not considered useful enough to be part of this SRFI. If they are
provided nevertheless, the recommended constructor name is
\texttt{immutable-box}.

The following table specifies the procedure names used by the Scheme
implementations that provide boxes:

\begin{longtable}[]{@{}llllllll@{}}
\toprule
Proposed &
\href{http://docs.racket-lang.org/reference/boxes.html}{Racket} &
\href{http://www.iro.umontreal.ca/~gambit/doc/gambit-c.html\#index-boxes}{Gambit}
& \href{http://sisc-scheme.org/manual/html/ch03.html\#Boxing}{SISC} &
\href{http://www.scheme.com/csug7/objects.html\#g50}{Chez} &
\href{http://wiki.call-cc.org/eggref/4/box}{Chicken} &
\href{http://web.mit.edu/scheme_v9.0.1/doc/mit-scheme-ref/Cells.html}{MIT}
&
\href{http://s48.org/1.1/manual/s48manual_42.html}{Scheme48/scsh}\tabularnewline
box & box & box & box & box & make-box & make-cell &
make-cell\tabularnewline
box? & box? & box? & box? & box? & box-mutable? & cell? &
cell?\tabularnewline
unbox & unbox & unbox & unbox & unbox & box-ref & cell-contents &
cell-ref\tabularnewline
set-box\texttt{!} & set-box\texttt{!} & set-box\texttt{!} &
set-box\texttt{!} & set-box\texttt{!} & box-set\texttt{!} &
set-cell-contents\texttt{!} & cell-set\texttt{!}\tabularnewline
\bottomrule
\end{longtable}

Note that Chicken also supports the procedure names defined by this SRFI
in addition to its native API. Although the native API uses the standard
naming pattern, for the purposes of this SRFI the unanimous voice of
Racket, Gambit, SISC, and Chez is considered more significant.

Racket, Gambit, SISC, Chez, and Chicken all support the lexical syntax
\texttt{\#\&}\emph{datum} to create a literal box (immutable in Racket,
mutable in the other implementations). However, this SRFI does not.
Incorporating a specifically mutable object into code makes it hard for
an implementation to separate compile-time and run-time reliably, and
it's not clear that providing them in data files provides enough
benefit, as the box has to be located by tree-walking in any case. MIT
Scheme and Scheme48/scsh do not provide a lexical syntax.

The features specified in the autoboxing section of specification are
based on those specified by RnRS for promises, which are analogous to
immutable boxes except that their value is specified by code instead of
data.

