\section{Rationale}\label{rationale}

The main purpose of generators is high performance. Although
\href{http://srfi.schemers.org/srfi-41/srfi-41.html}{SRFI 41} streams
can do everything generators can do and more, SRFI 41 uses lazy pairs
that require making a thunk for every item. Generators can generate
items without consing, so they are very lightweight and are useful for
implementing simple on-demand calculations.

Existing examples of generators are readers from the current input port
and \href{http://srfi.schemers.org/srfi-27/srfi-27.html}{SRFI 27} random
numbers. If Scheme had streams as one of its built-in abstractions,
these would have been naturally represented by lazy streams. But Scheme
usually does not expose this kind of API using lazy streams.
Generator-like interfaces are common, so it seems worthwhile to have
some common idioms extracted into a library.

Calling a generator is a side-effecting construct; you can't safely
backtrack, for example, as you can with streams. Persistent lazy
sequences based on generators and ordinary Scheme pairs (which are
heavier weight than generators, but lighter weight than lazy pairs) is
the subject of SRFI 127. Of course the efficiency of streams depends on
the implementation. Some implementations may have have super-light thunk
creation. But in most, thunk creation is probably slower than simple
consing.

The generators of this SRFI don't belong to a disjoint type. They are
just procedures that conform to a calling convention, so you can
construct a generator with \texttt{lambda}. The constructors of this
SRFI are provided for convenience. Any procedure that can be called with
no arguments can serve as a generator.

Using an end-of-file object to indicate that there is no more input
makes it impossible to include such an object in the stream of generated
values. However, it is compatible with the existing design of input
ports, and it makes for more compact code than returning a
user-specified termination object (as in Common Lisp) or returning
multiple values. (Note that some generators are infinite in length, and
never return an end-of-file object.)

The combination of \texttt{make-for-each-generator} and
\texttt{generator-unfold} makes it possible to convert any collection
that has a for-each procedure into any collection that has an unfold
constructor. This generalizes such conversion procedures as
\texttt{\&list-\textgreater{}vector} and
\texttt{string-\textgreater{}list}.
