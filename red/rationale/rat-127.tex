\section{SRFI 127}

Lazy sequences are more heavyweight than generators, on which they are
based, but they are more lightweight than
\href{http://srfi.schemers.org/srfi-41/srfi-41.html}{SRFI 41} streams.
However, streams are \emph{even}, as explained in the SRFI 41 rationale;
that is, the initial state of a stream does not have any elements that
have already been realized. By contrast, lazy sequences are \emph{odd},
meaning that at least one element is realized at all times unless the
lseq is empty. Therefore, when constructing an lseq in an iterative lazy
algorithm, only the cdr side of the lazy pair is lazily evaluated; the
car side is evaluated immediately, even if it is never used.

In most cases this doesn't matter, because calculating one additional
item is a negligible overhead. However, when you create a
self-referential lazy structure, in which the earlier elements of a
sequence are used to calculate the later elements of itself, a bit of
caution is needed; code that is valid for circular streams may not
terminate if it is mechanically converted to use lazy sequences. This
eagerness is also visible when side effects are involved; for example, a
lazy character sequence reading from a port may read one character
ahead.

This SRFI is less comprehensive than SRFI 1, because it omits many
procedures that process every element of their list arguments (at least,
when used in the absence of \texttt{call/cc}). Lseqs are meant to be
used with ordinary Scheme functions, which are strict, so neither left
nor right folds are provided: it is just as time-efficient to use
\texttt{lseq-realize} and then SRFI 1 \texttt{fold} or
\texttt{fold-right}, and just as space-efficient to use
\texttt{lseq-\textgreater{}generator} and SRFI 121's
\texttt{generator-fold}. Exceptions are \texttt{lseq-length} and
\texttt{lseq-for-each}, which are very common operations on lists and so
are provided.

Lazy alists have been omitted, as the point of an alist is to express a
function whose results are not algorithmically deducible from its
arguments. As a result, there is little point in using a generator
procedure to produce key-value mappings when an ordinary procedure can
compute the value from the key directly. The linear-update procedures of
SRFI 1 are also left out, as lazy sequences are not intended to be
mutated.