\section{Ephemerons}
\redno{17}  What ephemeron library should R7RS-large provide?

\begin{quote}
SRFI 124 provides ephemerons compatible with MIT Scheme except that they
are immutable. A dummy implementation for systems without GC support is
also provided.
\end{quote}


An ephemeron is an object with two components called its \emph{key} and
its \emph{datum}. It differs from an ordinary pair as follows: if the
garbage collector (GC) can prove that there are no references to the key
except from the ephemeron itself and possibly from the datum, then it is
free to \emph{break} the ephemeron, dropping its reference to both key
and datum. In other words, an ephemeron can be broken when nobody else
cares about its key. Ephemerons can be used to construct weak vectors or
lists and (possibly in combination with finalizers) weak hash tables.

Much of this specification is derived with thanks from the MIT Scheme
Reference Manual.

\setlibname{ephemeron}

\section{Specification}\label{specification}

\begin{entry}{%
  \Proto{ephemeron?}{ object}{procedure}{boolean}}

  Returns \texttt{\#t} if \emph{object} is an ephemeron; otherwise
  returns \texttt{\#f}.
\end{entry}

\begin{entry}{%
  \Proto{make-ephemeron}{ key datum}{procedure}{ephemeron}}

  Returns a newly allocated ephemeron, with components \emph{key} and
  \emph{datum}. Note that if \emph{key} and \emph{datum} are the same
  in the sense of \texttt{eq?}, the ephemeron is effectively a weak
  reference to the object.
\end{entry}

\begin{entry}{%
  \Proto{ephemeron-broken?}{ ephemeron}{procedure}{boolean}}

  Returns \#t if \emph{ephemeron} has been broken; otherwise returns
  \#f.

  This procedure must be used with care. If it returns \texttt{\#f},
  that guarantees only that prior evaluations of
  \texttt{ephemeron-key} or \texttt{ephemeron-datum} yielded the key
  or datum that was stored in \emph{ephemeron}. However, it makes no
  guarantees about subsequent calls to \texttt{ephemeron-key} or
  \texttt{ephemeron-datum}, because the GC may run and break the
  ephemeron immediately after \texttt{ephemeron-broken?}
  returns. Thus, the correct idiom to fetch an ephemeron's key and
  datum and use them if the ephemeron is not broken is:

\begin{verbatim}
     (let ((key (ephemeron-key ephemeron))
           (datum (ephemeron-datum ephemeron)))
       (if (ephemeron-broken? ephemeron)
           ... broken case ...
           ... code using key and datum ...))
\end{verbatim}
\end{entry}

\begin{entry}{%
  \Proto{ephemeron-key}{ ephemeron}{procedure}{value}
  \Proto{ephemeron-datum}{ ephemeron}{procedure}{value}}

  These return the key or datum component, respectively, of
  \emph{ephemeron}. If \emph{ephemeron} has been broken, these
  operations return \texttt{\#f}, but they can also return
  \texttt{\#f} if that is what was stored as the key or datum.
\end{entry}

\begin{entry}{%
  \Proto{reference-barrier}{ key}{procedure}{???}}

  \emph{This procedure is optional.}

  This procedure ensures that the garbage collector does not break an
  ephemeron containing an unreferenced key before a certain point in a
  program. The program can invoke a \emph{reference barrier} on the
  key by calling this procedure, which guarantees that even if the
  program does not use the key, it will be considered strongly
  reachable until after \texttt{reference-barrier} returns.

  \TODO{What does “optional” mean here? Can a conforming implementation
    just not export it? Or always generate an error? Is there a
    cond-expand feature for this?}
\end{entry}

\section{References}\label{references}

The original paper on ephemerons is Barry Hayes,
\href{https://static.aminer.org/pdf/PDF/000/522/273/ephemerons_a_new_finalization_mechanism.pdf}{``Ephemerons:
a New Finalization Mechanism''}, \emph{Object-Oriented Languages,
Programming, Systems, and Applications}, 1997.

A useful implementation paper is Bruno Haible,
\href{http://www.haible.de/bruno/papers/cs/weak/WeakDatastructures-writeup.html}{``Weak
References: Data Types and Implementation''} posted 2005-04-24.

\href{http://www.inf.puc-rio.br/~roberto/docs/ry08-06.pdf}{``Eliminating
Cycles in Weak Tables''} is about the addition of ephemeron tables to
Lua 5.2. It explains how Lua's tricolor mark-sweep GC is extended to
handle them, unfortunately using the O(N\^{}2) algorithm of Hayes 1997.

\href{https://www.cs.hmc.edu/~oneill/papers/Blobs-SPACE.pdf}{``Extending
Garbage Collection to Complex Data Structures''} extends ephemerons,
which have a fixed GC strategy, to a new data structure called
\emph{blobs} that explain to the GC how they are to be processed, thus
allowing an arbitrary number of key-like and value-like pointers.

\section{Verse}\label{verse}

\TODO{With all respect to gls, does this stay in the final report?}

Reclaimer, spare that tree!\\
Take not a single bit!\\
It used to point to me,\\
Now I'm protecting it.\\
It was the reader's \texttt{cons}\\
That made it, paired by dot;\\
Now, GC, for the nonce,\\
Thou shalt reclaim it not.

That old familiar tree,\\
Whose \texttt{cdr}s and whose \texttt{car}s\\
Are spread o'er memory ---\\
And wouldst thou it
\href{http://catb.org/jargon/html/P/parse.html}{unparse}?\\
GC, cease and desist!\\
In it no free list store;\\
Oh spare that \href{http://catb.org/jargon/html/M/moby.html}{moby}
list\\
Now pointing throughout core!

It was my parent tree\\
When it was circular;\\
It pointed then to me:\\
I was its \texttt{cadadr}.\\
My \texttt{cdr} was a list,\\
My \texttt{car} a dotted pair ---\\
That tree will sore be missed\\
If it remains not there.

And now I to thee point,\\
A saving root, old friend!\\
Thou shalt remain disjoint\\
From free lists to the end.\\
Old tree! The
\href{https://en.wikipedia.org/wiki/Tracing_garbage_collection\#Na.C3.AFve_mark-and-sweep}{sweep}
still brave!\\
And, GC,
\href{https://en.wikipedia.org/wiki/Tracing_garbage_collection\#Na.C3.AFve_mark-and-sweep}{mark}
this well:\\
While I exist to save,\\
Thou shan't reclaim one cell.

\begin{quote}
\href{https://en.wikipedia.org/wiki/Guy_L._Steele,_Jr.}{The Great Quux}
(with \href{http://www.bartleby.com/248/131.html}{apologies} to
\href{https://en.wikipedia.org/wiki/George_Pope_Morris}{George Pope
Morris})
\end{quote}

(Pedantic note: In Scheme, pairs constructed by \texttt{read} are
officially immutable, so this scenario is technically impossible.)



\begin{center}\rule{0.5\linewidth}{\linethickness}\end{center}

Editor:
\href{mailto:srfi-editors\%20at\%20srfi\%20dot\%20schemers\%20dot\%20org}{Arthur
A. Gleckler}
