\section{Boxes}
\redno{15}  What box library should R7RS-large provide? *

\begin{quote}
SRFI 111 is a trivial library for boxes (single-slot records).
\end{quote}

A box is a container for an object of any Scheme type, including another
box. It is like a single-element vector, or half of a pair, or a direct
representation of state. Boxes are normally used as minimal mutable
storage, and can inject a controlled amount of mutability into an
otherwise immutable data structure (or one that is conventionally
treated as immutable). They can be used to implement call-by-reference
semantics by passing a boxed value to a procedure and expecting the
procedure to mutate the box before returning.

Boxes are objects with a single mutable state. Several Schemes have
them, sometimes called \emph{cells}. A constructor, predicate, accessor,
and mutator are provided. \TODO{keep this ¶?}

\setlibname{box}


\subsection{Procedures}\label{procedures}

The following procedures implement the box type (which is disjoint from
all other Scheme types).\st{, and are exported by the \texttt{(srfi 111)}
library (or \texttt{(srfi :111)} on R6RS).}

\begin{entry}{%
  \Proto{box}{ value}{procedure}{box}}

  Constructor. Returns a newly allocated box initialized to
  \emph{value}.
\end{entry}

\begin{entry}{%
  \Proto{(box?}{ object}{procedure}{boolean}}

  Predicate. Returns \texttt{\#t} if \emph{object} is a box, and
  \texttt{\#f} otherwise.
\end{entry}

\begin{entry}{%
  \Proto{unbox}{ box}{procedure}{value}}

  Accessor. Returns the current value of \emph{box}.
\end{entry}

\begin{entry}{%
  \Proto{set-box!}{ box value}{procedure}{unspecified?}}

  Mutator. Changes \emph{box} to hold \emph{value}.

  The behavior of boxes with the equivalence predicates \texttt{eq?},
  \texttt{eqv?}, and \texttt{equal?} is the same as if they were
  implemented with records. That is, two boxes are both \texttt{eq?}
  and \texttt{eqv?} iff they are the product of the same call to
  \texttt{box} and not otherwise, and while they must be
  \texttt{equal?} if they are \texttt{eqv?}, the converse is
  implementation-dependent.
\end{entry}


\subsection{Autoboxing (optional)}\label{autoboxing-optional}

The following provisions of this SRFI are optional: \TODO{Are they
  optional in R7RS-Large?}

\begin{itemize}
\item
  A procedure, whether system-provided or user-written, that expects a
  box as an argument but receives a non-box may, if appropriate,
  allocate a box itself that holds the value, thus providing autoboxing.
\item
  A procedure that accepts arguments only of specified types (such as
  \texttt{+}) but receives a box instead may, if appropriate, unbox the
  box. Procedures that accept arguments of any type (such as
  \texttt{cons}) must not unbox their arguments.
\item
  Calling \texttt{unbox} on a non-box may simply return the non-box.
\end{itemize}

\begin{center}\rule{0.5\linewidth}{\linethickness}\end{center}

Editor:
\href{mailto:srfi-editors\%20at\%20srfi\%20dot\%20schemers\%20dot\%20org}{Mike
Sperber}

Last modified: Wed Jul 3 09:04:14 MST 2013
