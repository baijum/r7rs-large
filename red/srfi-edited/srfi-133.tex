\chapter{Vector libraries}

\redno{2} What vector library should R7RS-large provide?

This SRFI proposes a comprehensive library of
vector operations accompanied by a freely available and complete
reference implementation. The reference implementation is unencumbered
by copyright, and useable with no modifications on any Scheme system
that is R5RS-compliant. It also provides
several hooks for implementation-specific optimization as well.

\setlibname{vector}

\subsection{Procedure Index}

Here is an index of the procedures provided by this package. Those
marked by \emph{italics} are also provided in
\protect\hyperlink{R7RS-small}{R7RS-small}.

\begin{description}\obeyspaces\obeylines
\item[Constructors]%
\texttt{%
make-vector            vector                vector-unfold
vector-unfold-right    vector-copy           vector-reverse-copy
vector-append          vector-concatenate    vector-append-subvectors}
\item[Predicates]
\texttt{%
vector?                vector-empty?         vector=}
\item[Selectors]
\texttt{%
vector-ref             vector-length}
\item[Iteration]
\texttt{%
vector-fold            vector-fold-right     vector-map
vector-map!            vector-for-each       vector-count
vector-cumulate}
\item[Searching]
\texttt{%
vector-index           vector-index-right    vector-skip
vector-skip-right      vector-binary-search  vector-any 
vector-every           vector-partition}
\item[Mutators]
\texttt{%
vector-set!            vector-swap!          vector-fill!
vector-reverse!        vector-copy!          vector-reverse-copy!
vector-unfold!         vector-unfold-right!}
\item[Conversion]
\texttt{%
vector->list           reverse-vector->list  list->vector
reverse-list->vector   vector->string        string->vector}
\end{description}

\subsection{Notation}

\TODO{Unify with notation in the rest of the Report, move to a common
  section.}

In this section containing specifications of procedures, the following
notation is used to specify parameters and return values:
 
\begin{description}
\item[ (f arg$_1$ arg$_2$ \ldots{})
-> something]
Indicates a function \texttt{f} takes the parameters
\texttt{arg1\ arg2\ \ \ \ \ \ \ \ \ ...} and returns a value of the type
\texttt{something}. If \texttt{something} is \texttt{unspecified}, then
\texttt{f} returns a single implementation-dependent value; this SRFI
does not specify what it returns, and in order to write portable code,
the return value should be ignored.

\item[vec]
The argument in this place must be a vector, i.e. it must satisfy the
predicate \texttt{vector?}.

\item[i, j, start, size]
The argument in this place must be a exact nonnegative integer, i.e. it
must satisfy the predicates \texttt{exact?}, \texttt{integer?} and
either \texttt{zero?} or \texttt{positive?}. The third case of it
indicates the index at which traversal begins; the fourth case of it
indicates the size of a vector.

\item[end]
The argument in this place must be a exact positive integer, i.e. it
must satisfy the predicates \texttt{exact?}, \texttt{integer?} and
\texttt{positive?}. This indicates the index directly before which
traversal will stop --- processing will occur until the the index of the
vector is \texttt{end}. It is the closed right side of a
range.

\item[f]
The argument in this place must be a function of one or more arguments,
which returns (except as noted otherwise) exactly one
value.

\item[pred?]
The argument in this place must be a function of one or more arguments
that returns one value, which is treated as a boolean.

\item[ x, y, z, seed, knil, fill, key, value ]
The argument in this place may be any Scheme value.

\item[{[}something{]}]
Indicates that \texttt{something} is an optional argument; it needn't
necessarily be applied. \texttt{Something} needn't necessarily be one
thing; for example, this usage of it is perfectly
valid:
\texttt{\ \ \ \ \ \ \ \ \ \ \ ~~~{[}start\ {[}end{]}{]}\ \ \ \ \ \ \ \ \ }
and
is indeed used quite often.

\item[something \ldots{}]
Indicates that zero or more \texttt{something}s are allowed to be
arguments.

\item[ something$_1$ something$_2$ \ldots{} ]
Indicates that at least one \texttt{something} must be
arguments.

\item[ something$_1$ something$_2$ \ldots{}
something\textsubscript{n} ]
Exactly equivalent to the previous argument notation, but this also
indicates that \texttt{n} will be used later in the procedure
description.

\end{description}

It should be noted that all of the procedures that iterate across
multiple vectors in parallel stop iterating and produce the final result
when the end of the shortest vector is reached. The sole exception is
\texttt{vector=}, which automatically returns \texttt{\#f} if the
vectors' lengths vary.

\subsection{Constructors}

\begin{entry}{%
  \Proto{make-vector}{ size fill}{procedure}{vector}
  \Rproto{make-vector}{ size}{procedure}{vector}}

  {[}\protect\hyperlink{R7RS-small}{\emph{R7RS-small}}{]} Creates
  and returns a vector of size \texttt{size}. If \emph{fill} is
  specified, all the elements of the vector are initialized to
  \emph{fill}. Otherwise, their contents are indeterminate.  Example:
\begin{verbatim}
           (make-vector 5 3)         
           \#(3 3 3 3 3)         
\end{verbatim}
\end{entry}

\begin{entry}{%
  \Proto{vector}{ x \ldots}{procedure}{vector}}

  {[}\protect\hyperlink{R7RS-small}{\emph{R7RS-small}}{]} Creates and
  returns a vector whose elements are \texttt{x\ ...}.  Example:
\begin{verbatim}
           (vector 0 1 2 3 4)         
           \#(0 1 2 3 4)         
\end{verbatim}
\end{entry}

\begin{entry}{%
  \Proto{vector-unfold}{ f length initial-seed
    \ldots}{procedure}{vector}}

 ] The fundamental vector constructor. Creates a vector
  whose length is \texttt{length} and iterates across each index
  \texttt{k} between \texttt{0} and \texttt{length}, applying
  \texttt{f} at each iteration to the current index and current seeds,
  in that order, to receive \texttt{n\ +\ 1} values: first, the
  element to put in the \texttt{k}th slot of the new vector and
  \texttt{n} new seeds for the next iteration.  It is an error for the
  number of seeds to vary between iterations. Note that the
  termination condition is different from the \texttt{unfold}
  procedure of \protect\hyperlink{SRFIux5cux25201}{SRFI 1}.  Examples:
\begin{verbatim}
           (vector-unfold (λ (i x) (values x (- x 1))) 10 0)         
           #(0 -1 -2 -3 -4 -5 -6 -7 -8 -9)         
\end{verbatim}
  Construct
  a vector of the sequence of integers in the range {[}0,\texttt{n}).\\
\begin{verbatim}
           (vector-unfold values n)         
           \#(0 1 2 ... n-2 n-1)         
\end{verbatim}
  Copy \texttt{vector}.
\begin{verbatim}
           (vector-unfold (λ (i) (vector-ref vector i))           
                    (vector-length vector))         
\end{verbatim}
\end{entry}

\begin{entry}{%
  \Proto{vector-unfold-right}{ f length initial-seed
    \ldots}{procedure}{vector}}

  Like \texttt{vector-unfold}, but it uses
  \texttt{f} to generate elements from right-to-left, rather than
  left-to-right. The first index used is \emph{length} - 1. Note that
  the termination condition is different from the
  \texttt{unfold-right} procedure of
  \protect\hyperlink{SRFIux5cux25201}{SRFI 1}.  Examples: Construct a
  vector of pairs of non-negative integers whose values sum to 4.
\begin{verbatim}
           (vector-unfold-right (λ (i x) (values (cons i x) (+ x 1))) 5 0)         
           \#((0 . 4) (1 . 3) (2 . 2) (3 . 1) (4 . 0))         
\end{verbatim}
  Reverse \texttt{vector}.
\begin{verbatim}
           (vector-unfold-right (λ (i x)
                                  (values (vector-ref vector x)   
                                       (+ x 1)))
           (vector-length vector)
           0)         
\end{verbatim}
\end{entry}

\begin{entry}{%
  \Proto{vector-copy}{ vec start end}{procedure}{vector}
  \Rproto{vector-copy}{ vec start}{procedure}{vector}
  \Rproto{vector-copy}{ vec}{procedure}{vector}}

  {[}\protect\hyperlink{R7RS-small}{\emph{R7RS-small}}{]} Allocates a
  new vector whose length is \texttt{end\ -\ \ \ \ \ \ \ \ \ start}
  and fills it with elements from \texttt{vec}, taking elements from
  \texttt{vec} starting at index \texttt{start} and stopping at index
  \texttt{end}.  \texttt{start} defaults to \texttt{0} and
  \texttt{end} defaults to the value of \texttt{(vector-length\ \ \ \
    \ \ \ \ \ vec)}. SRFI 43 provides an optional fill argument to
  supply values if end is greater than the length of vec. Neither
  R7RS-small nor this SRFI requires support for this argument.
  Examples:
\begin{verbatim}
           (vector-copy '\#(a b c d e f g h i))         
           \#(a b c d e f g h i)         
           (vector-copy '\#(a b c d e f g h i) 6)        
           \#(g h i)         
           (vector-copy '\#(a b c d e f g h i) 3 6)         
           \#(d e f)         
\end{verbatim}
\end{entry}

\begin{entry}{%
  \Proto{vector-reverse-copy}{ vec start end}{procedure}{vector} }

 Like \texttt{vector-copy}, but
  it copies the elements in the reverse order from \texttt{vec}.
  Example:
\begin{verbatim}
           (vector-reverse-copy '\#(5 4 3 2 1 0) 1 5)         
           \#(1 2 3 4)         
\end{verbatim}
\end{entry}

\begin{entry}{%
  \Proto{vector-append}{ vec \ldots}{procedure}{vector}}

  {[}\protect\hyperlink{R7RS-small}{\emph{R7RS-small}}{]} Returns a
  newly allocated vector that contains all elements in order from the
  subsequent locations in \texttt{vec\ \ \ \ \ \ \ \ \ ...}.
  Examples:
\begin{verbatim}
           (vector-append '\#(x) '\#(y))         
           \#(x y)         
           (vector-append '\#(a) '\#(b c d))         
           \#(a b c d)         
           (vector-append '\#(a \#(b)) '\#(\#(c)))         
           \#(a \#(b) \#(c))         
\end{verbatim}
\end{entry}

\begin{entry}{%
  \Proto{vector-concatenate}{ list-of-vectors}{procedure}{vector}}

   Appends each vector in \texttt{list-of-vectors}. This is
  equivalent to: \texttt{(apply vector-append list-of-vectors)}
  However, it may be implemented better.  Example:
  \texttt{\ \ \ \ \ \ \ \ \ \ \ (vector-concatenate\ '(\#(a\ b)\ \#(c\ d)))\ \ \ \ \ \ \ \ \ }\\
  \texttt{\ \ \ \ \ \ \ \ \ \ \ \#(a\ b\ c\ d)\ \ \ \ \ \ \ \ \ }
\end{entry}
 
\begin{entry}{%
  \Proto{vector-append-subvectors}{ {[}vec start end{]} \ldots}{procedure}{vector}}

  \TODO{Check the signature on this one. }

  Returns a vector that contains every element of each \emph{vec} from
  \emph{start} to \emph{end} in the specified order. This procedure is
  a generalization of \texttt{vector-append}.  Example:
  \texttt{\ \ \ \ \ \ \ \ \ \ \ (vector-append-subvectors\ '\#(a\ b\ c\ d\ e)\ 0\ 2\ '\#(f\ g\ h\ i\ j)\ 2\ 4)\ \ \ \ \ \ \ \ \ }\\
  \texttt{\ \ \ \ \ \ \ \ \ \ \ \#(a\ b\ h\ i)\ \ \ \ \ \ \ \ \ }
\end{entry}

\subsection{{Predicates}}


\begin{entry}{%
  \Proto{vector?}{ x}{procedure}{boolean}}

  {[}\protect\hyperlink{R7RS-small}{\emph{R7RS-small}}{]} Disjoint
  type predicate for vectors: this returns \texttt{\#t} if \texttt{x}
  is a vector, and \texttt{\#f} if otherwise.  Examples:
  \texttt{~~~~~~~~~~~(vector?~'\#(a~b~c))~~~~~~~~~}\\
  \texttt{~~~~~~~~~~~\#t~~~~~~~~~}
  \texttt{~~~~~~~~~~~(vector?~'(a~b~c))~~~~~~~~~}\\
  \texttt{~~~~~~~~~~~\#f~~~~~~~~~}
  \texttt{~~~~~~~~~~~(vector?~\#t)~~~~~~~~~}\\
  \texttt{~~~~~~~~~~~\#f~~~~~~~~~}
  \texttt{~~~~~~~~~~~(vector?~'\#())~~~~~~~~~}\\
  \texttt{~~~~~~~~~~~\#t~~~~~~~~~}
  \texttt{~~~~~~~~~~~(vector?~'())~~~~~~~~~}\\
  \texttt{~~~~~~~~~~~\#f~~~~~~~~~}
\end{entry}

\begin{entry}{%
 \Proto{vector-empty?}{ vec}{procedure}{boolean}}

  Returns
  \texttt{\#t} if \texttt{vec} is empty, i.e. its length is
  \texttt{0}, and \texttt{\#f} if not.  Examples:
  \texttt{~~~~~~~~~~~(vector-empty?~'\#(a))~~~~~~~~~}\\
  \texttt{~~~~~~~~~~~\#f~~~~~~~~~}
  \texttt{~~~~~~~~~~~(vector-empty?~'\#(()))~~~~~~~~~}\\
  \texttt{~~~~~~~~~~~\#f~~~~~~~~~}
  \texttt{~~~~~~~~~~~(vector-empty?~'\#(\#()))~~~~~~~~~}\\
  \texttt{~~~~~~~~~~~\#f~~~~~~~~~}
  \texttt{~~~~~~~~~~~(vector-empty?~'\#())~~~~~~~~~}\\
  \texttt{~~~~~~~~~~~\#t~~~~~~~~~}
\end{entry}

\begin{entry}{%
  \Proto{vector=}{ elt=? vec \ldots}{procedure}{boolean}}

  Vector structure comparator, generalized across user-specified
  element comparators. Vectors \texttt{a} and \texttt{b} are
  considered equal by \texttt{vector=} iff their lengths are the same,
  and for each respective element \texttt{Ea} and \texttt{Eb},
  \texttt{(elt=?~Ea~~~~~~~~~Eb)} returns a true value.  \texttt{Elt=?}
  is always applied to two arguments. Element comparison must be
  consistent with \texttt{eq}; that is, if \texttt{(eq?~Ea~Eb)}
  results in a true value, then \texttt{(elt=?~Ea~~~~~~~~~Eb)} must
  also result in a true value. This may be exploited to avoid
  unnecessary element comparisons. (The reference implementation does,
  but it does not consider the situation where \texttt{elt=?} is in
  fact itself \texttt{eq?} to avoid yet more unnecessary comparisons.)
  If there are only zero or one vector arguments, \texttt{\#t} is
  automatically returned. The dynamic order in which comparisons of
  elements and of vectors are performed is left completely
  unspecified; do not rely on a particular order.  Examples:
  \texttt{~~~~~~~~~~~(vector=~eq?~'\#(a~b~c~d)~'\#(a~b~c~d))~~~~~~~~~}\\
  \texttt{~~~~~~~~~~~\#t~~~~~~~~~}
  \texttt{~~~~~~~~~~~(vector=~eq?~'\#(a~b~c~d)~'\#(a~b~d~c))~~~~~~~~~}\\
  \texttt{~~~~~~~~~~~\#f~~~~~~~~~}
  \texttt{~~~~~~~~~~~(vector=~=~'\#(1~2~3~4~5)~'\#(1~2~3~4))~~~~~~~~~}\\
  \texttt{~~~~~~~~~~~\#f~~~~~~~~~}
  \texttt{~~~~~~~~~~~(vector=~=~'\#(1~2~3~4)~'\#(1~2~3~4))~~~~~~~~~}\\
  \texttt{~~~~~~~~~~~\#t~~~~~~~~~} The two trivial cases.
  \texttt{~~~~~~~~~~~(vector=~eq?)~~~~~~~~~}\\
  \texttt{~~~~~~~~~~~\#t~~~~~~~~~}
  \texttt{~~~~~~~~~~~(vector=~eq?~'\#(a))~~~~~~~~~}\\
  \texttt{~~~~~~~~~~~\#t~~~~~~~~~} Note the fact that we don't use
  vector literals in the next two --- it is unspecified whether or not
  literal vectors with the same external representation are
  \texttt{eq?}.
  \texttt{~~~~~~~~~~~(vector=~eq?~(vector~(vector~'a))~(vector~(vector~'a)))~~~~~~~~~}\\
  \texttt{~~~~~~~~~~~\#f~~~~~~~~~}
  \texttt{~~~~~~~~~~~(vector=~equal?~(vector~(vector~'a))~(vector~(vector~'a)))~~~~~~~~~}\\
  \texttt{~~~~~~~~~~~\#t~~~~~~~~~}
\end{entry}

\subsection{Selectors}

\TODO{introduction}

\begin{entry}{%
  \Proto{vector-ref}{ vec i}{procedure}{value}}

  {[}\protect\hyperlink{R7RS-small}{\emph{R7RS-small}}{]} Vector
  element dereferencing: returns the value that the location in
  \texttt{vec} at \texttt{i} is mapped to in the store. Indexing is
  based on zero.  \texttt{I} must be within the range {[}0,
  \texttt{(vector-length~~~~~~~~~~~~~~vec)}).  Example:
  \texttt{~~~~~~~~~~~(vector-ref~'\#(a~b~c~d)~2)~~~~~~~~~}\\
  \texttt{~~~~~~~~~~~c~~~~~~~~~}
\end{entry}

\begin{entry}{%
  \Proto{vector-length}{ vec}{procedure}{exact nonnegative integer}} 

  {[}\protect\hyperlink{R7RS-small}{\emph{R7RS-small}}{]}
  Returns the length of \texttt{vec}, the number of locations
  reachable from \texttt{vec}. (The careful word `reachable' is used
  to allow for `vector slices,' whereby \texttt{vec} refers to a
  larger vector that contains more locations that are unreachable from
  \texttt{vec}. This SRFI does not define vector slices, but later
  SRFIs may.)  Example:
  \texttt{~~~~~~~~~~~(vector-length~'\#(a~b~c))~~~~~~~~~}\\
  \texttt{~~~~~~~~~~~3~~~~~~~~~}
\end{entry}

\subsection{Iteration}

\TODO{introduction}

\begin{entry}{%
   \Proto{vector-fold}{ kons knil vec$_1$ vec$_2$ \ldots}{procedure}{value}}

  The fundamental vector iterator. \texttt{Kons} is
  iterated over each value in all of the vectors, stopping at the end
  of the shortest; \texttt{kons} is applied as
  \texttt{~~~~~~~~~~~(kons~state~~~~~~~~~~~~~(vector-ref~~~~~~~~~~~~~~vec1~i)~~~~~~~~~~~~~(vector-ref~~~~~~~~~~~~~~vec2~i)~~~~~~~~~~~~~...)~~~~~~~~~}
  where \texttt{state} is the current state value --- the current
  state value begins with \texttt{knil}, and becomes whatever
  \texttt{kons} returned on the previous iteration ---, and \texttt{i}
  is the current index.  The iteration is strictly left-to-right.
  Examples: Find the
  longest string's length in \texttt{vector-of-strings}.\\
  \texttt{~~~~~~~~~~~(vector-fold~(λ~(len~str)~~~~~~~~~~~~~~~~~~~~~~~~~~(max~(string-length~str)~len))~~~~~~~~~~~~~~~~~~~~~~~~~~~~~~~~~~~~~~~~~~~~~~~~~~~~~~~~~~~0~vector-of-strings)~~~~~~~~~}
  Produce
  a list of the reversed elements of \texttt{vec}.\\
  \texttt{~~~~~~~~~~~(vector-fold~(λ~(tail~elt)~(cons~elt~tail))~~~~~~~~~~~~~~~~~~~~~~~~~~~~~~~~~~~~~~~~~~~~~~~~~~~~~~~~~~~'()~vec)~~~~~~~~~}
  Count
  the number of even numbers in \texttt{vec}.\\
  \texttt{~~~~~~~~~~~(vector-fold~(λ~(counter~n)~~~~~~~~~~~~~~~~~~~~~~~~~~~~~~~~~~~~~~~~~~~~~~~~~~~~~~~~~~~~~~~(if~(even?~n)~(+~counter~1)~counter))~~~~~~~~~~~~~~~~~~~~~~~~~~~~~~~~~~~~~~~~~~~~~~~~~~~~~~~~~~~0~vec)~~~~~~~~~}
\end{entry}

\begin{entry}{%
  \Proto{vector-fold-right}{ kons knil vec$_1$ vec$_2$ \ldots}{procedure}{value}} 

  Similar to \texttt{vector-fold}, but it
  iterates right to left instead of left to right.  Example: Convert a
  vector to a list.\\
  \texttt{~~~~~~~~~~~(vector-fold-right~(λ~(tail~elt)~~~~~~~~~~~~~~~~~~~~~~~~~~~~~~~~(cons~elt~tail))~~~~~~~~~~~~~~~~~~~~~~~~~~~~~~~~~~~~~~~~~~~~~~~~~~~~~~~~~~~~~~~~~~~~~~~'()~'\#(a~b~c~d))~~~~~~~~~}\\
  \texttt{~~~~~~~~~~~(a~b~c~d)~~~~~~~~~}
\end{entry}

\begin{entry}{%
  \Proto{vector-map}{ f vec$_1$ vec$_2$ \ldots}{procedure}{vector}}

  {[}\protect\hyperlink{R7RS-small}{\emph{R7RS-small}}{]}
  Constructs a new vector of the shortest size of the vector
  arguments. Each element at index \texttt{i} of the new vector is
  mapped from the old vectors by
  \texttt{(f~(vector-ref~~~~~~~~~~~~~~~vec1~~~~~~~~~~~~~~~i)~~~~~~~~~~~~~~(vector-ref~~~~~~~~~~~~~~~vec2~~~~~~~~~~~~~~~i)~~~~~~~~~~~~~~...)}.
  The dynamic order of application of \texttt{f} is unspecified.
  Examples:
  \texttt{~~~~~~~~~~~(vector-map~(λ~(x)~(*~x~x))~~~~~~~~~~~~~~~~~~~~~~~~~~~~~~~~~~~~~~~~~~~~~~~~~~~~~~~~~(vector-unfold~~~~~~~~~~~~~~~~~~~~~~~~(λ~(i~x)~(values~x~(+~x~1)))~~~~~~~~~~~~~~~~~~~~~~~~4~1))~~~~~~~~~}\\
  \texttt{~~~~~~~~~~~\#(1~4~9~16)~~~~~~~~~}
  \texttt{~~~~~~~~~~~(vector-map~(λ~(x~y)~(*~x~y))~~~~~~~~~~~~~~~~~~~~~~~~~~~~~~~~~~~(vector-unfold~~~~~~~~~~~~~~~~~~~~~~~~~~~~~~(λ~(x)~(values~x~(+~x~1)))~~~~~~~~~~~~~~~~~~~~~~~~~~~~~~5~1)~~~~~~~~~~~~~~~~~~~~~~~~~~~~~~~~~~~(vector-unfold~~~~~~~~~~~~~~~~~~~~~~~~~~~~~~(λ~(x)~(values~x~(-~x~1)))~~~~~~~~~~~~~~~~~~~~~~~~~~~~~~5~5))~~~~~~~~~}\\
  \texttt{~~~~~~~~~~~\#(5~8~9~8~5)~~~~~~~~~}
  \texttt{~~~~~~~~~~~(let~((count~0))~~~~~~~~~}\\
  \texttt{~~~~~~~~~~~~~~~~~~~~~~~~(vector-map~(λ~(ignored-elt)~~~~~~~~~}\\
  \texttt{~~~~~~~~~~~~~~~~~~~~~~~~~~~~~~~~~~~~~~~~~~~~~~~~(set!~count~(+~count~1))~~~~~~~~~}\\
  \texttt{~~~~~~~~~~~~~~~~~~~~~~~~~~~~~~~~~~~~~~~~~~~~~~~~count)~~~~~~~~~}\\
  \texttt{~~~~~~~~~~~~~~~~~~~~~~~~~~~~~~~~~~~~~~~~~~~~~~'\#(a~b)))~~~~~~~~~}\\
  \texttt{~~~~~~~~~~~\#(1~2)~OR~\#(2~1)~~~~~~~~~}
\end{entry}

\begin{entry}{%
  \Proto{vector-map!}{ f vec$_1$ vec$_2$
    \ldots}{procedure}{unspecified}}

  Similar to \texttt{vector-map}, but rather than
  mapping the new elements into a new vector, the new mapped elements
  are destructively inserted into \texttt{vec1}. Again, the dynamic
  order of application of \texttt{f} unspecified, so it is dangerous
  for \texttt{f} to apply either \texttt{vector-ref} or
  \texttt{vector-set!} to \texttt{vec1} in
  \texttt{f}.\\
\end{entry}

\begin{entry}{%
  \Proto{vector-for-each}{ f vec$_1$ vec$_2$
    \ldots}{procedure}{unspecified}}

  {[}\protect\hyperlink{R7RS-small}{\emph{R7RS-small}}{]} Simple
  vector iterator: applies \texttt{f} to the corresponding list of
  parallel elements from \texttt{vec1~vec2~~~~~~~~~...} in the range
  {[}0, \texttt{length}), where \texttt{length} is the length of the
  smallest vector argument passed, In contrast with
  \texttt{vector-map}, \texttt{f} is reliably applied to each
  subsequent element, starting at index 0, in the vectors.  Example:
  \texttt{~~~~~~~~~~~(vector-for-each~(λ~(x)~(display~x)~(newline))~~~~~~~~~}\\
  \texttt{~~~~~~~~~~~~~~~~~~~~~~~~~~~~~~~~~~~~~~~~~~~~~~~~'\#("foo"~"bar"~"baz"~"quux"~"zot"))~~~~~~~~~}\\
  Displays:\\

\begin{verbatim}
foo
bar
baz
quux
zot
\end{verbatim}
\end{entry}

\begin{entry}{%
  \Proto{vector-count}{ pred? vec$_1$ vec$_2$ \ldots}{procedure}{exact
    nonnegative integer}}

  Counts the number of parallel
  elements in the vectors that satisfy \texttt{pred?}, which is
  applied, for each index \texttt{i} in the range {[}0,
  \texttt{length}) where \texttt{length} is the length of the smallest
  vector argument, to each parallel element in the vectors, in order.
  Examples:
  \texttt{~~~~~~~~~~~(vector-count~even?~~~~~~~~~~~~~~~~~~~~~~~~~'\#(3~1~4~1~5~9~2~5~6))~~~~~~~~~}\\
  \texttt{~~~~~~~~~~~3~~~~~~~~~}
  \texttt{~~~~~~~~~~~(vector-count~\textless{}~~~~~~~~~~~~~~~~~~~~~~~~~'\#(1~3~6~9)~'\#(2~4~6~8~10~12))~~~~~~~~~}\\
  \texttt{~~~~~~~~~~~2~~~~~~~~~}
\end{entry}

\begin{entry}{%
  \Proto{vector-cumulate}{ f knil vec}{procedure}{vector}}

  Returns a newly allocated vector \texttt{new} with the same length
  as \texttt{vec}. Each element \emph{i} of \emph{new} is set to the
  result of invoking \texttt{f} on
  \emph{\texttt{new}\textsubscript{i-1}} and
  \emph{\texttt{vec}\textsubscript{i}}, except that for the first call
  on \emph{f}, the first argument is \emph{knil}. The \emph{new}
  vector is returned.

  Note that the order of arguments to \texttt{vector-cumulate} was
  changed by \texttt{errata-3} on 2016/9/2.

  Example:
  \texttt{~~~~~~~~~~~(vector-cumulate~+~0~'\#(3~1~4~1~5~9~2~5~6))~~~~~~~~~}\\
  \texttt{~~~~~~~~~~~\#(3~4~8~9~14~23~25~30~36)~~~~~~~~~}\\[4\baselineskip]
\end{entry}

\subsection{Searching}

\begin{entry}{%
  \Proto{vector-index}{ pred? vec$_1$ vec$_2$ \ldots}{procedure}{exact
    nonnegative integer or \#f}}

  Finds \& returns the index of
  the first elements in \texttt{vec1~vec2~~~~~~~~~...} that satisfy
  \texttt{pred?}. If no matching element is found by the end of the
  shortest vector, \texttt{\#f} is returned.  Examples:
  \texttt{~~~~~~~~~~~(vector-index~even?~'\#(3~1~4~1~5~9))~~~~~~~~~}\\
  \texttt{~~~~~~~~~~~2~~~~~~~~~}
  \texttt{~~~~~~~~~~~(vector-index~\textless{}~'\#(3~1~4~1~5~9~2~5~6)~'\#(2~7~1~8~2))~~~~~~~~~}\\
  \texttt{~~~~~~~~~~~1~~~~~~~~~}
  \texttt{~~~~~~~~~~~(vector-index~=~'\#(3~1~4~1~5~9~2~5~6)~'\#(2~7~1~8~2))~~~~~~~~~}\\
  \texttt{~~~~~~~~~~~\#f~~~~~~~~~}
\end{entry}

\begin{entry}{%
  \Proto{vector-index-right}{ pred? vec$_1$ vec$_2$
    \ldots}{procedure}{exact nonnegative integer or \#f}}

  Like
  \texttt{vector-index}, but it searches right-to-left, rather than
  left-to-right, and all of the vectors \emph{must} have the same
  length.
\end{entry}

\begin{entry}{%
  \Proto{vector-skip}{ pred? vec$_1$ vec$_2$ \ldots}{procedure}{exact
    nonnegative integer or \#f}}

  Finds \& returns the index of
  the first elements in \texttt{vec1~vec2~~~~~~~~~...} that do
  \emph{not} satisfy \texttt{pred?}. If all the values in the vectors
  satisfy \texttt{pred?}  until the end of the shortest vector, this
  returns \texttt{\#f}. This is equivalent to:
  \texttt{~~~~~~~~~~~(vector-index~~~~~~~~~~~~(λ~(x1~x2~~~~~~~~~~~~~~~~~~~~~~~~...)~~~~~~~~~~~~~~(not~(pred?~x1~~~~~~~~~~~~~~~~~~~~~~~~~~~~~~~~~x1~~~~~~~~~~~~~~~~~~~~~~~~~~~~~~~~~~...)))~~~~~~~~~~~~~~~~~~~~~~~~~~~~~~~~~~~~~~~~~~~~~~~~~~~~~~vec1~vec2~~~~~~~~~~~~...)~~~~~~~~~}
  Example:
  \texttt{~~~~~~~~~~~(vector-skip~number? '\#(1~2~a~b~3~4~c~d))~~~~~~~~~}\\
  \texttt{~~~~~~~~~~~2~~~~~~~~~}
\end{entry}

\begin{entry}{%
  \Proto{vector-skip-right}{ pred? vec$_1$ vec$_2$
    \ldots}{procedure}{exact nonnegative integer or \#f}}

  Like
  \texttt{vector-skip}, but it searches for a non-matching element
  right-to-left, rather than left-to-right, and it is an error if all
  of the vectors do not have the same length. This is equivalent to:
  \texttt{~~~~~~~~~~~(vector-index-right~~~~~~~~~~~~(λ~(x1~x2~~~~~~~~~~~~~~~~~~~~~~~~...)~~~~~~~~~~~~~~(not~(pred?~x1~~~~~~~~~~~~~~~~~~~~~~~~~~~~~~~~~x1~~~~~~~~~~~~~~~~~~~~~~~~~~~~~~~~~~...)))~~~~~~~~~~~~~~~~~~~~~~~~~~~~~~~~~~~~~~~~~~~~~~~~~~~~~~~~~~~~~~~~~~~~~~~vec1~vec2~~~~~~~~~~~~...)~~~~~~~~~}
\end{entry}

\begin{entry}{%
  \Proto{vector-binary-search}{ vec value cmp}{procedure}{exact
    nonnegative integer or \#f}}

  Similar to \texttt{vector-index} and
  \texttt{vector-index-right}, but instead of searching left to right
  or right to left, this performs a binary search. If there is more
  than one element of \emph{vec} that matches \emph{value} in the
  sense of \emph{cmp}, \texttt{vector-binary-search} may return the
  index of any of them.

  \texttt{cmp} should be a procedure of two arguments and return a
  negative integer, which indicates that its first argument is less
  than its second, zero, which indicates that they are equal, or a
  positive integer, which indicates that the first argument is greater
  than the second argument. An example \texttt{cmp} might be:
  \texttt{~~~~~~~~~~~(λ~(char1~char2)~~~~~~~~~}\\
  \texttt{~~~~~~~~~~~~~(cond~((char\textless{}?~char1~~~~~~~~~~~~~~~~~~~~~~~~~~~~~~~~~~~~~~~~~~~~char2)~~~~~~~~~~~~~~~~~~~~~~~~~~~~~~-1)~~~~~~~~~}\\
  \texttt{~~~~~~~~~~~~~~~~~~~~~~~~~~~~~((char=?~char1~~~~~~~~~~~~~~~~~~~~~~~char2)~~~~~~~~~~~~0)~~~~~~~~~}\\
  \texttt{~~~~~~~~~~~~~~~~~~~~~~~~~~~~~(else~1)))~~~~~~~~~}
\end{entry}

\begin{entry}{%
  \Proto{vector-any}{ pred? vec$_1$ vec$_2$ \ldots}{procedure}{value
    or \#f}}

  Finds the first set of elements in parallel from
  \texttt{vec1~vec2~~~~~~~~~...} for which \texttt{pred?}  returns a
  true value. If such a parallel set of elements exists,
  \texttt{vector-any} returns the value that \texttt{pred?} returned
  for that set of elements. The iteration is strictly left-to-right.
\end{entry}

\begin{entry}{%
  \Proto{vector-every}{ pred? vec$_1$ vec$_2$ \ldots}{procedure}{value
    or \#f}}

  If, for every index \texttt{i} between 0 and the
  length of the shortest vector argument, the set of elements
  \texttt{(vector-ref~vec1~~~~~~~~~~~~~~~~~~~~~~~~~~~~~~~~~~~~~~~~~~~~~~~~~~i)~~~~~~~~~~~~~(vector-ref~vec2~~~~~~~~~~~~~~~~~~~~~~~~~~~~~~~~~~~~~~~~~~~~~~~~~~i)~~~~~~~~~~~~~...}
  satisfies \texttt{pred?}, \texttt{vector-every} returns the value
  that \texttt{pred?} returned for the last set of elements, at the
  last index of the shortest vector. The iteration is strictly
  left-to-right.
\end{entry}

\begin{entry}{%
  \Proto{vector-partition}{ pred? vec}{procedure}{vector}}
 
  A vector the same size as \texttt{vec} is newly allocated and filled
  with all the elements of \texttt{vec} that satisfy \texttt{pred?} in
  their original order followed by all the elements that do not
  satisfy \texttt{pred}, also in their original order.  Two values are
  returned, the newly allocated vector and the index of the leftmost
  element that does not satisfy \texttt{pred}.
\end{entry}

\subsection{Mutators}


\begin{entry}{%
  \Proto{vector-set!}{ vec i value}{procedure}{unspecified}}

  {[}\protect\hyperlink{R7RS-small}{\emph{R7RS-small}}{]} Assigns the
  contents of the location at \texttt{i} in \texttt{vec} to
  \texttt{value}.
\end{entry}

\begin{entry}{%
  \Proto{vector-swap!}{ vec i j}{procedure}{unspecified}}

  Swaps or exchanges the values of the locations in \texttt{vec} at
  \texttt{i} \& \texttt{j}.
\end{entry}

\begin{entry}{%
  \Proto{vector-fill!}{ vec fill start end}{procedure}{unspecified}
  \Rproto{vector-fill!}{ vec fill start}{procedure}{unspecified}
  \Rproto{vector-fill!}{ vec fill}{procedure}{unspecified}}

  {[}\protect\hyperlink{R7RS-small}{\emph{R7RS-small}}{]} Assigns the
  value of every location in \texttt{vec} between \texttt{start},
  which defaults to \texttt{0} and \texttt{end}, which defaults to the
  length of \texttt{vec}, to \texttt{fill}.
\end{entry}

\begin{entry}{%
  \Proto{vector-reverse!}{ vec start end}{procedure}{unspecified}
  \Rproto{vector-reverse!}{ vec start}{procedure}{unspecified}
  \Rproto{vector-reverse!}{ vec}{procedure}{unspecified}}

  Destructively reverses the contents of the sequence of locations in
  \texttt{vec} between \texttt{start} and \texttt{end}. \texttt{Start}
  defaults to \texttt{0} and \texttt{end} defaults to the length of
  \texttt{vec}. Note that this does not deeply reverse.
\end{entry}

\begin{entry}{%
  \Proto{vector-copy!}{ to at from start end}{procedure}{unspecified}
  \Rproto{vector-copy!}{ to at from start}{procedure}{unspecified}
  \Rproto{vector-copy!}{ to at from}{procedure}{unspecified}}

  {[}\protect\hyperlink{R7RS-small}{\emph{R7RS-small}}{]} Copies the
  elements of vector \texttt{from} between \texttt{start} and
  \texttt{end} to vector \texttt{to}, starting at \texttt{at}. The
  order in which elements are copied is unspecified, except that if
  the source and destination overlap, copying takes place as if the
  source is first copied into a temporary vector and then into the
  destination. This can be achieved without allocating storage by
  making sure to copy in the correct direction in such circumstances.
\end{entry}

\begin{entry}{%
  \Proto{vector-reverse-copy!}{ to at from start
    end}{procedure}{unspecified}
  \Rproto{vector-reverse-copy!}{ to at from
    start}{procedure}{unspecified}
  \Rproto{vector-reverse-copy!}{ to at from}{procedure}{unspecified}}

  Like \texttt{vector-copy!}, but the elements appear in \texttt{to}
  in reverse order.
  \texttt{~~~~~~~~~~~(vector-reverse!~~~~~~~~~~~~target~~~~~~~~~~~~tstart~~~~~~~~~~~~send)~~~~~~~~~
  } would.
\end{entry}

\begin{entry}{%
  \Proto{vector-unfold!}{ f vec start end initial-seed \ldots}{procedure}{unspecified}}

  Like \texttt{vector-unfold}, but the elements are copied into the
  vector \emph{vec} starting at element \emph{start} rather than into
  a newly allocated vector. Terminates when \emph{end-start} elements
  have been generated.
\end{entry}

\begin{entry}{%
  \Proto{vector-unfold-right!}{ f vec start end initial-seed \ldots}{procedure}{unspecified}}

  Like \texttt{vector-unfold!}, but the elements are copied in reverse
  order into the vector \emph{vec} starting at the index preceding
  \emph{end}.
\end{entry}

\subsection{Conversion}

\begin{entry}{%
  \Proto{vector->list}{vec start end}{procedure}{proper-list} 
  \Rproto{vector->list}{vec start}{procedure}{proper-list} 
  \Rproto{vector->list}{vec}{procedure}{proper-list}}

  {[}\protect\hyperlink{R7RS-small}{\emph{R7RS-small}}{]} Creates a
  list containing the elements in \texttt{vec} between \texttt{start},
  which defaults to \texttt{0}, and \texttt{end}, which defaults to
  the length of \texttt{vec}.
\end{entry}

\begin{entry}{%
  \Proto{reverse-vector->list}{ vec start end}{procedure}{proper-list}
  \Rproto{reverse-vector->list}{ vec start}{procedure}{proper-list}
  \Rproto{reverse-vector->list}{ vec}{procedure}{proper-list}}

] Like \texttt{vector->list},
  but the resulting list contains the elements in reverse of
  \texttt{vector}.
\end{entry}

\begin{entry}{%
  \Proto{list->vector}{ proper-list}{procedure}{vector}}

  {[}\protect\hyperlink{R7RS-small}{\emph{R7RS-small}}{]} Creates a
  vector of elements from \texttt{proper-list}.
\end{entry}

\begin{entry}{%
  \Proto{reverse-list->vector}{ proper-list}{procedure}{vector}}

  Like \texttt{list->vector}, but the resulting vector contains the
  elements in reverse of \texttt{proper-list}.
\end{entry}

\begin{entry}{%
  \Proto{string->vector}{ string start end}{procedure}{vector} 
  \Rproto{string->vector}{ string start}{procedure}{vector} 
  \Rproto{string->vector}{ string}{procedure}{vector}}

  {[}\protect\hyperlink{R7RS-small}{\emph{R7RS-small}}{]} Creates a
  vector containing the elements in \texttt{string} between
  \texttt{start}, which defaults to \texttt{0}, and \texttt{end},
  which defaults to the length of \texttt{string}.
\end{entry}

\begin{entry}{%
  \Proto{vector->string}{ vec start end}{procedure}{string}
  \Rproto{vector->string}{ vec start}{procedure}{string} 
  \Rproto{vector->string}{ vec}{procedure}{string}}

  {[}\protect\hyperlink{R7RS-small}{\emph{R7RS-small}}{]} Creates a
  string containing the elements in \texttt{vec} between
  \texttt{start}, which defaults to \texttt{0}, and \texttt{end},
  which defaults to the length of \texttt{vec}. It is an error if the
  elements are not characters.
\end{entry}


\subsection{References}

\TODO{move to consolidated reference list.}

\begin{description}
\tightlist
\item[\href{}{R5RS}]
\emph{R5RS: The Revised\textsuperscript{5} Report on Scheme}\\
R. Kelsey, W. Clinger, J. Rees (editors).\\
Higher-Order and Symbolic Computation, Vol. 11, No. 1, September, 1998\\
and\\
ACM SIGPLAN Notices, Vol. 33, No. 9, October, 1998\\
Available at:
\url{http://www.schemers.org/Documents/Standards/R5RS/}

\item[\href{}{R7RS-small}]
\emph{R7RS: The Revised\textsuperscript{7} Report on Scheme}\\
A. Shinn et al. (editors).\\
Available at: \url{http://r7rs.org}

\item[\href{}{SRFI}]
\emph{SRFI: Scheme Request for Implementation}\\
The SRFI website can be found at: \url{http://srfi.schemers.org/}\\
The SRFIs mentioned in this document are described
later.

\item[\href{}{SRFI 1}]
\emph{SRFI 1: List Library}\\
A SRFI of list processing procedures, written by Olin Shivers.\\
Available at: \url{http://srfi.schemers.org/srfi-1/}

\item[\href{}{SRFI 13}]
\emph{SRFI 13: String Library}\\
A SRFI of string processing procedures, written by Olin Shivers.\\
Available at: \url{http://srfi.schemers.org/srfi-13/}

\item[\href{}{SRFI 23}]
\emph{SRFI 23: Error Reporting Mechanism}\\
A SRFI that defines a new primitive (\texttt{error}) for reporting that
an error occurred, written by Stephan Houben.\\
Available at: \url{http://srfi.schemers.org/srfi-23/}
 
\item[\href{}{SRFI 43}]
\emph{SRFI 43: Vector Library (draft)}\\
The direct predecessor of this SRFI, written by Taylor Campbell.\\
Available at: \url{http://srfi.schemers.org/srfi-43/}
\end{description}

\begin{center}\rule{0.5\linewidth}{\linethickness}\end{center}

Editor:
\href{mailto:srfi\%20minus\%20editors\%20at\%20srfi\%20dot\%20schemers\%20dot\%20org}{Arthur
A. Gleckler}
