\chapter{Sorting library}
\redno{4} What sorting library should R7RS-large provide?

\begin{quote}
SRFI 132 is an edited version of the withdrawn SRFI 32, with the
algorithm-specific procedures removed and a few additional procedures;
it provides sortedness testing, stable and unstable sorting, merging,
duplicate removal, median, and selection. R6RS provides only stable
sorting and merging.
\end{quote}

This SRFI describes the API for a full-featured sort toolkit.
\setlibname{sort}

\subsection{Procedure naming and
functionality}\label{Procedurenamingandfunctionality}

Most of the procedures described below are variants of two basic
operations: sorting and merging. These procedures are consistently named
by composing a set of basic lexemes to indicate what they do.

\begin{tabular}{ll}
Lexeme & Meaning\tabularnewline
\texttt{vector} & The procedure operates upon vectors.\tabularnewline
\texttt{list} & The procedure operates upon lists.\tabularnewline
\texttt{stable} & This lexeme indicates that the sort is a stable
one.\tabularnewline
\texttt{sort} & The procedure sorts its input data set by some ordering
function.\tabularnewline
\texttt{merge} & The procedure merges two ordered data sets into a
single ordered result.\tabularnewline
\texttt{!} & Procedures that end in \texttt{!} are allowed, and
sometimes required, to reuse their input storage to construct their
answer.\tabularnewline
\end{tabular}

\subsection{Types of parameters and return
values}\label{Typesofparametersandreturnvalues}

In the procedures specified below:

\begin{itemize}
\tightlist
\item
  A \emph{lis} parameter is a list.
\end{itemize}

\begin{itemize}
\tightlist
\item
  A \emph{v} parameter is a vector.
\end{itemize}

\begin{itemize}
\tightlist
\item
  An \emph{=} parameter is an equality predicate. See
  \href{http://srfi.schemers.org/srfi-128/srfi-128.html}{SRFI 128} for
  the requirements on equality predicates. Note that neither this SRFI
  nor its sample implementation depend on SRFI 128.
\end{itemize}

\begin{itemize}
\tightlist
\item
  A \emph{\textless{}} parameter is an ordering predicate. See
  \href{http://srfi.schemers.org/srfi-128/srfi-128.html}{SRFI 128} for
  the requirements on ordering predicates. \TODO{use le? as a clearer
    name.}

\end{itemize}

\begin{itemize}
\tightlist
\item
  A \emph{start} parameter or \emph{start} and \emph{end} parameter pair
  are exact non-negative integers such that 0 \textless{}= \emph{start}
  \textless{}= \emph{end} \textless{}=
  \texttt{(vector-length\ }\emph{v}\texttt{)}, where \emph{v} is the
  related vector parameter. If not specified, they default to 0 and the
  length of the vector, respectively. They are interpreted to select the
  range {[}\emph{start}, \emph{end}), that is, all elements from index
  \emph{start} (inclusive) up to, but not including, index \emph{end}.
\end{itemize}

Passing values to procedures with these parameters that do not satisfy
these constraints is an error.

If a procedure is said to return ``an unspecified value'', this means
that nothing at all is said about what the procedure returns, except
that it returns one value.

\subsection{Predicates}

\TODO{Change textless to le? throughout this section.}


\begin{entry}{%
  \Proto{list-sorted?}{ \textless{} lis}{procedure}{boolean}
  \Proto{vector-sorted?}{ \textless{} v start end}{procedure}{boolean}
  \Rproto{vector-sorted?}{ \textless{} v start}{procedure}{boolean}
  \Rproto{vector-sorted?}{ \textless{} v}{procedure}{boolean}}

  These procedures return true iff their input list or vector is in
  sorted order, as determined by \emph{\textless{}}. Specifically,
  they return \texttt{\#f} iff there is an adjacent pair \ldots{} X Y
  \ldots{} in the input list or vector such that Y \textless{} X in
  the sense of \emph{\textless{}}. The optional \emph{start} and
  \emph{end} range arguments restrict \texttt{vector-sorted?} to
  examining the indicated subvector.

  These procedures are equivalent to the SRFI 95 \texttt{sorted?}
  procedure when applied to lists or vectors respectively, except that
  they do not accept a key procedure.
\end{entry}
\subsection{General sort procedures}\label{Generalsortprocedures}

These procedures provide basic sorting and merging functionality
suitable for general programming. The procedures are named by their
semantic properties, i.e., what they do to the data (sort, stable sort,
and so forth).


\begin{entry}{%
  \Proto{list-sort}{ \textless{} lis}{procedure}{list???}
  \Proto{list-stable-sort}{ \textless{} lis}{procedure}{list???}}

  These procedures do not alter their inputs, but are allowed to
  return a value that shares a common tail with a list argument.

  The \texttt{list-stable-sort} procedure is equivalent to the R6RS
  \texttt{list-sort} procedure. It is also equivalent to the SRFI 95
  \texttt{sort} procedure when applied to lists, except that it does
  not accept a key procedure.
\end{entry}

\begin{entry}{%
  \Proto{list-sort!}{ \textless{} lis}{procedure}{list???}
  \Proto{list-stable-sort!}{ \textless{} lis}{procedure}{list???}}

  These procedures are linear update operators --- they are allowed,
  but not required, to alter the cons cells of their arguments to
  produce their results. They return a sorted list containing the same
  elements as \emph{lis}.

  The \texttt{list-stable-sort!} procedure is equivalent to the SRFI
  95 \texttt{sort!} procedure when applied to lists, except that it
  does not accept a key procedure.
\end{entry}


\begin{entry}{%
  \Proto{vector-sort}{ \textless{} v start end}{procedure}{vector???}
  \Rproto{vector-sort}{ \textless{} v start}{procedure}{vector???}
  \Rproto{vector-sort}{ \textless{} v end}{procedure}{vector???}
  \Proto{vector-stable-sort}{ \textless{} v start end}{procedure}{vector???}
  \Rproto{vector-stable-sort}{ \textless{} v start}{procedure}{vector???}
  \Rproto{vector-stable-sort}{ \textless{} v}{procedure}{vector???}}

  These procedures do not alter their inputs, but allocate a fresh
  vector as their result, of length \emph{end} - \emph{start}. The
  \texttt{vector-stable-sort} procedure with no optional arguments is
  equivalent to the R6RS \texttt{vector-sort} procedure. It is also
  equivalent to the SRFI 95 \texttt{sort} procedure when applied to
  vectors, except that it does not accept a key procedure.
\end{entry}

\begin{entry}{%
  \Proto{vector-sort!}{ \textless{} v start end}{procedure}{vector???}
  \Rproto{vector-sort!}{ \textless{} v start}{procedure}{vector???}
  \Rproto{vector-sort!}{ \textless{} v}{procedure}{vector???} 
  \Proto{vector-stable-sort!}{ \textless{} v start end}{procedure}{vector???}
  \Rproto{vector-stable-sort!}{ \textless{} v start}{procedure}{vector???}
  \Rproto{vector-stable-sort!}{ \textless{} v}{procedure}{vector???} }

  These procedures sort their data in-place. (But note that
  \texttt{vector-stable-sort!} may allocate temporary storage
  proportional to the size of the input --- there are no known O(n lg
  n) stable vector sorting algorithms that run in constant space.)
  They return an unspecified value.

  The \texttt{vector-sort!} procedure with no optional arguments is
  equivalent to the R6RS \texttt{vector-sort!} procedure.
\end{entry}

\subsection{Merge procedures}\label{Mergeprocedures}

All four merge operations are stable: an element of the initial list
\emph{lis$_1$} or vector \emph{v$_1$} will
come before an equal-comparing element in the second list
\emph{lis$_2$} or vector \emph{v$_2$} in the
result.

\begin{entry}{%
  \Proto{list-merge}{ \textless{} lis$_1$ lis$_2$}{procedure}{list???}}

  This procedure does not alter its inputs, and is allowed to return a
  value that shares a common tail with a list argument.

  This procedure is equivalent to the SRFI 95 \texttt{merge} procedure
  when applied to lists, except that it does not accept a key
  procedure.
\end{entry}

\begin{entry}{%
  \Proto{list-merge!}{ \textless{} lis$_1$ lis$_2$}{procedure}{list???}}

  This procedure makes only a single, iterative, linear-time pass over
  its argument lists, using \texttt{set-cdr!}s to rearrange the cells
  of the lists into the list that is returned --- it works ``in
  place.'' Hence, any cons cell appearing in the result must have
  originally appeared in an input. It returns the sorted input.

  Additionally, \texttt{list-merge!} is iterative, not recursive ---
  it can operate on arguments of arbitrary size without requiring an
  unbounded amount of stack space. The intent of this
  iterative-algorithm commitment is to allow the programmer to be sure
  that if, for example, \texttt{list-merge!} is asked to merge two
  ten-million-element lists, the operation will complete without
  performing some extremely (possibly twenty-million) deep recursion.

  This procedure is equivalent to the SRFI 95 \texttt{merge!}
  procedure when applied to lists, except that it does not accept a
  key procedure.
\end{entry}

\begin{entry}{%
  \Proto{vector-merge}{ \textless{} v$_1$ v$_2$ start$_1$ end$_1$
    start$_2$ end$_2$}{procedure}{vector???}
  \Rproto{vector-merge}{ \textless{} v$_1$ v$_2$ start$_1$ end$_1$
    start$_2$}{procedure}{vector???}
  \Proto{vector-merge}{ \textless{} v$_1$ v$_2$ start$_1$ end$_1$}{procedure}{vector???}
  \Proto{vector-merge}{ \textless{} v$_1$ v$_2$ start$_1$}{procedure}{vector???}
  \Proto{vector-merge}{ \textless{} v$_1$ v$_2$}{procedure}{vector???}}

  This procedure does not alter its inputs, and returns a newly
  allocated vector of length (\emph{end$_1$} - \emph{start$_1$}) +
  (\emph{end$_2$} - \emph{start$_2$}\texttt{)}.

This procedure is equivalent to the SRFI 95 \texttt{merge} procedure
when applied to vectors, except that it does not accept a key procedure.
\end{entry}

\begin{entry}{%
  \Proto{vector-merge!}{ \textless{} to from$_1$ from$_2$ start
    start$_1$ end$_1$ start$_2$ end$_2$}{procedure}{vector???}
  \Rproto{vector-merge!}{ \textless{} to from$_1$ from$_2$ start
    start$_1$ end$_1$ start$_2$}{procedure}{vector???}
  \Rproto{vector-merge!}{ \textless{} to from$_1$ from$_2$ start
    start$_1$ end$_1$}{procedure}{vector???}
  \Rproto{vector-merge!}{ \textless{} to from$_1$ from$_2$ start
    start$_1$}{procedure}{vector???}
  \Rproto{vector-merge!}{ \textless{} to from$_1$ from$_2$ start}{procedure}{vector???}
  \Rproto{vector-merge!}{ \textless{} to from$_1$ from$_2$}{procedure}{vector???}}

  This procedure writes its result into vector \emph{to}, beginning at
  index \emph{start}, for indices less than \emph{end}, which is
  defined as \emph{start} + (\emph{end$_1$} - \emph{start$_1$}) +
  (\emph{end$_2$} - \emph{start$_2$}\texttt{)}. The target subvector
  \emph{to}{[}\emph{start}, \emph{end}) may not overlap either of the
  source subvectors \emph{from$_1$}{[}\emph{start$_1$},
  \emph{end$_1$}{]} and \emph{from$_2$}{[}\emph{start$_2$},
  \emph{end$_2$}{]}. It returns an unspecified value.

  This procedure is equivalent to the SRFI 95 \texttt{merge!}
  procedure when applied to lists, except that it does not accept a
  key procedure.
\end{entry}

\subsection{Deleting duplicate neighbors}

These procedures delete adjacent duplicate elements from a list or a
vector, using a given element-equality procedure. The first/leftmost
element of a run of equal elements is the one that survives. The list or
vector is not otherwise disordered.

These procedures are linear time --- much faster than the
O(n\textsuperscript{2}) general duplicate-element deletion procedures
that do not assume any ``bunching'' of elements provided by
\href{http://srfi.schemers.org/srfi-1/srfi-1.html}{SRFI 1}. If you want
to delete duplicate elements from a large list or vector, sort the
elements to bring equal items together, then use one of these
procedures, for a total time of O(n lg n).

The equality procedure is always invoked as \texttt{(=\ x\ y)}, where
\emph{x} comes before \emph{y} in the containing list or vector.

\begin{entry}{%
  \Proto{list-delete-neighbor-dups}{ = lis}{procedure}{list???}}

  This procedure does not alter its input list, but its result may
  share storage with the input list.
\end{entry}

\begin{entry}{%
  \Proto{list-delete-neighbor-dups!}{ = lis}{procedure}{list???}}

  This procedure mutates its input list in order to construct its
  result.  It makes only a single, iterative, linear-time pass over
  its argument, using \texttt{set-cdr!}s to rearrange the cells of the
  list into the final result --- it works ``in place.'' Hence, any
  cons cell appearing in the result must have originally appeared in
  the input.
\end{entry}

\begin{entry}{%
  \Proto{vector-delete-neighbor-dups}{ = v start
    end}{procedure}{vector???}
  \Rproto{vector-delete-neighbor-dups}{ = v
    start}{procedure}{vector???}
  \Rproto{vector-delete-neighbor-dups}{ = v}{procedure}{vector???}}

  This procedure does not alter its input vector, but rather newly
  allocates and returns a vector to hold the result.
\end{entry}


\begin{entry}{%
  \Proto{vector-delete-neighbor-dups!}{ = v start
    end}{procedure}{vector???}
  \Rproto{vector-delete-neighbor-dups!}{ = v
    start}{procedure}{vector???}
  \Rproto{vector-delete-neighbor-dups!}{ = v}{procedure}{vector???}}

  This procedure reuses its input vector to hold the answer, packing
  it into the index range {[}\emph{start, newend}), where
  \emph{newend} is the non-negative exact integer that is returned as
  its value. The vector is not altered outside the range
  {[}\emph{start, newend}).
\end{entry}
Examples:

\begin{verbatim}
  (list-delete-neighbor-dups = '(1 1 2 7 7 7 0 -2 -2))
               => (1 2 7 0 -2)

    (vector-delete-neighbor-dups = '#(1 1 2 7 7 7 0 -2 -2))
               => #(1 2 7 0 -2)

    (vector-delete-neighbor-dups < '#(1 1 2 7 7 7 0 -2 -2) 3 7))
               => #(7 0 -2)

;; Result left in v[3,9):
(let ((v (vector 0 0 0 1 1 2 2 3 3 4 4 5 5 6 6)))
  (cons (vector-delete-neighbor-dups! < v 3)
        v))
              => (9 . #(0 0 0 1 2 3 4 5 6 4 4 5 5 6 6))
\end{verbatim}

\subsection{Finding the median}\label{Findingthemedian}

These procedures do not have SRFI 32 counterparts. They find the median
element of a vector after sorting it in accordance with an ordering
procedure. If the number of elements in \texttt{v} is odd, the
middlemost element of the sorted result is returned. If the number of
elements is zero, \texttt{knil} is returned. Otherwise, \texttt{mean} is
applied to the two middlemost elements in the order in which they appear
in \texttt{v}, and whatever it returns is returned. If \texttt{mean} is
omitted, then the default mean procedure is
\texttt{(lambda\ (a\ b)\ (/\ (+\ a\ b)\ 2)}, but this procedure is
applicable to non-numeric values as well.

\begin{entry}{%
  \Proto{vector-find-median}{ \textless{} v knil
    mean}{procedure}{value???}
  \Rproto{vector-find-median}{ \textless{} v knil}{procedure}{value???}}

  This procedure does not alter its input vector, but rather newly
  allocates a vector to hold the intermediate result. Runs in $O(n)$
  time.
\end{entry}

\begin{entry}{%
  \Proto{vector-find-median!}{ \textless{} v knil
    mean}{procedure}{value???}
  \Rproto{vector-find-median!}{ \textless{} v knil}{procedure}{value???}}

  This procedure reuses its input vector to hold the intermediate
  result, leaving it sorted, but is otherwise the same as
  \texttt{vector-find-median}. Runs in $O(n\ln{} n)$
  time. \TODO{Should that be $O(n\lg{} n)$? And aren't there median
    algorithms that run in $O(n)$ time?}
\end{entry}

\subsection{Selection}\label{Selection}

These procedures do not have SRFI 32 counterparts.

\begin{entry}{%
  \Proto{vector-select!}{ \textless{} v k start end}{procedure}{value???}
  \Rproto{vector-select!}{ \textless{} v k start}{procedure}{value???}
  \Rproto{vector-select!}{ \textless{} v k}{procedure}{value???}}

  This procedure returns the \emph{k}th smallest element (in the sense
  of the \textless{} argument) of the region of a vector between
  \emph{start} and \emph{end}. Elements within the range may be
  reordered, whereas those outside the range are left alone. Runs in
  $O(n)$ time.
\end{entry}


\begin{entry}{%
  \Proto{vector-separate!}{ \textless{} v k start
    end}{procedure}{unspecified}  
  \Rproto{vector-separate!}{ \textless{} v k start}{procedure}{unspecified}  
  \Rproto{vector-separate!}{ \textless{} v k}{procedure}{unspecified}}

This procedure places the smallest \emph{k} elements (in the sense of
the \textless{} argument) of the region of a vector between
\emph{start} and \emph{end} into the first \emph{k} positions of that
range, and the remaining elements into the remaining
positions. Otherwise, the elements are not in any particular
order. Elements outside the range are left alone. Runs in $O(n)$
time. Returns an unspecified value.
\end{entry}

Editor:
\href{mailto:srfi-editors\%20at\%20srfi\%20dot\%20schemers\%20dot\%20org}{Arthur
A. Gleckler}
