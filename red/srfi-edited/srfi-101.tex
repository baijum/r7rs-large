\section{Random access pairs and lists}
\redno{9} What random access pair/list library should R7RS-large provide?

\begin{quote}
SRFI 101 is a library for purely functional pairs and lists with O(log
n) access time. To avoid conflicts, procedure names will be modified for
R7RS-large purposes as follows:
\end{quote}

\begin{itemize}
\tightlist
\item
  \texttt{make-list} to become \texttt{make-rlist}
\item
  \texttt{random-access-list-\textgreater{}linear-access-list} to become
  \texttt{rlist-\textgreater{}list}
\item
  \texttt{linear-access-list-\textgreater{}random-access-list} to become
  \texttt{list-\textgreater{}rlist}
\item
  all other identifiers to be prefixed with \texttt{r} (\texttt{rcons},
  \texttt{rpair?}, \texttt{rcar}, \texttt{rmap}, etc.)
\end{itemize}


Random-access lists {[}\protect\hyperlink{note-1}{1}{]} are a purely
functional data structure for representing lists of values. A
random-access list may act as a drop in replacement for the usual
linear-access pair and list data structures (\texttt{pair?},
\texttt{cons}, \texttt{car}, \texttt{cdr}), which additionally supports
fast index-based addressing and updating (\texttt{list-ref},
\texttt{list-set}). The impact is a whole class of purely-functional
algorithms expressed in terms of index-based list addressing become
feasible compared with their linear-access list counterparts.

This document proposes a library API for purely functional random-access
lists consistent with the R\textsuperscript{6}RS
{[}\protect\hyperlink{note-2}{2}{]} base library and list utility
standard library {[}\protect\hyperlink{note-3}{3}{]}.

\setlibname{rlist}

\TODO{The name changes given above have not yet been incorporated into
  this document.}

\TODO{This library somehow redefines features found in R$^7$RS
  small. I'm not sure how to describe that here. I'm leaving this SRFI
  for the next iteration.}

\subsection{{Random-access pairs and lists}}

A \emph{random-access pair} (or just \emph{pair}) is a compound
structure with two fields called the car and the cdr fields (consistent
with the historical naming of pair fields in Scheme). Pairs are created
by the procedure \texttt{cons}. The car and cdr fields are accessed by
the procedures \texttt{car} and \texttt{cdr}.

Pairs are used primarily to represents lists. A list can be defined
recursively as either the empty list or a pair whose cdr is a list. More
precisely, the set of lists is defined as the smallest set \emph{X} such
that

\begin{itemize}
\tightlist
\item
  The empty list is in \emph{X}.
\item
  If \emph{list} is in \emph{X}, then any pair whose cdr field contains
  \emph{list} is also in \emph{X}.
\end{itemize}

The objects in the car fields of successive pairs of a list are the
elements of the list. For example, a two-element list is a pair whose
car is the first element and whose cdr is a pair whose car is the second
element and whose cdr is the empty list. The length of a list is the
number of elements, which is the same as the number of pairs.

The empty list is a special object of its own type. It is not a pair. It
has no elements and its length is zero.

\begin{quote}
\emph{Note:} The above definitions imply that all lists have finite
length and are terminated by the empty list.
\end{quote}

A chain of pairs is defined recursively as either a non-pair object or a
pair whose cdr is a chain of pairs (Note: \emph{every value} is a chain
of pairs). A chain of pairs ending in the empty list is a list. A chain
of pairs not ending in the empty list is called an improper list. Note
that an improper list is not a list. Whether a given pair is a list
depends upon what is stored in the cdr field.

The external representation of pairs is not specified by this SRFI,
however the examples below do use the typical notation for writing pair
and list values.

Random-access pairs and lists are specified to be fully functional, or,
to use the term from the academic literature, \emph{fully persistent}
{[}\protect\hyperlink{note-1}{1}{]}. Full persistence means that all
operations on random-access lists, notably including \texttt{cons},
\texttt{list-ref}, \texttt{list-set}, and \texttt{list-ref/update}, are
specified

\begin{enumerate}
\tightlist
\item
  not to mutate any of their arguments; perforce
\item
  to be safe to execute concurrently on shared arguments; and
\item
  to suffer no degradation of performance as a consequence of the
  history of operations carried out to produce their arguments (except
  as it is reflected in the lengths of those arguments); but permitted
\item
  to produce results that share structure with their arguments.
\end{enumerate}

It is usually taken for granted that standard Scheme lists have these
properties. This SRFI explicitly specifies that random-access lists
share them.

syntax: \texttt{(quote\ datum)}

\emph{Syntax:} \textless{}Datum\textgreater{} should be a syntactic
datum.

\emph{Semantics:} \texttt{(quote\ \textless{}datum\textgreater{})}
evaluates to the datum value represented by
\textless{}datum\textgreater{} (see section
\href{http://www.r6rs.org/final/html/r6rs/r6rs-Z-H-7.html\#node_sec_4.3}{4.3}
of R6RS). This notation is used to include constants.

When the datum value represented by \textless{}datum\textgreater{}
contains pair structure, \texttt{quote} produces random-access pairs.

\begin{verbatim}
(quote a)                                      ⇒ a
(quote #(a b c))                               ⇒ #(a b c)
(quote (+ 1 2))                                ⇒ (+ 1 2)
\end{verbatim}

As noted in section
\href{http://www.r6rs.org/final/html/r6rs/r6rs-Z-H-7.html\#node_sec_4.3.5}{4.3.5}
of R6RS, \texttt{(quote\ \textless{}datum\textgreater{})} may be
abbreviated as
\texttt{\textquotesingle{}\textless{}datum\textgreater{}}:

\begin{verbatim}
'"abc"                                         ⇒ "abc"
'145932                                        ⇒ 145932
'a                                             ⇒ a
'#(a b c)                                      ⇒ #(a b c)
'()                                            ⇒ ()
'(+ 1 2)                                       ⇒ (+ 1 2)
'(quote a)                                     ⇒ (quote a)
''a                                            ⇒ (quote a)
\end{verbatim}

As noted in section
\href{http://www.r6rs.org/final/html/r6rs/r6rs-Z-H-8.html\#node_sec_5.10}{5.10}
of R6RS, constants are immutable.

\begin{quote}
\emph{Note:} Different constants that are the value of \texttt{quote}
expression may share the same locations.
\end{quote}

procedure: \texttt{(equal?\ obj1\ obj2)\ →\ bool}

The \texttt{equal?} predicate returns \texttt{\#t} if and only if the
(possibly infinite) unfoldings of its arguments into regular trees are
equal as ordered trees.

The \texttt{equal?} predicate treats pairs and vectors as nodes with
outgoing edges, uses \texttt{string=?} to compare strings, uses
\texttt{bytevector=?} to compare bytevectors, and uses \texttt{eqv?} to
compare other nodes.

\begin{verbatim}
(equal? 'a 'a)                                 ⇒ #t
(equal? '(a) '(a))                             ⇒ #t
(equal? '(a (b) c)
        '(a (b) c))                            ⇒ #t
(equal? "abc" "abc")                           ⇒ #t
(equal? 2 2)                                   ⇒ #t
(equal? (make-vector 5 'a)
        (make-vector 5 'a))                    ⇒ #t
(equal? '#vu8(1 2 3 4 5)
        (u8-list->bytevector
         '(1 2 3 4 5))                         ⇒ #t
(equal? (lambda (x) x)
        (lambda (y) y))                        ⇒ unspecified

(let* ((x (list 'a))
       (y (list 'a))
       (z (list x y)))
  (list (equal? z (list y x))
        (equal? z (list x x))))                ⇒ (#t #t)
\end{verbatim}

procedure: \texttt{(pair?\ obj)\ →\ bool}

Returns \texttt{\#t} if \emph{obj} is a pair, and otherwise returns
\texttt{\#f}. This operation must take \emph{O(1)} time.

\begin{verbatim}
(pair? '(a . b))                               ⇒ #t
(pair? '(a b c))                               ⇒ #t
(pair? '())                                    ⇒ #f
(pair? '#(a b))                                ⇒ #f
\end{verbatim}

procedure: \texttt{(cons\ obj1\ obj2)\ →\ pair}

Returns a newly allocated pair whose car is \emph{obj\textsubscript{1}}
and whose cdr is \emph{obj\textsubscript{2}}. The pair is guaranteed to
be different (in the sense of \texttt{eqv?}) from every existing object.
This operation must take \emph{O(1)} time.

\begin{verbatim}
(cons 'a '())                                  ⇒  (a)
(cons '(a) '(b c d))                           ⇒  ((a) b c d)
(cons "a" '(b c))                              ⇒  ("a" b c)
(cons 'a 3)                                    ⇒  (a . 3)
(cons '(a b) 'c)                               ⇒  ((a b) . c)
\end{verbatim}

procedure: \texttt{(car\ pair)\ →\ obj}

Returns the contents of the car field of \emph{pair}. This operation
must take \emph{O(1)} time.

\begin{verbatim}
(car '(a b c))                                 ⇒  a
(car '((a) b c d))                             ⇒  (a)
(car '(1 . 2))                                 ⇒  1
(car '())                               &assertion exception
\end{verbatim}

procedure: \texttt{(cdr\ pair)\ →\ obj}

Returns the contents of the cdr field of \emph{pair}. This operation
must take \emph{O(1)} time.

\begin{verbatim}
(cdr '((a) b c d))                             ⇒  (b c d)
(cdr '(1 . 2))                                 ⇒  2
(cdr '())                               &assertion exception
\end{verbatim}

procedure: \texttt{(caar\ pair)\ →\ obj}\\
procedure: \texttt{(cadr\ pair)\ →\ obj}\\
\ldots{}\\
procedure: \texttt{(cdddar\ pair)\ →\ obj}\\
procedure: \texttt{(cddddr\ pair)\ →\ obj}\\

These procedures are compositions of \texttt{car} and \texttt{cdr},
where for example \texttt{caddr} could be defined by\\
\texttt{(define\ caddr\ (lambda\ (x)\ (car\ (cdr\ (cdr\ x)))))}.

Arbitrary compositions, up to four deep, are provided. There are
twenty-eight of these procedures in all. These operations must take
\emph{O(1)} time.

procedure: \texttt{(null?\ obj)\ →\ bool}

Returns \texttt{\#t} if \emph{obj} is the empty list, \texttt{\#f}
otherwise. This procedure is equivalent to the \texttt{null?} procedure
of the R6RS base library.

procedure: \texttt{(list?\ obj)\ →\ bool}

Returns \texttt{\#t} if \emph{obj} is a list, \texttt{\#f} otherwise. By
definition, all lists are chains of pairs that have finite length and
are terminated by the empty list. This operation must take time bounded
by \emph{O(log(n))}, where \emph{n} is the number of pairs in the chain
forming the potential list.

\begin{verbatim}
(list? '(a b c))                               ⇒  #t
(list? '())                                    ⇒  #t
(list? '(a . b))                               ⇒  #f
\end{verbatim}

procedure: \texttt{(list\ obj\ ...)\ →\ list}

Returns a newly allocated list of its arguments. This operation must
take time bounded by \emph{O(n)}, where \emph{n} is the number of
arguments to \texttt{list}.

\begin{verbatim}
(list 'a (+ 3 4) 'c)                           ⇒  (a 7 c)
(list)                                         ⇒  ()
\end{verbatim}

procedure: \texttt{(make-list\ k)\ →\ list}\\
procedure: \texttt{(make-list\ k\ obj)\ →\ list}

Returns a newly allocated list of \emph{k} elements. If a second
argument is given, then each element is initialized to \emph{obj}.
Otherwise the initial contents of each element is unspecified. This
operation must take time and space bounded by \emph{O(log(k))}.

\begin{verbatim}
(make-list 5 0)                                ⇒  (0 0 0 0 0)
\end{verbatim}

procedure: \texttt{(length\ list)\ →\ k}

Returns the length of \emph{list}. This operation must take time bounded
by \emph{O(log(n))}, where \emph{n} is the length of the list.

\begin{verbatim}
(length '(a b c))                              ⇒  3
(length '(a (b) (c)))                          ⇒  3
(length '())                                   ⇒  0
\end{verbatim}

procedure: \texttt{(length\textless{}=?\ obj\ k)\ →\ bool}

Returns true if \emph{obj} is a chain of at least \emph{k} pairs and
false otherwise. This operation must take time bounded by
\emph{O(log(min(k,n)))}, where \emph{n} is the length of the chain of
pairs.

\begin{verbatim}
(length<=? 'not-a-list 0)                      ⇒  #t
(length<=? '(a . b) 0)                         ⇒  #t
(length<=? '(a . b) 1)                         ⇒  #t
(length<=? '(a . b) 2)                         ⇒  #f
\end{verbatim}

procedure: \texttt{(append\ list\ ...\ obj)\ →\ obj}

Returns a chain of pairs consisting of the elements of the first
\emph{list} followed by the elements of the other lists, with \emph{obj}
as the cdr of the final pair. An improper list results if \emph{obj} is
not a list. This operation must take time bounded by \emph{O(log(n))},
where \emph{n} is the total number of elements in the given lists.

\begin{verbatim}
(append '(x) '(y))                             ⇒  (x y)
(append '(a) '(b c d))                         ⇒  (a b c d)
(append '(a (b)) '((c)))                       ⇒  (a (b) (c))
(append '(a b) '(c . d))                       ⇒  (a b c . d)
(append '() 'a)                                ⇒  a
\end{verbatim}

procedure: \texttt{(reverse\ list)\ →\ list}

Returns a newly allocated list consisting of the element of \emph{list}
in reverse order. This operation must take time bounded by \emph{O(n)}
where \emph{n} is the length of the list.

\begin{verbatim}
(reverse '(a b c))                             ⇒  (c b a)
(reverse '(a (b c) 'd '(e (f))))               ⇒  ((e (f)) d (b c) a)
\end{verbatim}

procedure: \texttt{(list-tail\ obj\ k)\ →\ obj}

\emph{Obj} should be a chain of pairs with a count of at least \emph{k}.
The \texttt{list-tail} procedure returns the object obtained by omitting
the first \emph{k} elements in \emph{obj}. This operation must take time
bounded by \emph{O(log(min(k,n)))}, where \emph{n} is the length of the
chain of pairs.

\begin{verbatim}
(list-tail '(a b c d) 0)                       ⇒  (a b c d)
(list-tail '(a b c d) 2)                       ⇒  (c d)
(list-tail 'not-a-list 0)                      ⇒  not-a-list
\end{verbatim}

\emph{Implementation responsibilities:} The implementation must check
that \emph{obj} is a chain of pairs whose count is at least \emph{k}.

procedure: \texttt{(list-ref\ pair\ k)\ →\ obj}

\emph{Pair} must be a chain of pairs whose count is at least \emph{k +
1}. The \texttt{list-ref} procedure returns the \emph{k}th element of
\emph{pair}. This operation must take time bounded by
\emph{O(min(k,log(n)))}, where \emph{n} is the length of the chain of
pairs.

\begin{verbatim}
(list-ref '(a b c d) 2)                        ⇒  c
\end{verbatim}

\emph{Implementation responsibilities:} The implementation must check
that \emph{pair} is a chain of pairs whose count is at least \emph{k +
1}.

procedure: \texttt{(list-set\ pair\ k\ obj)\ →\ obj}

\emph{Pair} must be a chain of pairs whose count is at least \emph{k +
1}. The \texttt{list-set} procedure returns the chain of pairs obtained
by replacing the \emph{k}th element with \emph{obj}. This operation must
take time bounded by \emph{O(min(k,log(n)))}, where \emph{n} is the
length of the chain of pairs.

\begin{verbatim}
(list-set '(a b c d) 2 'x)                     ⇒  (a b x d)
\end{verbatim}

\emph{Implementation responsibilities:} The implementation must check
that \emph{pair} is a chain of pairs whose count is at least \emph{k +
1}.

procedure: \texttt{(list-ref/update\ pair\ k\ proc)\ →\ obj1\ obj2}

Returns the same results as:

\begin{verbatim}
(values (list-ref pair k) 
        (list-set pair k (proc (list-ref pair k))))
\end{verbatim}

but it may be implemented more efficiently.

\begin{verbatim}
(list-ref/update '(7 8 9 10) 2 -)              ⇒  9 (7 8 -9 10)
\end{verbatim}

procedure: \texttt{(map\ proc\ list1\ list2\ ...)\ →\ list}

The \emph{list}s should all have the same length. \emph{Proc} should
accept as many arguments as there are \emph{list}s and return a single
value.

The \texttt{map} procedure applies \emph{proc} element-wise to the
elements of the \emph{list}s and returns a list of the results, in
order. \emph{Proc} is always called in the same dynamic environment as
\texttt{map} itself. The order in which \emph{proc} is applied to the
elements of the \emph{list}s is unspecified.

\begin{verbatim}
(map cadr '((a b) (d e) (g h)))                ⇒  (b e h)

(map (lambda (n) (expt n n))
     '(1 2 3 4 5))
                ⇒  (1 4 27 256 3125)

(map + '(1 2 3) (4 5 6))                       ⇒  (5 7 9)

(let ((count 0))
  (map (lambda (ignored)
         (set! count (+ count 1))
         count)
       '(a b)))                                ⇒  (1 2) or (2 1)
\end{verbatim}

\emph{Implementation responsibilities:} The implementation should check
that the \emph{list}s all have the same length. The implementation must
check the restrictions on \emph{proc} to the extent performed by
applying it as described. An implementation may check whether
\emph{proc} is an appropriate argument before applying it.

procedure: \texttt{(for-each\ proc\ list1\ list2\ ...)\ →\ unspecified}

The lists should all have the same length. \emph{Proc} should accept as
many arguments as there are lists.

The \texttt{for-each} procedure applies \emph{proc} element-wise to the
elements of the \emph{list}s for its side effects, in order from the
first element to the last. \emph{Proc} is always called in the same
dynamic environment as \texttt{for-each} itself. The return values of
\texttt{for-each} are unspecified.

\begin{verbatim}
(let ((v (make-vector 5)))
  (for-each (lambda (i)
              (vector-set! v i (* i i)))
            '(0 1 2 3 4))
  v)                                           ⇒  #(0 1 4 9 16)

(for-each (lambda (x) x) '(1 2 3 4))           ⇒  unspecified
(for-each even? '())                           ⇒  unspecified
\end{verbatim}

\emph{Implementation responsibilities:} The implementation should check
that the \emph{list}s all have the same length. The implementation must
check the restrictions on \emph{proc} to the extent performed by
applying it as described. An implementation may check whether
\emph{proc} is an appropriate argument before applying it.

\begin{quote}
\emph{Note:} Implementations of \texttt{for-each} may or may not
tail-call \emph{proc} on the last element.
\end{quote}

\subsection{{Representation conversion}}\label{representation-conversion}

procedure:
\texttt{(random-access-list-\textgreater{}linear-access-list\ ra-list)\ →\ la-list}\\
procedure:
\texttt{(linear-access-list-\textgreater{}random-access-list\ la-list)\ →\ ra-list}

These procedures convert between (potentially) distinct representations
of lists. To avoid confusion, parameters named \emph{ra-list} range over
lists represented with random-access lists, i.e. objects satisfying the
\texttt{list?} predicate described above, while parameters named
\emph{la-list} range over lists represented with the more traditional
linear-access lists, i.e. objects satisfying the \texttt{list?}
predicate of R6RS. In systems that represent all lists as random-access
lists, these conversions may simply be list identity procedures.

\subsection{{Implementation requirements}}\label{implementation-requirements}

Random-access pairs must be disjoint from all other base types with the
possible exception of (linear-access) pairs.

The external representation of random-access pairs is unspecified. The
behavior of \texttt{equal?} when given a random-access pair and a
sequential-access pair is unspecified in implementations with disjoint
representations.

The behavior of \texttt{eq?} and \texttt{eqv?} on random-access pairs
must be the same as that for pairs, vectors, or records. Namely, two
random-access pair objects are \texttt{eq?} if and only if they are
\texttt{eqv?}, and they are \texttt{eqv?} if and only if they refer to
the same location in the store.

All argument checking for each operation must be done within the time
bounds given for that operation.

Implementations are encouraged, but not required, to support
random-access pairs and lists as their primary pair and list
representation. In such an implementation, the external representation
of random-access pairs and list should be as described in section
\href{http://www.r6rs.org/final/html/r6rs/r6rs-Z-H-7.html\#node_sec_4.3.2}{4.3.2
(Pairs and lists)} of R\textsuperscript{6}RS, the behavior of
equivalence predicates on random-access pairs should be as described in
section
\href{http://www.r6rs.org/final/html/r6rs/r6rs-Z-H-14.html\#node_idx_436}{11.5
(Equivalence predicates)} of R\textsuperscript{6}RS, and so on. In
short, all pairs should be random-access pairs.

Implementations supporting SRFI Libraries
{[}\protect\hyperlink{note-4}{4}{]} and SRFI 101 must provide the
following libraries:

\begin{verbatim}
    (srfi :101)                                 ; Composite libraries
    (srfi :101 random-access-lists)

    (srfi :101 random-access-lists procedures)  ; Procedures only
    (srfi :101 random-access-lists syntax)      ; Syntax only
\end{verbatim}

\section{{References}}\label{references}

\begin{description}
\tightlist
\item[\href{}{{[}1{]}} Purely Functional Random-Access Lists]
Chris Okasaki, \emph{Functional Programming Languages and Computer
Architecture}, June 1995, pages 86-95.\\
\url{http://www.eecs.usma.edu/webs/people/okasaki/pubs.html\#fpca95}
\item[\href{}{{[}2{]}} Revised\textsuperscript{6} Report on the
Algorithmic Language Scheme]
Michael Sperber, \emph{et al.} (Editors)\\
\url{http://www.r6rs.org/}
\item[\href{}{{[}3{]}} Revised\textsuperscript{6} Report on the
Algorithmic Language Scheme, Standard Libraries]
Michael Sperber, \emph{et al.} (Editors)\\
\url{http://www.r6rs.org/}
\item[\href{}{{[}4{]}} SRFI 97: SRFI Libraries]
David Van Horn\\
\url{http://srfi.schemers.org/srfi-97/}
\item[\href{}{{[}5{]}} PLaneT: Purely Functional Random-Access Lists]
David Van Horn\\
\url{http://planet.plt-scheme.org/display.ss?package=ralist.plt\&owner=dvanhorn}
\end{description}


\begin{center}\rule{0.5\linewidth}{\linethickness}\end{center}

Editor:
\href{mailto:srfi\%20minus\%20editors\%20at\%20srfi\%20dot\%20schemers\%20dot\%20org}{Donovan
Kolbly}
