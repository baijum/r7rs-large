\section{List queues}
\redno{16}  What list queue library should R7RS-large provide?

\begin{quote}
SRFI 117 provides mutable half-deques with $O(1)$ addition at both ends
and $O(1)$ deletion from the front. They are based directly on ordinary
Scheme lists, and are known in other Lisps as ``tconc structures''.
\end{quote}

List queues are mutable ordered collections that can contain any Scheme
object. Each list queue is based on an ordinary Scheme list containing
the elements of the list queue by maintaining pointers to the first and
last pairs of the list. It's cheap to add or remove elements from the
front of the list or to add elements to the back, but not to remove
elements from the back. List queues are disjoint from other types of
Scheme objects.

\setlibname{list-queue}


\subsection{Constructors}\label{Constructors}

\begin{entry}{%
  \Proto{make-list-queue}{ list last}{procedure}{list-queue}
  \Rproto{make-list-queue}{ list}{procedure}{list-queue}}

  Returns a newly allocated list queue containing the elements of list
  in order. The result shares storage with list. If the last argument
  is not provided, this operation is $O(n)$ where $n$ is the length of
  list.

  However, if last is provided, \texttt{make-list-queue} returns a
  newly allocated list queue containing the elements of the list whose
  first pair is first and whose last pair is last. It is an error if
  the pairs do not belong to the same list. Alternatively, both first
  and last can be the empty list. In either case, the operation is
  $O(1)$.

  Note: To apply a non-destructive list procedure to a list queue and
  return a new list queue, use \texttt{(make-list-queue\
    (}proc\texttt{\ (list-queue-list\ }list-queue\texttt{)))}.
\end{entry}

\begin{entry}{%
  \Proto{list-queue}{ element \ldots}{list-queue}}

  Returns a newly allocated list queue containing the elements. This
  operation is $O(n)$ where $n$ is the number of elements.
\end{entry}

\begin{entry}{%
  \Proto{list-queue-copy}{ list-queue}{procedure}{list-queue}}

  Returns a newly allocated list queue containing the elements of
  list-queue. This operation is $O(n)$ where $n$ is the length of
  list-queue.
\end{entry}

\begin{entry}{%
  \Proto{list-queue-unfold}{ stop? mapper successor seed queue}{procedure}{list-queue}
  \Rproto{list-queue-unfold}{ stop? mapper successor seed}{list-queue}}

  Performs the following algorithm:

  If the result of applying the predicate stop? to seed is true,
  return queue. Otherwise, apply the procedure mapper to seed,
  returning a value which is added to the front of queue. Then get a
  new seed by applying the procedure successor to seed, and repeat
  this algorithm.

  If queue is omitted, a newly allocated list queue is used.
\end{entry}

\begin{entry}{%
  \Proto{list-queue-unfold-right}{ stop? mapper successor seed queue}{list-queue}}

  Performs the following algorithm:

  If the result of applying the predicate stop? to seed is true,
  return the list queue. Otherwise, apply the procedure mapper to
  seed, returning a value which is added to the back of the list
  queue. Then get a new seed by applying the procedure successor to
  seed, and repeat this algorithm.

  If queue is omitted, a newly allocated list queue is used.
\end{entry}

\subsection{Predicates}\label{Predicates}

\begin{entry}{%
  \Proto{list-queue?}{ obj}{procedure}{boolean}}

  Returns \texttt{\#t} if obj is a list queue, and \texttt{\#f}
  otherwise.  This operation is $O(1)$.
\end{entry}

\begin{entry}{%
  \Proto{list-queue-empty?}{ list-queue}{procedure}{boolean}}

  Returns \texttt{\#t} if list-queue has no elements, and \texttt{\#f}
  otherwise. This operation is $O(1)$.
\end{entry}

\subsection{Accessors}\label{Accessors}

\begin{entry}{%
  \Proto{list-queue-front}{ list-queue}{procedure}value}

  Returns the first element of list-queue. If the list queue is empty,
  it is an error. This operation is $O(1)$.
\end{entry}

\begin{entry}{%
  \Proto{list-queue-back}{ list-queue}{procedure}{value}}

  Returns the last element of list-queue. If the list queue is empty,
  it is an error. This operation is $O(1)$.
\end{entry}

\begin{entry}{%
  \Proto{list-queue-list}{ list-queue}{procedure}{list}}

  Returns the list that contains the members of list-queue in
  order. The result shares storage with list-queue. This operation is
  $O(1)$.
\end{entry}

\begin{entry}{%
  \Proto{list-queue-first-last}{ list-queue}{procedure}{value value}}

  Returns two values, the first and last pairs of the list that
  contains the members of list-queue in order. If list-queue is empty,
  returns two empty lists. The results share storage with
  list-queue. This operation is $O(1)$.
\end{entry}

\subsection{Mutators}\label{Mutators}

\begin{entry}{%
  \Proto{list-queue-add-front!}{ list-queue element}{procedure}{unspecified}}

  Adds element to the beginning of list-queue. Returns an unspecified
  value. This operation is $O(1)$.
\end{entry}

\begin{entry}{%
  \Proto{list-queue-add-back!}{ list-queue element}{procedure}{unspecified}}

  Adds element to the end of list-queue. Returns an unspecified value.
  This operation is $O(1)$.
\end{entry}

\begin{entry}{%
  \Proto{list-queue-remove-front!}{ list-queue}{procedure}{value}}

  Removes the first element of list-queue and returns it. If the list
  queue is empty, it is an error. This operation is $O(1)$.
\end{entry}

\begin{entry}{%
  \Proto{list-queue-remove-back!}{ list-queue}{procedure}{value}}

  Removes the last element of list-queue and returns it. If the list
  queue is empty, it is an error. This operation is $O(n)$ where $n$ is
  the length of list-queue, because queues do not not have backward
  links.
\end{entry}

\begin{entry}{%
  \Proto{list-queue-remove-all!}{ list-queue}{procedure}{list}}

  Removes all the elements of list-queue and returns them in order as
  a list. This operation is $O(1)$.
\end{entry}

\begin{entry}{%
  \Proto{list-queue-set-list!}{ list-queue list last}{procedure}{unspecified}
  \Rproto{list-queue-set-list!}{ list-queue list}{procedure}{unspecified}}

  Replaces the list associated with list-queue with list, effectively
  discarding all the elements of list-queue in favor of those in list.
  Returns an unspecified value. This operation is $O(n)$ where $n$ is the
  length of list. If last is provided, it is treated in the same way
  as in \texttt{make-list-queue}, and the operation is $O(1)$.

  Note: To apply a destructive list procedure to a list queue, use
  \texttt{(list-queue-set-list!\ (}proc\texttt{\ (list-queue-list\
  }list-queue\texttt{)))}.
\end{entry}

\begin{entry}{%
  \Proto{list-queue-append}{ list-queue \ldots}{procedure}{list-queue}}

  Returns a list queue which contains all the elements in
  front-to-back order from all the list-queues in front-to-back
  order. The result does not share storage with any of the
  arguments. This operation is $O(n)$ in the total number of elements in
  all queues.

  \TODO{Does this belong in a subsection entitled “Mutators”?}
\end{entry}

\begin{entry}{%
  \Proto{list-queue-append!}{ list-queue \ldots}{procedure}{list-queue}}

  Returns a list queue which contains all the elements in
  front-to-back order from all the list-queues in front-to-back
  order. It is an error to assume anything about the contents of the
  list-queues after the procedure returns. This operation is $O(n)$ in
  the total number of queues, not elements. It is not part of the
  R7RS-small list API, but is included here for efficiency when pure
  functional append is not required.
\end{entry}

\begin{entry}{%
  \Proto{list-queue-concatenate}{ list-of-list-queues}{procedure}{list-queue}}

  Returns a list queue which contains all the elements in
  front-to-back order from all the list queues which are members of
  list-of-list-queues in front-to-back order. The result does not
  share storage with any of the arguments. This operation is $O(n)$ in
  the total number of elements in all queues. It is not part of the
  R7RS-small list API, but is included here to make appending a large
  number of queues possible in Schemes that limit the number of
  arguments to \texttt{apply}.
\end{entry}

\subsection{Mapping}\label{Mapping}

\begin{entry}{%
  \Proto{list-queue-map}{ proc list-queue}{procedure}{list-queue}}

  Applies proc to each element of list-queue in unspecified order and
  returns a newly allocated list queue containing the results. This
  operation is $O(n)$ where $n$ is the length of list-queue.
\end{entry}

\begin{entry}{%
  \Proto{list-queue-map!}{ proc list-queue}{procedure}{list-queue}}

  Applies proc to each element of list-queue in front-to-back order
  and modifies list-queue to contain the results. This operation is
  $O(n)$ in the length of list-queue. It is not part of the R7RS-small
  list API, but is included here to make transformation of a list
  queue by mutation more efficient.
\end{entry}

\begin{entry}{%
  \Proto{list-queue-for-each}{ proc list-queue}{procedure}{unspecified}}

  Applies proc to each element of list-queue in front-to-back order,
  discarding the returned values. Returns an unspecified value. This
  operation is $O(n)$ where $n$ is the length of list-queue.
\end{entry}

Editor:
\href{mailto:srfi-editors\%20at\%20srfi\%20dot\%20schemers\%20dot\%20org}{Michael
Sperber}

Last modified: Tue Jun 9 08:44:01 MST 2015
