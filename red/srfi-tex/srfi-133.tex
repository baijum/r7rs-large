\section{Title}\label{title}

Vector Library (R7RS-compatible)

\section{Author}\label{author}

John Cowan (based on SRFI 43 by Taylor Campbell)

\section{Issues}\label{issues}

None at this time.

\section{Table of Contents}\label{table-of-contents}

\begin{itemize}
\tightlist
\item
  1. \protect\hyperlink{Status}{Status}
\item
  2. \protect\hyperlink{Abstract}{Abstract}
\item
  3. \protect\hyperlink{Rationale}{Rationale}
\item
  4. \protect\hyperlink{ProcIndex}{Procedure Index}
\item
  5. \protect\hyperlink{Procs}{Procedures}

  \begin{itemize}
  \tightlist
  \item
    5.1. \protect\hyperlink{Constructors}{Constructors}
  \item
    5.2. \protect\hyperlink{Predicates}{Predicates}
  \item
    5.3. \protect\hyperlink{Selectors}{Selectors}
  \item
    5.4. \protect\hyperlink{Iteration}{Iteration}
  \item
    5.5. \protect\hyperlink{Searching}{Searching}
  \item
    5.6. \protect\hyperlink{Mutators}{Mutators}
  \item
    5.7. \protect\hyperlink{Conversion}{Conversion}
  \end{itemize}
\item
  6. \protect\hyperlink{SampImpl}{Sample Implementation}
\item
  7. \protect\hyperlink{Acknowledgements}{Acknowledgements}
\item
  8. \protect\hyperlink{References}{References}
\item
  9. \protect\hyperlink{Copyright}{Copyright}
\end{itemize}

\section{\texorpdfstring{\href{}{1. Status}}{1. Status}}\label{status}

This SRFI is currently in \emph{final} status. Here is
\href{http://srfi.schemers.org/srfi-process.html}{an explanation} of
each status that a SRFI can hold. To provide input on this SRFI, please
send email to \texttt{srfi-133@nospamsrfi.schemers.org}. To subscribe to
the list, follow
\href{http://srfi.schemers.org/srfi-list-subscribe.html}{these
instructions}. You can access previous messages via the mailing list
\href{http://srfi-email.schemers.org/srfi-133}{archive}.

\begin{itemize}
\tightlist
\item
  Received: 2015/12/15
\item
  60-day deadline: 2016/2/14
\item
  Draft \#1 published: 2015/12/16
\item
  Draft \#2 published: 2016/1/24
\item
  Draft \#3 published: 2016/1/31
\item
  Draft \#4 published: 2016/3/12
\item
  Finalized: 2016/3/20
\item
  Revised to fix errata:

  \begin{itemize}
  \tightlist
  \item
    2016/4/1 (Fixed description of \texttt{vector-unfold!})
  \item
    2016/8/10 (Fixed typos.)
  \item
    2016/9/2 (Changed order of arguments to \texttt{vector-cumulate}
    from \texttt{(vector-cumulate\ f\ vec\ knil)} to
    \texttt{(vector-cumulate\ f\ knil\ vec)} to match the argument order
    of vector-fold.)
  \end{itemize}
\end{itemize}

\section{\texorpdfstring{\href{}{2.
Abstract}}{2. Abstract}}\label{abstract}

This \protect\hyperlink{SRFI}{SRFI} proposes a comprehensive library of
vector operations accompanied by a freely available and complete
reference implementation. The reference implementation is unencumbered
by copyright, and useable with no modifications on any Scheme system
that is \protect\hyperlink{R5RS}{R5RS}-compliant. It also provides
several hooks for implementation-specific optimization as well.

\section{\texorpdfstring{\href{}{3.
Rationale}}{3. Rationale}}\label{rationale}

\protect\hyperlink{R5RS}{R5RS} provides very few list-processing
procedures, for which reason \protect\hyperlink{SRFI-1}{SRFI 1} exists.
However, \protect\hyperlink{R5RS}{R5RS} provides even fewer vector
operations --- while it provides mapping, appending, et cetera
operations for lists, it specifies only nine vector manipulation
operations:

\begin{itemize}
\tightlist
\item
  \texttt{vector?}
\item
  \texttt{make-vector}
\item
  \texttt{vector}
\item
  \texttt{vector-length}
\item
  \texttt{vector-ref}
\item
  \texttt{vector-set!}
\item
  \texttt{vector-\textgreater{}list}
\item
  \texttt{list-\textgreater{}vector}
\item
  \texttt{vector-fill!}
\end{itemize}

\protect\hyperlink{R7RS-small}{R7RS-small} added support for
\emph{start} and \emph{end} arguments to
\texttt{vector-\textgreater{}list}, \texttt{list-\textgreater{}vector}
and \texttt{vector-fill!}. It also provided seven additional vector
procedures, bringing vectors and lists to approximate parity:

\begin{itemize}
\tightlist
\item
  \texttt{vector-copy}
\item
  \texttt{vector-copy!}
\item
  \texttt{vector-append}
\item
  \texttt{vector-map}
\item
  \texttt{vector-for-each}
\item
  \texttt{vector-\textgreater{}string}
\item
  \texttt{string-\textgreater{}vector}
\end{itemize}

\protect\hyperlink{SRFI-43}{SRFI 43} standardized more vector
procedures, all of which are included in this SRFI. Unfortunately,
R7RS-small and SRFI 43 placed irreconcileable requirements on the
procedures invoked by \texttt{vector-map} and \texttt{vector-for-each}.
This SRFI resolves that issue by changing these SRFI 43 procedures as
well as \texttt{vector-map!}, \texttt{vector-fold},
\texttt{vector-fold-right}, and \texttt{vector-count} to leave out the
index argument that is passed under SRFI 43's definition.

In addition, the version of \texttt{vector-copy} in this SRFI does not
require support for a fill argument, which makes it equivalent to the
R7RS-small definition.

This SRFI also provides the following new procedures (some from Python,
some from other sources):

\begin{itemize}
\tightlist
\item
  \texttt{vector-unfold!}
\item
  \texttt{vector-unfold-right!}
\item
  \texttt{vector-append-subvectors}
\item
  \texttt{vector-cumulate}
\item
  \texttt{vector-partition}
\end{itemize}

It should be noted that no vector sorting procedures are provided by
this SRFI, because there already are several SRFIs for that purpose.

\section{\texorpdfstring{\href{}{4. Procedure
Index}}{4. Procedure Index}}\label{procedure-index}

Here is an index of the procedures provided by this package. Those
marked by \emph{italics} are also provided in
\protect\hyperlink{R7RS-small}{R7RS-small}.

\begin{description}
\tightlist
\item[{
·~\protect\hyperlink{Constructors}{Constructors}\\[2\baselineskip]}]
\texttt{\ \ \ \ \ \ \ \ \ \ \ make-vector\ \ \ \ \ \ \ \ \ \ \ vector\ \ \ \ \ \ \ \ \ \ \ \ \ \ \ \ \ \ \ \ \ \ vector-unfold\ \ \ \ \ \ \ \ \ \ \ vector-unfold-right\ \ \ \ \ \ \ \ \ \ \ \ \ \ \ \ \ \ \ \ \ \ vector-copy\ \ \ \ \ \ \ \ \ \ \ vector-reverse-copy\ \ \ \ \ \ \ \ \ \ \ \ \ \ \ \ \ \ \ \ \ \ vector-append\ \ \ \ \ \ \ \ \ \ \ vector-concatenate\ \ \ \ \ \ \ \ \ \ \ vector-append-subvectors\ \ \ \ \ \ \ \ \ \ \ \ \ \ \ \ \ \ \ \ \ \ \ \ \ \ \ \ \ \ \ }
\item[{ ·~\protect\hyperlink{Predicates}{Predicates}\\[2\baselineskip]}]
\texttt{\ \ \ \ \ \ \ \ \ \ \ vector?\ \ \ \ \ \ \ \ \ \ \ \ \ \ \ \ \ \ \ \ \ \ vector-empty?\ \ \ \ \ \ \ \ \ \ \ \ \ \ \ \ \ \ \ \ \ \ vector=\ \ \ \ \ \ \ \ \ \ \ \ \ \ \ \ \ \ \ \ \ \ \ \ \ \ \ \ \ \ \ }
\item[{ ·~\protect\hyperlink{Selectors}{Selectors}\\[2\baselineskip]}]
\texttt{\ \ \ \ \ \ \ \ \ \ \ vector-ref\ \ \ \ \ \ \ \ \ \ \ \ \ \ \ \ \ \ \ \ \ \ vector-length\ \ \ \ \ \ \ \ \ \ \ \ \ \ \ \ \ \ \ \ \ \ \ \ \ \ \ \ \ \ \ }
\item[{ ·~\protect\hyperlink{Iteration}{Iteration}\\[2\baselineskip]}]
\texttt{\ \ \ \ \ \ \ \ \ \ \ vector-fold\ \ \ \ \ \ \ \ \ \ \ vector-fold-right\ \ \ \ \ \ \ \ \ \ \ \ \ \ \ \ \ \ \ \ \ \ vector-map\ \ \ \ \ \ \ \ \ \ \ vector-map!\ \ \ \ \ \ \ \ \ \ \ \ \ \ \ \ \ \ \ \ \ \ vector-for-each\ \ \ \ \ \ \ \ \ \ \ vector-count\ \ \ \ \ \ \ \ \ \ \ \ \ \ \ \ \ \ \ \ \ \ vector-cumulate\ \ \ \ \ \ \ \ \ \ \ \ \ \ \ \ \ \ \ \ \ \ \ \ \ \ \ \ \ \ \ }
\item[{ ·~\protect\hyperlink{Searching}{Searching}\\[2\baselineskip]}]
\texttt{\ \ \ \ \ \ \ \ \ \ \ vector-index\ \ \ \ \ \ \ \ \ \ \ vector-index-right\ \ \ \ \ \ \ \ \ \ \ \ \ \ \ \ \ \ \ \ \ \ vector-skip\ \ \ \ \ \ \ \ \ \ \ vector-skip-right\ \ \ \ \ \ \ \ \ \ \ \ \ \ \ \ \ \ \ \ \ \ vector-binary-search\ \ \ \ \ \ \ \ \ \ \ \ \ \ \ \ \ \ \ \ \ \ vector-any\ \ \ \ \ \ \ \ \ \ \ vector-every\ \ \ \ \ \ \ \ \ \ \ \ \ \ \ \ \ \ \ \ \ \ vector-partition\ \ \ \ \ \ \ \ \ \ \ \ \ \ \ \ \ \ \ \ \ \ \ \ \ \ \ \ \ \ \ }
\item[{ ·~\protect\hyperlink{Mutators}{Mutators}\\[2\baselineskip]}]
\texttt{\ \ \ \ \ \ \ \ \ \ \ vector-set!\ \ \ \ \ \ \ \ \ \ \ vector-swap!\ \ \ \ \ \ \ \ \ \ \ \ \ \ \ \ \ \ \ \ \ \ vector-fill!\ \ \ \ \ \ \ \ \ \ \ vector-reverse!\ \ \ \ \ \ \ \ \ \ \ \ \ \ \ \ \ \ \ \ \ \ vector-copy!\ \ \ \ \ \ \ \ \ \ \ vector-reverse-copy!\ \ \ \ \ \ \ \ \ \ \ \ \ \ \ \ \ \ \ \ \ \ vector-unfold!\ \ \ \ \ \ \ \ \ \ \ vector-unfold-right!\ \ \ \ \ \ \ \ \ \ \ \ \ \ \ \ \ \ \ \ \ \ \ \ \ \ \ \ \ \ \ }
\item[{ ·~\protect\hyperlink{Conversion}{Conversion}\\[2\baselineskip]}]
\texttt{\ \ \ \ \ \ \ \ \ \ \ vector-\textgreater{}list\ \ \ \ \ \ \ \ \ \ \ reverse-vector-\textgreater{}list\ \ \ \ \ \ \ \ \ \ \ \ \ \ \ \ \ \ \ \ \ \ list-\textgreater{}vector\ \ \ \ \ \ \ \ \ \ \ reverse-list-\textgreater{}vector\ \ \ \ \ \ \ \ \ \ \ \ \ \ \ \ \ \ \ \ \ \ vector-\textgreater{}string\ \ \ \ \ \ \ \ \ \ \ string-\textgreater{}vector\ \ \ \ \ \ \ \ \ }
\end{description}

\section{\texorpdfstring{\href{}{5.
Procedures}}{5. Procedures}}\label{procedures}

In this section containing specifications of procedures, the following
notation is used to specify parameters and return values:

\begin{description}
\tightlist
\item[ (f arg\textsubscript{1} arg\textsubscript{2} \ldots{})
-\textgreater{} something]
Indicates a function \texttt{f} takes the parameters
\texttt{arg1\ arg2\ \ \ \ \ \ \ \ \ ...} and returns a value of the type
\texttt{something}. If \texttt{something} is \texttt{unspecified}, then
\texttt{f} returns a single implementation-dependent value; this SRFI
does not specify what it returns, and in order to write portable code,
the return value should be ignored.\\[2\baselineskip]
\item[vec]
The argument in this place must be a vector, i.e. it must satisfy the
predicate \texttt{vector?}.\\[2\baselineskip]
\item[i, j, start, size]
The argument in this place must be a exact nonnegative integer, i.e. it
must satisfy the predicates \texttt{exact?}, \texttt{integer?} and
either \texttt{zero?} or \texttt{positive?}. The third case of it
indicates the index at which traversal begins; the fourth case of it
indicates the size of a vector.\\[2\baselineskip]
\item[end]
The argument in this place must be a exact positive integer, i.e. it
must satisfy the predicates \texttt{exact?}, \texttt{integer?} and
\texttt{positive?}. This indicates the index directly before which
traversal will stop --- processing will occur until the the index of the
vector is \texttt{end}. It is the closed right side of a
range.\\[2\baselineskip]
\item[f]
The argument in this place must be a function of one or more arguments,
which returns (except as noted otherwise) exactly one
value.\\[2\baselineskip]
\item[pred?]
The argument in this place must be a function of one or more arguments
that returns one value, which is treated as a boolean.\\[2\baselineskip]
\item[ x, y, z, seed, knil, fill, key, value ]
The argument in this place may be any Scheme value.\\[2\baselineskip]
\item[{[}something{]}]
Indicates that \texttt{something} is an optional argument; it needn't
necessarily be applied. \texttt{Something} needn't necessarily be one
thing; for example, this usage of it is perfectly
valid:\\[2\baselineskip]\texttt{\ \ \ \ \ \ \ \ \ \ \ ~~~{[}start\ {[}end{]}{]}\ \ \ \ \ \ \ \ \ }\\[2\baselineskip]and
is indeed used quite often.\\[2\baselineskip]
\item[something \ldots{}]
Indicates that zero or more \texttt{something}s are allowed to be
arguments.\\[2\baselineskip]
\item[ something\textsubscript{1} something\textsubscript{2} \ldots{} ]
Indicates that at least one \texttt{something} must be
arguments.\\[2\baselineskip]
\item[ something\textsubscript{1} something\textsubscript{2} \ldots{}
something\textsubscript{n} ]
Exactly equivalent to the previous argument notation, but this also
indicates that \texttt{n} will be used later in the procedure
description.\\[2\baselineskip]
\end{description}

It should be noted that all of the procedures that iterate across
multiple vectors in parallel stop iterating and produce the final result
when the end of the shortest vector is reached. The sole exception is
\texttt{vector=}, which automatically returns \texttt{\#f} if the
vectors' lengths vary.

\subsection{\texorpdfstring{\href{}{5.1.
Constructors}}{5.1. Constructors}}\label{constructors}

\begin{description}
\tightlist
\item[ \href{}{(make-vector \emph{size} {[}\emph{fill}{]})
-\textgreater{} vector} ]
{[}\protect\hyperlink{R7RS-small}{\emph{R7RS-small}}{]} Creates and
returns a vector of size \texttt{size}. If \emph{fill} is specified, all
the elements of the vector are initialized to \emph{fill}. Otherwise,
their contents are
indeterminate.\\[2\baselineskip]Example:\\[2\baselineskip]\texttt{\ \ \ \ \ \ \ \ \ \ \ (make-vector\ 5\ 3)\ \ \ \ \ \ \ \ \ }\\
\texttt{\ \ \ \ \ \ \ \ \ \ \ \#(3\ 3\ 3\ 3\ 3)\ \ \ \ \ \ \ \ \ }\\[2\baselineskip]
\item[ \href{}{(vector \emph{x \ldots{}}) -\textgreater{} vector} ]
{[}\protect\hyperlink{R7RS-small}{\emph{R7RS-small}}{]} Creates and
returns a vector whose elements are
\texttt{x\ ...}.\\[2\baselineskip]Example:\\[2\baselineskip]\texttt{\ \ \ \ \ \ \ \ \ \ \ (vector\ 0\ 1\ 2\ 3\ 4)\ \ \ \ \ \ \ \ \ }\\
\texttt{\ \ \ \ \ \ \ \ \ \ \ \#(0\ 1\ 2\ 3\ 4)\ \ \ \ \ \ \ \ \ }\\[2\baselineskip]
\item[ \href{}{(vector-unfold \emph{f length initial-seed \ldots{}})
-\textgreater{} vector} ]
The fundamental vector constructor. Creates a vector whose length is
\texttt{length} and iterates across each index \texttt{k} between
\texttt{0} and \texttt{length}, applying \texttt{f} at each iteration to
the current index and current seeds, in that order, to receive
\texttt{n\ +\ 1} values: first, the element to put in the \texttt{k}th
slot of the new vector and \texttt{n} new seeds for the next iteration.
It is an error for the number of seeds to vary between iterations. Note
that the termination condition is different from the \texttt{unfold}
procedure of \protect\hyperlink{SRFIux5cux25201}{SRFI
1}.\\[2\baselineskip]Examples:\\[2\baselineskip]\texttt{\ \ \ \ \ \ \ \ \ \ \ (vector-unfold\ (λ\ (i\ x)\ (values\ x\ (-\ x\ 1)))\ \ \ \ \ \ \ \ \ \ \ ~~~~~~~~~~\ \ \ \ \ \ \ \ \ \ \ ~~~~~\ \ \ \ \ \ \ \ \ \ \ \ \ \ \ \ \ \ \ \ \ \ \ \ \ \ 10\ 0)\ \ \ \ \ \ \ \ \ }\\
\texttt{\ \ \ \ \ \ \ \ \ \ \ \#(0\ -1\ -2\ -3\ -4\ -5\ -6\ -7\ -8\ -9)\ \ \ \ \ \ \ \ \ }\\[2\baselineskip]Construct
a vector of the sequence of integers in the range {[}0,\texttt{n}).\\
\texttt{\ \ \ \ \ \ \ \ \ \ \ (vector-unfold\ values\ n)\ \ \ \ \ \ \ \ \ }\\
\texttt{\ \ \ \ \ \ \ \ \ \ \ \#(0\ 1\ 2\ ...\ n-2\ n-1)\ \ \ \ \ \ \ \ \ }\\[2\baselineskip]Copy
\texttt{vector}.\\[2\baselineskip]\texttt{\ \ \ \ \ \ \ \ \ \ \ (vector-unfold\ (λ\ (i)\ (vector-ref\ vector\ i))\ \ \ \ \ \ \ \ \ \ \ ~~~~~~~~~~\ \ \ \ \ \ \ \ \ \ \ ~~~~~\ \ \ \ \ \ \ \ \ \ \ \ \ \ \ \ \ \ \ \ \ \ \ \ \ \ (vector-length\ vector))\ \ \ \ \ \ \ \ \ }\\[2\baselineskip]
\item[ \href{}{(vector-unfold-right \emph{f length initial-seed
\ldots{}}) -\textgreater{} vector} ]
Like \texttt{vector-unfold}, but it uses \texttt{f} to generate elements
from right-to-left, rather than left-to-right. The first index used is
\emph{length} - 1. Note that the termination condition is different from
the \texttt{unfold-right} procedure of
\protect\hyperlink{SRFIux5cux25201}{SRFI
1}.\\[2\baselineskip]Examples:\\[2\baselineskip]Construct a vector of
pairs of non-negative integers whose values sum to
4.\\[2\baselineskip]\texttt{\ \ \ \ \ \ \ \ \ \ \ (vector-unfold-right\ (λ\ (i\ x)\ (values\ (cons\ i\ x)\ (+\ x\ 1)))\ 5\ 0)\ \ \ \ \ \ \ \ \ }\\
\texttt{\ \ \ \ \ \ \ \ \ \ \ \#((0\ .\ 4)\ (1\ .\ 3)\ (2\ .\ 2)\ (3\ .\ 1)\ (4\ .\ 0))\ \ \ \ \ \ \ \ \ }\\[2\baselineskip]Reverse
\texttt{vector}.\\[2\baselineskip]\texttt{\ \ \ \ \ \ \ \ \ \ \ (vector-unfold-right\ (λ\ (i\ x)\ \ \ \ \ \ \ \ \ \ \ \ \ \ \ \ \ \ \ \ \ \ \ \ \ \ \ \ \ \ \ \ \ \ (values\ (vector-ref\ vector\ x)\ \ \ \ \ \ \ \ \ \ \ \ \ \ \ \ \ \ \ \ \ \ \ \ \ \ \ \ \ \ \ \ \ \ \ \ \ \ \ \ \ \ (+\ x\ 1)))\ \ \ \ \ \ \ \ \ \ \ ~~~~~~~~~~\ \ \ \ \ \ \ \ \ \ \ ~~~~~~~~~~~\ \ \ \ \ \ \ \ \ \ \ \ \ \ \ \ \ \ \ \ \ \ \ \ \ \ \ \ \ \ \ \ (vector-length\ vector)\ \ \ \ \ \ \ \ \ \ \ ~~~~~~~~~~\ \ \ \ \ \ \ \ \ \ \ ~~~~~~~~~~~\ \ \ \ \ \ \ \ \ \ \ \ \ \ \ \ \ \ \ \ \ \ \ \ \ \ \ \ \ \ \ \ 0)\ \ \ \ \ \ \ \ \ }\\[2\baselineskip]
\item[ \href{}{(vector-copy \emph{vec} {[}\emph{start}
{[}\emph{end}{]}{]}) -\textgreater{} vector} ]
{[}\protect\hyperlink{R7RS-small}{\emph{R7RS-small}}{]} Allocates a new
vector whose length is \texttt{end\ -\ \ \ \ \ \ \ \ \ start} and fills
it with elements from \texttt{vec}, taking elements from \texttt{vec}
starting at index \texttt{start} and stopping at index \texttt{end}.
\texttt{start} defaults to \texttt{0} and \texttt{end} defaults to the
value of \texttt{(vector-length\ \ \ \ \ \ \ \ \ vec)}. SRFI 43 provides
an optional fill argument to supply values if end is greater than the
length of vec. Neither R7RS-small nor this SRFI requires support for
this
argument.\\[2\baselineskip]Examples:\\[2\baselineskip]\texttt{\ \ \ \ \ \ \ \ \ \ \ (vector-copy\ \textquotesingle{}\#(a\ b\ c\ d\ e\ f\ g\ h\ i))\ \ \ \ \ \ \ \ \ }\\
\texttt{\ \ \ \ \ \ \ \ \ \ \ \#(a\ b\ c\ d\ e\ f\ g\ h\ i)\ \ \ \ \ \ \ \ \ }\\[2\baselineskip]\texttt{\ \ \ \ \ \ \ \ \ \ \ (vector-copy\ \textquotesingle{}\#(a\ b\ c\ d\ e\ f\ g\ h\ i)\ 6)\ \ \ \ \ \ \ \ \ }\\
\texttt{\ \ \ \ \ \ \ \ \ \ \ \#(g\ h\ i)\ \ \ \ \ \ \ \ \ }\\[2\baselineskip]\texttt{\ \ \ \ \ \ \ \ \ \ \ (vector-copy\ \textquotesingle{}\#(a\ b\ c\ d\ e\ f\ g\ h\ i)\ 3\ 6)\ \ \ \ \ \ \ \ \ }\\
\texttt{\ \ \ \ \ \ \ \ \ \ \ \#(d\ e\ f)\ \ \ \ \ \ \ \ \ }\\[2\baselineskip]
\item[ \href{}{(vector-reverse-copy \emph{vec} {[}\emph{start}
{[}\emph{end}{]}{]}) -\textgreater{} vector} ]
Like \texttt{vector-copy}, but it copies the elements in the reverse
order from
\texttt{vec}.\\[2\baselineskip]Example:\\[2\baselineskip]\texttt{\ \ \ \ \ \ \ \ \ \ \ (vector-reverse-copy\ \textquotesingle{}\#(5\ 4\ 3\ 2\ 1\ 0)\ 1\ 5)\ \ \ \ \ \ \ \ \ }\\
\texttt{\ \ \ \ \ \ \ \ \ \ \ \#(1\ 2\ 3\ 4)\ \ \ \ \ \ \ \ \ }\\[2\baselineskip]
\item[ \href{}{(vector-append \emph{vec \ldots{}}) -\textgreater{}
vector} ]
{[}\protect\hyperlink{R7RS-small}{\emph{R7RS-small}}{]} Returns a newly
allocated vector that contains all elements in order from the subsequent
locations in
\texttt{vec\ \ \ \ \ \ \ \ \ ...}.\\[2\baselineskip]Examples:\\[2\baselineskip]\texttt{\ \ \ \ \ \ \ \ \ \ \ (vector-append\ \textquotesingle{}\#(x)\ \textquotesingle{}\#(y))\ \ \ \ \ \ \ \ \ }\\
\texttt{\ \ \ \ \ \ \ \ \ \ \ \#(x\ y)\ \ \ \ \ \ \ \ \ }\\[2\baselineskip]\texttt{\ \ \ \ \ \ \ \ \ \ \ (vector-append\ \textquotesingle{}\#(a)\ \textquotesingle{}\#(b\ c\ d))\ \ \ \ \ \ \ \ \ }\\
\texttt{\ \ \ \ \ \ \ \ \ \ \ \#(a\ b\ c\ d)\ \ \ \ \ \ \ \ \ }\\[2\baselineskip]\texttt{\ \ \ \ \ \ \ \ \ \ \ (vector-append\ \textquotesingle{}\#(a\ \#(b))\ \textquotesingle{}\#(\#(c)))\ \ \ \ \ \ \ \ \ }\\
\texttt{\ \ \ \ \ \ \ \ \ \ \ \#(a\ \#(b)\ \#(c))\ \ \ \ \ \ \ \ \ }\\[2\baselineskip]
\item[ \href{}{(vector-concatenate \emph{list-of-vectors})
-\textgreater{} vector} ]
Appends each vector in \texttt{list-of-vectors}. This is equivalent
to:\\[2\baselineskip]\texttt{\ \ \ \ \ \ \ \ \ \ \ (apply\ vector-append\ \ \ \ \ \ \ \ \ \ \ \ \ \ \ \ \ \ list-of-vectors)\ \ \ \ \ \ \ \ \ }\\[2\baselineskip]However,
it may be implemented
better.\\[2\baselineskip]Example:\\[2\baselineskip]\texttt{\ \ \ \ \ \ \ \ \ \ \ (vector-concatenate\ \textquotesingle{}(\#(a\ b)\ \#(c\ d)))\ \ \ \ \ \ \ \ \ }\\
\texttt{\ \ \ \ \ \ \ \ \ \ \ \#(a\ b\ c\ d)\ \ \ \ \ \ \ \ \ }\\[2\baselineskip]
\end{description}

\href{}{(vector-append-subvectors {[}\emph{vec start end{]} \ldots{}})
-\textgreater{} vector}

Returns a vector that contains every element of each \emph{vec} from
\emph{start} to \emph{end} in the specified order. This procedure is a
generalization of
\texttt{vector-append}.\\[2\baselineskip]Example:\\[2\baselineskip]\texttt{\ \ \ \ \ \ \ \ \ \ \ (vector-append-subvectors\ \textquotesingle{}\#(a\ b\ c\ d\ e)\ 0\ 2\ \textquotesingle{}\#(f\ g\ h\ i\ j)\ 2\ 4)\ \ \ \ \ \ \ \ \ }\\
\texttt{\ \ \ \ \ \ \ \ \ \ \ \#(a\ b\ h\ i)\ \ \ \ \ \ \ \ \ }\\[2\baselineskip]

\subsection{\texorpdfstring{\href{}{5.2.
Predicates}}{5.2. Predicates}}\label{predicates}

\begin{description}
\tightlist
\item[ \href{}{(vector? \emph{x}) -\textgreater{} boolean} ]
{[}\protect\hyperlink{R7RS-small}{\emph{R7RS-small}}{]} Disjoint type
predicate for vectors: this returns \texttt{\#t} if \texttt{x} is a
vector, and \texttt{\#f} if
otherwise.\\[2\baselineskip]Examples:\\[2\baselineskip]\texttt{\ \ \ \ \ \ \ \ \ \ \ (vector?\ \textquotesingle{}\#(a\ b\ c))\ \ \ \ \ \ \ \ \ }\\
\texttt{\ \ \ \ \ \ \ \ \ \ \ \#t\ \ \ \ \ \ \ \ \ }\\[2\baselineskip]\texttt{\ \ \ \ \ \ \ \ \ \ \ (vector?\ \textquotesingle{}(a\ b\ c))\ \ \ \ \ \ \ \ \ }\\
\texttt{\ \ \ \ \ \ \ \ \ \ \ \#f\ \ \ \ \ \ \ \ \ }\\[2\baselineskip]\texttt{\ \ \ \ \ \ \ \ \ \ \ (vector?\ \#t)\ \ \ \ \ \ \ \ \ }\\
\texttt{\ \ \ \ \ \ \ \ \ \ \ \#f\ \ \ \ \ \ \ \ \ }\\[2\baselineskip]\texttt{\ \ \ \ \ \ \ \ \ \ \ (vector?\ \textquotesingle{}\#())\ \ \ \ \ \ \ \ \ }\\
\texttt{\ \ \ \ \ \ \ \ \ \ \ \#t\ \ \ \ \ \ \ \ \ }\\[2\baselineskip]\texttt{\ \ \ \ \ \ \ \ \ \ \ (vector?\ \textquotesingle{}())\ \ \ \ \ \ \ \ \ }\\
\texttt{\ \ \ \ \ \ \ \ \ \ \ \#f\ \ \ \ \ \ \ \ \ }\\[2\baselineskip]
\item[ \href{}{(vector-empty? \emph{vec}) -\textgreater{} boolean} ]
Returns \texttt{\#t} if \texttt{vec} is empty, i.e. its length is
\texttt{0}, and \texttt{\#f} if
not.\\[2\baselineskip]Examples:\\[2\baselineskip]\texttt{\ \ \ \ \ \ \ \ \ \ \ (vector-empty?\ \textquotesingle{}\#(a))\ \ \ \ \ \ \ \ \ }\\
\texttt{\ \ \ \ \ \ \ \ \ \ \ \#f\ \ \ \ \ \ \ \ \ }\\[2\baselineskip]\texttt{\ \ \ \ \ \ \ \ \ \ \ (vector-empty?\ \textquotesingle{}\#(()))\ \ \ \ \ \ \ \ \ }\\
\texttt{\ \ \ \ \ \ \ \ \ \ \ \#f\ \ \ \ \ \ \ \ \ }\\[2\baselineskip]\texttt{\ \ \ \ \ \ \ \ \ \ \ (vector-empty?\ \textquotesingle{}\#(\#()))\ \ \ \ \ \ \ \ \ }\\
\texttt{\ \ \ \ \ \ \ \ \ \ \ \#f\ \ \ \ \ \ \ \ \ }\\[2\baselineskip]\texttt{\ \ \ \ \ \ \ \ \ \ \ (vector-empty?\ \textquotesingle{}\#())\ \ \ \ \ \ \ \ \ }\\
\texttt{\ \ \ \ \ \ \ \ \ \ \ \#t\ \ \ \ \ \ \ \ \ }\\[2\baselineskip]
\item[ \href{}{(vector= \emph{elt=? vec \ldots{}}) -\textgreater{}
boolean} ]
Vector structure comparator, generalized across user-specified element
comparators. Vectors \texttt{a} and \texttt{b} are considered equal by
\texttt{vector=} iff their lengths are the same, and for each respective
element \texttt{Ea} and \texttt{Eb},
\texttt{(elt=?\ Ea\ \ \ \ \ \ \ \ \ Eb)} returns a true value.
\texttt{Elt=?} is always applied to two arguments. Element comparison
must be consistent with \texttt{eq}; that is, if \texttt{(eq?\ Ea\ Eb)}
results in a true value, then \texttt{(elt=?\ Ea\ \ \ \ \ \ \ \ \ Eb)}
must also result in a true value. This may be exploited to avoid
unnecessary element comparisons. (The reference implementation does, but
it does not consider the situation where \texttt{elt=?} is in fact
itself \texttt{eq?} to avoid yet more unnecessary
comparisons.)\\[2\baselineskip]If there are only zero or one vector
arguments, \texttt{\#t} is automatically returned. The dynamic order in
which comparisons of elements and of vectors are performed is left
completely unspecified; do not rely on a particular
order.\\[2\baselineskip]Examples:\\[2\baselineskip]\texttt{\ \ \ \ \ \ \ \ \ \ \ (vector=\ eq?\ \textquotesingle{}\#(a\ b\ c\ d)\ \textquotesingle{}\#(a\ b\ c\ d))\ \ \ \ \ \ \ \ \ }\\
\texttt{\ \ \ \ \ \ \ \ \ \ \ \#t\ \ \ \ \ \ \ \ \ }\\[2\baselineskip]\texttt{\ \ \ \ \ \ \ \ \ \ \ (vector=\ eq?\ \textquotesingle{}\#(a\ b\ c\ d)\ \textquotesingle{}\#(a\ b\ d\ c))\ \ \ \ \ \ \ \ \ }\\
\texttt{\ \ \ \ \ \ \ \ \ \ \ \#f\ \ \ \ \ \ \ \ \ }\\[2\baselineskip]\texttt{\ \ \ \ \ \ \ \ \ \ \ (vector=\ =\ \textquotesingle{}\#(1\ 2\ 3\ 4\ 5)\ \textquotesingle{}\#(1\ 2\ 3\ 4))\ \ \ \ \ \ \ \ \ }\\
\texttt{\ \ \ \ \ \ \ \ \ \ \ \#f\ \ \ \ \ \ \ \ \ }\\[2\baselineskip]\texttt{\ \ \ \ \ \ \ \ \ \ \ (vector=\ =\ \textquotesingle{}\#(1\ 2\ 3\ 4)\ \textquotesingle{}\#(1\ 2\ 3\ 4))\ \ \ \ \ \ \ \ \ }\\
\texttt{\ \ \ \ \ \ \ \ \ \ \ \#t\ \ \ \ \ \ \ \ \ }\\[2\baselineskip]The
two trivial
cases.\\[2\baselineskip]\texttt{\ \ \ \ \ \ \ \ \ \ \ (vector=\ eq?)\ \ \ \ \ \ \ \ \ }\\
\texttt{\ \ \ \ \ \ \ \ \ \ \ \#t\ \ \ \ \ \ \ \ \ }\\[2\baselineskip]\texttt{\ \ \ \ \ \ \ \ \ \ \ (vector=\ eq?\ \textquotesingle{}\#(a))\ \ \ \ \ \ \ \ \ }\\
\texttt{\ \ \ \ \ \ \ \ \ \ \ \#t\ \ \ \ \ \ \ \ \ }\\[2\baselineskip]Note
the fact that we don't use vector literals in the next two --- it is
unspecified whether or not literal vectors with the same external
representation are
\texttt{eq?}.\\[2\baselineskip]\texttt{\ \ \ \ \ \ \ \ \ \ \ (vector=\ eq?\ (vector\ (vector\ \textquotesingle{}a))\ (vector\ (vector\ \textquotesingle{}a)))\ \ \ \ \ \ \ \ \ }\\
\texttt{\ \ \ \ \ \ \ \ \ \ \ \#f\ \ \ \ \ \ \ \ \ }\\[2\baselineskip]\texttt{\ \ \ \ \ \ \ \ \ \ \ (vector=\ equal?\ (vector\ (vector\ \textquotesingle{}a))\ (vector\ (vector\ \textquotesingle{}a)))\ \ \ \ \ \ \ \ \ }\\
\texttt{\ \ \ \ \ \ \ \ \ \ \ \#t\ \ \ \ \ \ \ \ \ }\\[2\baselineskip]
\end{description}

\subsection{\texorpdfstring{\href{}{5.3.
Selectors}}{5.3. Selectors}}\label{selectors}

\begin{description}
\tightlist
\item[ \href{}{(vector-ref \emph{vec i}) -\textgreater{} value} ]
{[}\protect\hyperlink{R7RS-small}{\emph{R7RS-small}}{]} Vector element
dereferencing: returns the value that the location in \texttt{vec} at
\texttt{i} is mapped to in the store. Indexing is based on zero.
\texttt{I} must be within the range {[}0,
\texttt{(vector-length\ \ \ \ \ \ \ \ \ \ \ \ \ \ vec)}).\\[2\baselineskip]Example:\\[2\baselineskip]\texttt{\ \ \ \ \ \ \ \ \ \ \ (vector-ref\ \textquotesingle{}\#(a\ b\ c\ d)\ 2)\ \ \ \ \ \ \ \ \ }\\
\texttt{\ \ \ \ \ \ \ \ \ \ \ c\ \ \ \ \ \ \ \ \ }\\[2\baselineskip]
\item[ \href{}{(vector-length \emph{vec}) -\textgreater{} exact
nonnegative integer} ]
{[}\protect\hyperlink{R7RS-small}{\emph{R7RS-small}}{]} Returns the
length of \texttt{vec}, the number of locations reachable from
\texttt{vec}. (The careful word `reachable' is used to allow for `vector
slices,' whereby \texttt{vec} refers to a larger vector that contains
more locations that are unreachable from \texttt{vec}. This SRFI does
not define vector slices, but later SRFIs
may.)\\[2\baselineskip]Example:\\[2\baselineskip]\texttt{\ \ \ \ \ \ \ \ \ \ \ (vector-length\ \textquotesingle{}\#(a\ b\ c))\ \ \ \ \ \ \ \ \ }\\
\texttt{\ \ \ \ \ \ \ \ \ \ \ 3\ \ \ \ \ \ \ \ \ }\\[2\baselineskip]
\end{description}

\subsection{\texorpdfstring{\href{}{5.4.
Iteration}}{5.4. Iteration}}\label{iteration}

\begin{description}
\tightlist
\item[ \href{}{(vector-fold \emph{kons knil vec\textsubscript{1}
vec\textsubscript{2} \ldots{}}) -\textgreater{} value} ]
The fundamental vector iterator. \texttt{Kons} is iterated over each
value in all of the vectors, stopping at the end of the shortest;
\texttt{kons} is applied as
\texttt{\ \ \ \ \ \ \ \ \ \ \ (kons\ state\ \ \ \ \ \ \ \ \ \ \ \ \ (vector-ref\ \ \ \ \ \ \ \ \ \ \ \ \ \ vec1\ i)\ \ \ \ \ \ \ \ \ \ \ \ \ (vector-ref\ \ \ \ \ \ \ \ \ \ \ \ \ \ vec2\ i)\ \ \ \ \ \ \ \ \ \ \ \ \ ...)\ \ \ \ \ \ \ \ \ }
where \texttt{state} is the current state value --- the current state
value begins with \texttt{knil}, and becomes whatever \texttt{kons}
returned on the previous iteration ---, and \texttt{i} is the current
index.\\[2\baselineskip]The iteration is strictly
left-to-right.\\[2\baselineskip]Examples:\\[2\baselineskip]Find the
longest string's length in \texttt{vector-of-strings}.\\
\texttt{\ \ \ \ \ \ \ \ \ \ \ (vector-fold\ (λ\ (len\ str)\ \ \ \ \ \ \ \ \ \ \ \ \ \ \ \ \ \ \ \ \ \ \ \ \ \ (max\ (string-length\ str)\ len))\ \ \ \ \ \ \ \ \ \ \ ~~~~~~~~~\ \ \ \ \ \ \ \ \ \ \ ~~~~\ \ \ \ \ \ \ \ \ \ \ \ \ \ \ \ \ \ \ \ \ \ \ \ 0\ vector-of-strings)\ \ \ \ \ \ \ \ \ }\\[2\baselineskip]Produce
a list of the reversed elements of \texttt{vec}.\\
\texttt{\ \ \ \ \ \ \ \ \ \ \ (vector-fold\ (λ\ (tail\ elt)\ (cons\ elt\ tail))\ \ \ \ \ \ \ \ \ \ \ ~~~~~~~~~\ \ \ \ \ \ \ \ \ \ \ ~~~~\ \ \ \ \ \ \ \ \ \ \ \ \ \ \ \ \ \ \ \ \ \ \ \ \textquotesingle{}()\ vec)\ \ \ \ \ \ \ \ \ }\\[2\baselineskip]Count
the number of even numbers in \texttt{vec}.\\
\texttt{\ \ \ \ \ \ \ \ \ \ \ (vector-fold\ (λ\ (counter\ n)\ \ \ \ \ \ \ \ \ \ \ ~~~~~~~~~\ \ \ \ \ \ \ \ \ \ \ ~~~~~~\ \ \ \ \ \ \ \ \ \ \ \ \ \ \ \ \ \ \ \ \ \ \ \ \ \ (if\ (even?\ n)\ (+\ counter\ 1)\ counter))\ \ \ \ \ \ \ \ \ \ \ ~~~~~~~~~\ \ \ \ \ \ \ \ \ \ \ ~~~~\ \ \ \ \ \ \ \ \ \ \ \ \ \ \ \ \ \ \ \ \ \ \ \ 0\ vec)\ \ \ \ \ \ \ \ \ }\\[2\baselineskip]
\item[ \href{}{(vector-fold-right \emph{kons knil vec\textsubscript{1}
vec\textsubscript{2} \ldots{}}) -\textgreater{} value} ]
Similar to \texttt{vector-fold}, but it iterates right to left instead
of left to right.\\[2\baselineskip]Example:\\[2\baselineskip]Convert a
vector to a list.\\
\texttt{\ \ \ \ \ \ \ \ \ \ \ (vector-fold-right\ (λ\ (tail\ elt)\ \ \ \ \ \ \ \ \ \ \ \ \ \ \ \ \ \ \ \ \ \ \ \ \ \ \ \ \ \ \ \ (cons\ elt\ tail))\ \ \ \ \ \ \ \ \ \ \ ~~~~~~~~~\ \ \ \ \ \ \ \ \ \ \ ~~~~~~~~~~\ \ \ \ \ \ \ \ \ \ \ \ \ \ \ \ \ \ \ \ \ \ \ \ \ \ \ \ \ \ \textquotesingle{}()\ \textquotesingle{}\#(a\ b\ c\ d))\ \ \ \ \ \ \ \ \ }\\
\texttt{\ \ \ \ \ \ \ \ \ \ \ (a\ b\ c\ d)\ \ \ \ \ \ \ \ \ }\\[2\baselineskip]
\end{description}

\begin{description}
\tightlist
\item[ \href{}{(vector-map \emph{f vec\textsubscript{1}
vec\textsubscript{2} \ldots{}}) -\textgreater{} vector} ]
{[}\protect\hyperlink{R7RS-small}{\emph{R7RS-small}}{]} Constructs a new
vector of the shortest size of the vector arguments. Each element at
index \texttt{i} of the new vector is mapped from the old vectors by
\texttt{(f\ (vector-ref\ \ \ \ \ \ \ \ \ \ \ \ \ \ \ vec1\ \ \ \ \ \ \ \ \ \ \ \ \ \ \ i)\ \ \ \ \ \ \ \ \ \ \ \ \ \ (vector-ref\ \ \ \ \ \ \ \ \ \ \ \ \ \ \ vec2\ \ \ \ \ \ \ \ \ \ \ \ \ \ \ i)\ \ \ \ \ \ \ \ \ \ \ \ \ \ ...)}.
The dynamic order of application of \texttt{f} is
unspecified.\\[2\baselineskip]Examples:\\[2\baselineskip]\texttt{\ \ \ \ \ \ \ \ \ \ \ (vector-map\ (λ\ (x)\ (*\ x\ x))\ \ \ \ \ \ \ \ \ \ \ ~~~~~~~~~\ \ \ \ \ \ \ \ \ \ \ ~~~\ \ \ \ \ \ \ \ \ \ \ \ \ \ \ \ \ \ \ \ \ \ \ (vector-unfold\ \ \ \ \ \ \ \ \ \ \ \ \ \ \ \ \ \ \ \ \ \ \ \ (λ\ (i\ x)\ (values\ x\ (+\ x\ 1)))\ \ \ \ \ \ \ \ \ \ \ \ \ \ \ \ \ \ \ \ \ \ \ \ 4\ 1))\ \ \ \ \ \ \ \ \ }\\
\texttt{\ \ \ \ \ \ \ \ \ \ \ \#(1\ 4\ 9\ 16)\ \ \ \ \ \ \ \ \ }\\[2\baselineskip]\texttt{\ \ \ \ \ \ \ \ \ \ \ (vector-map\ (λ\ (x\ y)\ (*\ x\ y))\ \ \ \ \ \ \ \ \ \ \ ~~~~~~~~~~\ \ \ \ \ \ \ \ \ \ \ ~~~(vector-unfold\ \ \ \ \ \ \ \ \ \ \ \ \ \ \ \ \ \ \ \ \ \ \ \ \ \ \ \ \ \ (λ\ (x)\ (values\ x\ (+\ x\ 1)))\ \ \ \ \ \ \ \ \ \ \ \ \ \ \ \ \ \ \ \ \ \ \ \ \ \ \ \ \ \ 5\ 1)\ \ \ \ \ \ \ \ \ \ \ ~~~~~~~~~~\ \ \ \ \ \ \ \ \ \ \ ~~~(vector-unfold\ \ \ \ \ \ \ \ \ \ \ \ \ \ \ \ \ \ \ \ \ \ \ \ \ \ \ \ \ \ (λ\ (x)\ (values\ x\ (-\ x\ 1)))\ \ \ \ \ \ \ \ \ \ \ \ \ \ \ \ \ \ \ \ \ \ \ \ \ \ \ \ \ \ 5\ 5))\ \ \ \ \ \ \ \ \ }\\
\texttt{\ \ \ \ \ \ \ \ \ \ \ \#(5\ 8\ 9\ 8\ 5)\ \ \ \ \ \ \ \ \ }\\[2\baselineskip]\texttt{\ \ \ \ \ \ \ \ \ \ \ (let\ ((count\ 0))\ \ \ \ \ \ \ \ \ }\\
\texttt{\ \ \ \ \ \ \ \ \ \ \ ~~\ \ \ \ \ \ \ \ \ \ \ (vector-map\ (λ\ (ignored-elt)\ \ \ \ \ \ \ \ \ }\\
\texttt{\ \ \ \ \ \ \ \ \ \ \ ~~~~~~~~~~\ \ \ \ \ \ \ \ \ \ \ ~~~~~\ \ \ \ \ \ \ \ \ \ \ (set!\ count\ (+\ count\ 1))\ \ \ \ \ \ \ \ \ }\\
\texttt{\ \ \ \ \ \ \ \ \ \ \ ~~~~~~~~~~\ \ \ \ \ \ \ \ \ \ \ ~~~~~\ \ \ \ \ \ \ \ \ \ \ count)\ \ \ \ \ \ \ \ \ }\\
\texttt{\ \ \ \ \ \ \ \ \ \ \ ~~~~~~~~~~\ \ \ \ \ \ \ \ \ \ \ ~~~\ \ \ \ \ \ \ \ \ \ \ \textquotesingle{}\#(a\ b)))\ \ \ \ \ \ \ \ \ }\\
\texttt{\ \ \ \ \ \ \ \ \ \ \ \#(1\ 2)\ OR\ \#(2\ 1)\ \ \ \ \ \ \ \ \ }\\[2\baselineskip]
\end{description}

\begin{description}
\tightlist
\item[ \href{}{(vector-map! \emph{f vec\textsubscript{1}
vec\textsubscript{2} \ldots{}}) -\textgreater{} unspecified} ]
Similar to \texttt{vector-map}, but rather than mapping the new elements
into a new vector, the new mapped elements are destructively inserted
into \texttt{vec1}. Again, the dynamic order of application of
\texttt{f} unspecified, so it is dangerous for \texttt{f} to apply
either \texttt{vector-ref} or \texttt{vector-set!} to \texttt{vec1} in
\texttt{f}.\\
\end{description}

\begin{description}
\item[ \href{}{(vector-for-each \emph{f vec\textsubscript{1}
vec\textsubscript{2} \ldots{}}) -\textgreater{} unspecified} ]
{[}\protect\hyperlink{R7RS-small}{\emph{R7RS-small}}{]} Simple vector
iterator: applies \texttt{f} to the corresponding list of parallel
elements from \texttt{vec1\ vec2\ \ \ \ \ \ \ \ \ ...} in the range
{[}0, \texttt{length}), where \texttt{length} is the length of the
smallest vector argument passed, In contrast with \texttt{vector-map},
\texttt{f} is reliably applied to each subsequent element, starting at
index 0, in the
vectors.\\[2\baselineskip]Example:\\[2\baselineskip]\texttt{\ \ \ \ \ \ \ \ \ \ \ (vector-for-each\ (λ\ (x)\ (display\ x)\ (newline))\ \ \ \ \ \ \ \ \ }\\
\texttt{\ \ \ \ \ \ \ \ \ \ \ ~~~~~~~~~~\ \ \ \ \ \ \ \ \ \ \ ~~~~~\ \ \ \ \ \ \ \ \ \ \ \textquotesingle{}\#("foo"\ "bar"\ "baz"\ "quux"\ "zot"))\ \ \ \ \ \ \ \ \ }\\
Displays:\\

\begin{verbatim}
foo
bar
baz
quux
zot
\end{verbatim}
\end{description}

\begin{description}
\tightlist
\item[ \href{}{(vector-count \emph{pred? vec\textsubscript{1}
vec\textsubscript{2} \ldots{}}) -\textgreater{} exact nonnegative
integer} ]
Counts the number of parallel elements in the vectors that satisfy
\texttt{pred?}, which is applied, for each index \texttt{i} in the range
{[}0, \texttt{length}) where \texttt{length} is the length of the
smallest vector argument, to each parallel element in the vectors, in
order.\\[2\baselineskip]Examples:\\[2\baselineskip]\texttt{\ \ \ \ \ \ \ \ \ \ \ (vector-count\ even?\ \ \ \ \ \ \ \ \ \ \ \ \ \ \ \ \ \ \ \ \ \ \ \ \ \textquotesingle{}\#(3\ 1\ 4\ 1\ 5\ 9\ 2\ 5\ 6))\ \ \ \ \ \ \ \ \ }\\
\texttt{\ \ \ \ \ \ \ \ \ \ \ 3\ \ \ \ \ \ \ \ \ }\\[2\baselineskip]\texttt{\ \ \ \ \ \ \ \ \ \ \ (vector-count\ \textless{}\ \ \ \ \ \ \ \ \ \ \ \ \ \ \ \ \ \ \ \ \ \ \ \ \ \textquotesingle{}\#(1\ 3\ 6\ 9)\ \textquotesingle{}\#(2\ 4\ 6\ 8\ 10\ 12))\ \ \ \ \ \ \ \ \ }\\
\texttt{\ \ \ \ \ \ \ \ \ \ \ 2\ \ \ \ \ \ \ \ \ }\\[2\baselineskip]
\end{description}

\begin{description}
\item[ \href{}{(vector-cumulate \emph{f knil vec}) -\textgreater{}
vector} ]
Returns a newly allocated vector \texttt{new} with the same length as
\texttt{vec}. Each element \emph{i} of \emph{new} is set to the result
of invoking \texttt{f} on \emph{\texttt{new}\textsubscript{i-1}} and
\emph{\texttt{vec}\textsubscript{i}}, except that for the first call on
\emph{f}, the first argument is \emph{knil}. The \emph{new} vector is
returned.

Note that the order of arguments to \texttt{vector-cumulate} was changed
by \texttt{errata-3} on 2016/9/2.

Example:\\[2\baselineskip]\texttt{\ \ \ \ \ \ \ \ \ \ \ (vector-cumulate\ +\ 0\ \textquotesingle{}\#(3\ 1\ 4\ 1\ 5\ 9\ 2\ 5\ 6))\ \ \ \ \ \ \ \ \ }\\
\texttt{\ \ \ \ \ \ \ \ \ \ \ \#(3\ 4\ 8\ 9\ 14\ 23\ 25\ 30\ 36)\ \ \ \ \ \ \ \ \ }\\[4\baselineskip]
\end{description}

\subsection{\texorpdfstring{\href{}{5.5.
Searching}}{5.5. Searching}}\label{searching}

\begin{description}
\tightlist
\item[ \href{}{(vector-index \emph{pred? vec\textsubscript{1}
vec\textsubscript{2} \ldots{}}) -\textgreater{} exact nonnegative
integer or \#f} ]
Finds \& returns the index of the first elements in
\texttt{vec1\ vec2\ \ \ \ \ \ \ \ \ ...} that satisfy \texttt{pred?}. If
no matching element is found by the end of the shortest vector,
\texttt{\#f} is
returned.\\[2\baselineskip]Examples:\\[2\baselineskip]\texttt{\ \ \ \ \ \ \ \ \ \ \ (vector-index\ even?\ \textquotesingle{}\#(3\ 1\ 4\ 1\ 5\ 9))\ \ \ \ \ \ \ \ \ }\\
\texttt{\ \ \ \ \ \ \ \ \ \ \ 2\ \ \ \ \ \ \ \ \ }\\[2\baselineskip]\texttt{\ \ \ \ \ \ \ \ \ \ \ (vector-index\ \textless{}\ \textquotesingle{}\#(3\ 1\ 4\ 1\ 5\ 9\ 2\ 5\ 6)\ \textquotesingle{}\#(2\ 7\ 1\ 8\ 2))\ \ \ \ \ \ \ \ \ }\\
\texttt{\ \ \ \ \ \ \ \ \ \ \ 1\ \ \ \ \ \ \ \ \ }\\[2\baselineskip]\texttt{\ \ \ \ \ \ \ \ \ \ \ (vector-index\ =\ \textquotesingle{}\#(3\ 1\ 4\ 1\ 5\ 9\ 2\ 5\ 6)\ \textquotesingle{}\#(2\ 7\ 1\ 8\ 2))\ \ \ \ \ \ \ \ \ }\\
\texttt{\ \ \ \ \ \ \ \ \ \ \ \#f\ \ \ \ \ \ \ \ \ }\\[2\baselineskip]
\item[ \href{}{(vector-index-right \emph{pred? vec\textsubscript{1}
vec\textsubscript{2} \ldots{}}) -\textgreater{} exact nonnegative
integer or \#f} ]
Like \texttt{vector-index}, but it searches right-to-left, rather than
left-to-right, and all of the vectors \emph{must} have the same
length.\\[2\baselineskip]
\item[ \href{}{(vector-skip \emph{pred? vec\textsubscript{1}
vec\textsubscript{2} \ldots{}}) -\textgreater{} exact nonnegative
integer or \#f} ]
Finds \& returns the index of the first elements in
\texttt{vec1\ vec2\ \ \ \ \ \ \ \ \ ...} that do \emph{not} satisfy
\texttt{pred?}. If all the values in the vectors satisfy \texttt{pred?}
until the end of the shortest vector, this returns \texttt{\#f}. This is
equivalent
to:\\[2\baselineskip]\texttt{\ \ \ \ \ \ \ \ \ \ \ (vector-index\ \ \ \ \ \ \ \ \ \ \ \ (λ\ (x1\ x2\ \ \ \ \ \ \ \ \ \ \ \ \ \ \ \ \ \ \ \ \ \ \ \ ...)\ \ \ \ \ \ \ \ \ \ \ \ \ \ (not\ (pred?\ x1\ \ \ \ \ \ \ \ \ \ \ \ \ \ \ \ \ \ \ \ \ \ \ \ \ \ \ \ \ \ \ \ \ x1\ \ \ \ \ \ \ \ \ \ \ \ \ \ \ \ \ \ \ \ \ \ \ \ \ \ \ \ \ \ \ \ \ \ ...)))\ \ \ \ \ \ \ \ \ \ \ \ ~~~~~~~~~\ \ \ \ \ \ \ \ \ \ \ \ ~~~~~~~~~\ \ \ \ \ \ \ \ \ \ \ \ vec1\ vec2\ \ \ \ \ \ \ \ \ \ \ \ ...)\ \ \ \ \ \ \ \ \ }\\[2\baselineskip]Example:\\[2\baselineskip]\texttt{\ \ \ \ \ \ \ \ \ \ \ (vector-skip\ number?\ \textquotesingle{}\#(1\ 2\ a\ b\ 3\ 4\ c\ d))\ \ \ \ \ \ \ \ \ }\\
\texttt{\ \ \ \ \ \ \ \ \ \ \ 2\ \ \ \ \ \ \ \ \ }\\[2\baselineskip]
\item[ \href{}{(vector-skip-right \emph{pred? vec\textsubscript{1}
vec\textsubscript{2} \ldots{}}) -\textgreater{} exact nonnegative
integer or \#f} ]
Like \texttt{vector-skip}, but it searches for a non-matching element
right-to-left, rather than left-to-right, and it is an error if all of
the vectors do not have the same length. This is equivalent
to:\\[2\baselineskip]\texttt{\ \ \ \ \ \ \ \ \ \ \ (vector-index-right\ \ \ \ \ \ \ \ \ \ \ \ (λ\ (x1\ x2\ \ \ \ \ \ \ \ \ \ \ \ \ \ \ \ \ \ \ \ \ \ \ \ ...)\ \ \ \ \ \ \ \ \ \ \ \ \ \ (not\ (pred?\ x1\ \ \ \ \ \ \ \ \ \ \ \ \ \ \ \ \ \ \ \ \ \ \ \ \ \ \ \ \ \ \ \ \ x1\ \ \ \ \ \ \ \ \ \ \ \ \ \ \ \ \ \ \ \ \ \ \ \ \ \ \ \ \ \ \ \ \ \ ...)))\ \ \ \ \ \ \ \ \ \ \ \ ~~~~~~~~~\ \ \ \ \ \ \ \ \ \ \ \ ~~~~~~~~~~\ \ \ \ \ \ \ \ \ \ \ \ ~~~~\ \ \ \ \ \ \ \ \ \ \ \ vec1\ vec2\ \ \ \ \ \ \ \ \ \ \ \ ...)\ \ \ \ \ \ \ \ \ }\\[2\baselineskip]
\item[ \href{}{(vector-binary-search \emph{vec value cmp})
-\textgreater{} exact nonnegative integer or \#f} ]
Similar to \texttt{vector-index} and \texttt{vector-index-right}, but
instead of searching left to right or right to left, this performs a
binary search. If there is more than one element of \emph{vec} that
matches \emph{value} in the sense of \emph{cmp},
\texttt{vector-binary-search} may return the index of any of them.

\texttt{cmp} should be a procedure of two arguments and return a
negative integer, which indicates that its first argument is less than
its second, zero, which indicates that they are equal, or a positive
integer, which indicates that the first argument is greater than the
second argument. An example \texttt{cmp} might
be:\\[2\baselineskip]\texttt{\ \ \ \ \ \ \ \ \ \ \ (λ\ (char1\ char2)\ \ \ \ \ \ \ \ \ }\\
\texttt{\ \ \ \ \ \ \ \ \ \ \ ~~(cond\ ((char\textless{}?\ char1\ \ \ \ \ \ \ \ \ \ \ \ \ \ \ \ \ \ \ \ \ \ \ \ \ \ \ \ \ \ \ \ \ \ \ \ \ \ \ \ \ \ \ \ char2)\ \ \ \ \ \ \ \ \ \ \ \ \ \ \ \ \ \ \ \ \ \ \ \ \ \ \ \ \ \ -1)\ \ \ \ \ \ \ \ \ }\\
\texttt{\ \ \ \ \ \ \ \ \ \ \ ~~~~~~~\ \ \ \ \ \ \ \ \ \ \ ((char=?\ char1\ \ \ \ \ \ \ \ \ \ \ \ \ \ \ \ \ \ \ \ \ \ \ char2)\ \ \ \ \ \ \ \ \ \ \ \ 0)\ \ \ \ \ \ \ \ \ }\\
\texttt{\ \ \ \ \ \ \ \ \ \ \ ~~~~~~~\ \ \ \ \ \ \ \ \ \ \ (else\ 1)))\ \ \ \ \ \ \ \ \ }\\[2\baselineskip]
\item[ \href{}{(vector-any \emph{pred? vec\textsubscript{1}
vec\textsubscript{2} \ldots{}}) -\textgreater{} value or \#f} ]
Finds the first set of elements in parallel from
\texttt{vec1\ vec2\ \ \ \ \ \ \ \ \ ...} for which \texttt{pred?}
returns a true value. If such a parallel set of elements exists,
\texttt{vector-any} returns the value that \texttt{pred?} returned for
that set of elements. The iteration is strictly
left-to-right.\\[2\baselineskip]
\item[ \href{}{(vector-every \emph{pred? vec\textsubscript{1}
vec\textsubscript{2} \ldots{}}) -\textgreater{} value or \#f} ]
If, for every index \texttt{i} between 0 and the length of the shortest
vector argument, the set of elements
\texttt{(vector-ref\ vec1\ \ \ \ \ \ \ \ \ \ \ \ \ \ \ \ \ \ \ \ \ \ \ \ \ \ \ \ \ \ \ \ \ \ \ \ \ \ \ \ \ \ \ \ \ \ \ \ \ \ i)\ \ \ \ \ \ \ \ \ \ \ \ \ (vector-ref\ vec2\ \ \ \ \ \ \ \ \ \ \ \ \ \ \ \ \ \ \ \ \ \ \ \ \ \ \ \ \ \ \ \ \ \ \ \ \ \ \ \ \ \ \ \ \ \ \ \ \ \ i)\ \ \ \ \ \ \ \ \ \ \ \ \ ...}
satisfies \texttt{pred?}, \texttt{vector-every} returns the value that
\texttt{pred?} returned for the last set of elements, at the last index
of the shortest vector. The iteration is strictly
left-to-right.\\[2\baselineskip]
\item[ \href{}{(vector-partition \emph{pred? vec}) -\textgreater{}
vector} ]
A vector the same size as \texttt{vec} is newly allocated and filled
with all the elements of \texttt{vec} that satisfy \texttt{pred?} in
their original order followed by all the elements that do not satisfy
\texttt{pred}, also in their original order.\\[2\baselineskip]Two values
are returned, the newly allocated vector and the index of the leftmost
element that does not satisfy \texttt{pred}.\\[2\baselineskip]
\end{description}

\subsection{\texorpdfstring{\href{}{5.7.
Mutators}}{5.7. Mutators}}\label{mutators}

\href{}{(vector-set! \emph{vec i value}) -\textgreater{} unspecified}

{[}\protect\hyperlink{R7RS-small}{\emph{R7RS-small}}{]} Assigns the
contents of the location at \texttt{i} in \texttt{vec} to
\texttt{value}.\\[2\baselineskip]

\href{}{(vector-swap! \emph{vec i j}) -\textgreater{} unspecified}

Swaps or exchanges the values of the locations in \texttt{vec} at
\texttt{i} \& \texttt{j}.\\[2\baselineskip]

\href{}{(vector-fill! \emph{vec fill} {[}\emph{start}
{[}\emph{end}{]}{]}) -\textgreater{} unspecified}

{[}\protect\hyperlink{R7RS-small}{\emph{R7RS-small}}{]} Assigns the
value of every location in \texttt{vec} between \texttt{start}, which
defaults to \texttt{0} and \texttt{end}, which defaults to the length of
\texttt{vec}, to \texttt{fill}.\\[2\baselineskip]

\href{}{(vector-reverse! \emph{vec} {[}\emph{start} {[}\emph{end}{]}{]})
-\textgreater{} unspecified}

Destructively reverses the contents of the sequence of locations in
\texttt{vec} between \texttt{start} and \texttt{end}. \texttt{Start}
defaults to \texttt{0} and \texttt{end} defaults to the length of
\texttt{vec}. Note that this does not deeply reverse.\\[2\baselineskip]

\href{}{(vector-copy! \emph{to at from} {[}\emph{start}
{[}\emph{end}{]}{]}) -\textgreater{} unspecified}

{[}\protect\hyperlink{R7RS-small}{\emph{R7RS-small}}{]} Copies the
elements of vector \texttt{from} between \texttt{start} and \texttt{end}
to vector \texttt{to}, starting at \texttt{at}. The order in which
elements are copied is unspecified, except that if the source and
destination overlap, copying takes place as if the source is first
copied into a temporary vector and then into the destination. This can
be achieved without allocating storage by making sure to copy in the
correct direction in such circumstances.\\[2\baselineskip]

\href{}{(vector-reverse-copy! \emph{to at from} {[}\emph{start}
{[}\emph{end}{]}{]}) -\textgreater{} unspecified}

Like \texttt{vector-copy!}, but the elements appear in \texttt{to} in
reverse order.
\texttt{\ \ \ \ \ \ \ \ \ \ \ (vector-reverse!\ \ \ \ \ \ \ \ \ \ \ \ target\ \ \ \ \ \ \ \ \ \ \ \ tstart\ \ \ \ \ \ \ \ \ \ \ \ send)\ \ \ \ \ \ \ \ \ }
would.\\[2\baselineskip]

\href{}{(vector-unfold! f vec start end initial-seed \ldots{})
-\textgreater{} unspecified}

Like \texttt{vector-unfold}, but the elements are copied into the vector
\emph{vec} starting at element \emph{start} rather than into a newly
allocated vector. Terminates when \emph{end-start} elements have been
generated.\\[2\baselineskip]

\href{}{(vector-unfold-right! f vec start end initial-seed \ldots{})
-\textgreater{} unspecified}

Like \texttt{vector-unfold!}, but the elements are copied in reverse
order into the vector \emph{vec} starting at the index preceding
\emph{end}.\\[2\baselineskip]

\subsection{\texorpdfstring{\href{}{5.8.
Conversion}}{5.8. Conversion}}\label{conversion}

\begin{description}
\tightlist
\item[ \href{}{(vector-\textgreater{}list \emph{vec} {[}\emph{start}
{[}\emph{end}{]}{]}) -\textgreater{} proper-list} ]
{[}\protect\hyperlink{R7RS-small}{\emph{R7RS-small}}{]} Creates a list
containing the elements in \texttt{vec} between \texttt{start}, which
defaults to \texttt{0}, and \texttt{end}, which defaults to the length
of \texttt{vec}.\\[2\baselineskip]
\item[ \href{}{(reverse-vector-\textgreater{}list \emph{vec}
{[}\emph{start} {[}\emph{end}{]}{]}) -\textgreater{} proper-list} ]
Like \texttt{vector-\textgreater{}list}, but the resulting list contains
the elements in reverse of \texttt{vector}.\\[2\baselineskip]
\item[ \href{}{(list-\textgreater{}vector \emph{proper-list})
-\textgreater{} vector} ]
{[}\protect\hyperlink{R7RS-small}{\emph{R7RS-small}}{]} Creates a vector
of elements from \texttt{proper-list}.\\[2\baselineskip]
\item[ \href{}{(reverse-list-\textgreater{}vector \emph{proper-list})
-\textgreater{} vector} ]
Like \texttt{list-\textgreater{}vector}, but the resulting vector
contains the elements in reverse of
\texttt{proper-list}.\\[2\baselineskip]
\item[ \href{}{(string-\textgreater{}vector \emph{string}
{[}\emph{start} {[}\emph{end}{]}{]}) -\textgreater{} vector} ]
{[}\protect\hyperlink{R7RS-small}{\emph{R7RS-small}}{]} Creates a vector
containing the elements in \texttt{string} between \texttt{start}, which
defaults to \texttt{0}, and \texttt{end}, which defaults to the length
of \texttt{string}.\\[2\baselineskip]
\item[ \href{}{(vector-\textgreater{}string \emph{vec} {[}\emph{start}
{[}\emph{end}{]}{]}) -\textgreater{} string} ]
{[}\protect\hyperlink{R7RS-small}{\emph{R7RS-small}}{]} Creates a string
containing the elements in \texttt{vec} between \texttt{start}, which
defaults to \texttt{0}, and \texttt{end}, which defaults to the length
of \texttt{vec}. It is an error if the elements are not
characters.\\[2\baselineskip]
\end{description}

\section{\texorpdfstring{\href{}{6. Sample
Implementation}}{6. Sample Implementation}}\label{sample-implementation}

The sample implementation is in the repository of this SRFI. It has only
one non-R5RS dependency: \protect\hyperlink{SRFI-23}{SRFI 23}'s
\texttt{error} procedure, which is also provided by R7RS-small. It is in
the public domain, or alternatively under the same copyright as this
SRFI. The following files are provided:

\begin{itemize}
\tightlist
\item
  \texttt{vectors-impl.scm} - a modified version of the implementation
  of SRFI 43
\item
  \texttt{vectors.scm} - a Chicken library showing what to export for an
  R5RS implementation
\item
  \texttt{vectors.sld} - an R7RS library that excludes what R7RS-small
  already provides
\item
  \texttt{vectors-test.scm} - tests using the Chicken test egg (also
  available on Chibi)
\end{itemize}

\section{\texorpdfstring{\href{}{7.
Acknowledgements}}{7. Acknowledgements}}\label{acknowledgements}

These acknowledgements are copied from SRFI 43.

Thanks to Olin Shivers for his wonderfully complete
\protect\hyperlink{SRFI-1}{list} and \protect\hyperlink{SRFI-13}{string}
packages; to all the members of the
\href{http://scheme-irc.webhop.org/}{\texttt{\#scheme} IRC channel} on
\href{http://www.freenode.net/}{Freenode} who nitpicked a great deal,
but also helped quite a lot in general, and helped test the reference
implementation in various Scheme systems; to Michael Burschik for his
numerous comments; to Sergei Egorov for helping to narrow down the
procedures; to Mike Sperber for putting up with an \emph{extremely}
overdue draft; to Felix Winkelmann for continually bugging me about
finishing up the SRFI so that it would be only overdue and not
withdrawn; and to everyone else who gave questions, comments, thoughts,
or merely attention to the SRFI.

\section{\texorpdfstring{\href{}{8.
References}}{8. References}}\label{references}

\begin{description}
\tightlist
\item[\href{}{R5RS}]
\emph{R5RS: The Revised\textsuperscript{5} Report on Scheme}\\
R. Kelsey, W. Clinger, J. Rees (editors).\\
Higher-Order and Symbolic Computation, Vol. 11, No. 1, September, 1998\\
and\\
ACM SIGPLAN Notices, Vol. 33, No. 9, October, 1998\\
Available at:
\url{http://www.schemers.org/Documents/Standards/R5RS/}\\[2\baselineskip]
\item[\href{}{R7RS-small}]
\emph{R7RS: The Revised\textsuperscript{7} Report on Scheme}\\
A. Shinn et al. (editors).\\
Available at: \url{http://r7rs.org}\\[2\baselineskip]
\item[\href{}{SRFI}]
\emph{SRFI: Scheme Request for Implementation}\\
The SRFI website can be found at: \url{http://srfi.schemers.org/}\\
The SRFIs mentioned in this document are described
later.\\[2\baselineskip]
\item[\href{}{SRFI 1}]
\emph{SRFI 1: List Library}\\
A SRFI of list processing procedures, written by Olin Shivers.\\
Available at: \url{http://srfi.schemers.org/srfi-1/}\\[2\baselineskip]
\item[\href{}{SRFI 13}]
\emph{SRFI 13: String Library}\\
A SRFI of string processing procedures, written by Olin Shivers.\\
Available at: \url{http://srfi.schemers.org/srfi-13/}\\[2\baselineskip]
\item[\href{}{SRFI 23}]
\emph{SRFI 23: Error Reporting Mechanism}\\
A SRFI that defines a new primitive (\texttt{error}) for reporting that
an error occurred, written by Stephan Houben.\\
Available at: \url{http://srfi.schemers.org/srfi-23/}\\[2\baselineskip]
\item[\href{}{SRFI 43}]
\emph{SRFI 43: Vector Library (draft)}\\
The direct predecessor of this SRFI, written by Taylor Campbell.\\
Available at: \url{http://srfi.schemers.org/srfi-43/}
\end{description}

\section{\texorpdfstring{\href{}{9.
Copyright}}{9. Copyright}}\label{copyright}

Copyright (C) Taylor Campbell (2003). All rights reserved.

Permission is hereby granted, free of charge, to any person obtaining a
copy of this software and associated documentation files (the
``Software''), to deal in the Software without restriction, including
without limitation the rights to use, copy, modify, merge, publish,
distribute, sublicense, and/or sell copies of the Software, and to
permit persons to whom the Software is furnished to do so, subject to
the following conditions:

The above copyright notice and this permission notice shall be included
in all copies or substantial portions of the Software.

THE SOFTWARE IS PROVIDED ``AS IS'', WITHOUT WARRANTY OF ANY KIND,
EXPRESS OR IMPLIED, INCLUDING BUT NOT LIMITED TO THE WARRANTIES OF
MERCHANTABILITY, FITNESS FOR A PARTICULAR PURPOSE AND NONINFRINGEMENT.
IN NO EVENT SHALL THE AUTHORS OR COPYRIGHT HOLDERS BE LIABLE FOR ANY
CLAIM, DAMAGES OR OTHER LIABILITY, WHETHER IN AN ACTION OF CONTRACT,
TORT OR OTHERWISE, ARISING FROM, OUT OF OR IN CONNECTION WITH THE
SOFTWARE OR THE USE OR OTHER DEALINGS IN THE SOFTWARE.

\begin{center}\rule{0.5\linewidth}{\linethickness}\end{center}

Editor:
\href{mailto:srfi\%20minus\%20editors\%20at\%20srfi\%20dot\%20schemers\%20dot\%20org}{Arthur
A. Gleckler}
