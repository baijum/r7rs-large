\section{Title}\label{title}

Immutable Deques

\section{Author}\label{author}

Kevin Wortman, John Cowan

\section{Status}\label{status}

This SRFI is currently in \emph{final} status. Here is
\href{http://srfi.schemers.org/srfi-process.html}{an explanation} of
each status that a SRFI can hold. To provide input on this SRFI, please
send email to \texttt{srfi-134@nospamsrfi.schemers.org}. To subscribe to
the list, follow
\href{http://srfi.schemers.org/srfi-list-subscribe.html}{these
instructions}. You can access previous messages via the mailing list
\href{http://srfi-email.schemers.org/srfi-134}{archive}.

\begin{itemize}
\tightlist
\item
  Received: 2015/12/9
\item
  60-day deadline: 2016/2/7
\item
  Draft \#1 published: 2015/12/9
\item
  Draft \#2 published: 2016/1/20
\item
  Draft \#3 published: 2016/1/31
\item
  Draft \#4 published: 2016/2/6
\item
  Draft \#5 published: 2016/2/9 (mostly code changes)
\item
  Draft \#6 published: 2016/4/7
\item
  Draft \#7 published: 2016/4/12
\item
  Draft \#8 published: 2016/5/9
\item
  Draft \#9 published: 2016/5/28
\item
  Finalized: 2016/7/1
\end{itemize}

\section{Abstract}\label{abstract}

This SRFI defines immutable deques. A deque is a double-ended queue, a
sequence which allows elements to be added or removed efficiently from
either end. A structure is immutable when all its operations leave the
structure unchanged. Note that none of the procedures specified here
ends with an exclamation point.

\section{Rationale}\label{rationale}

A double-ended queue, or \emph{deque} (pronounced ``deck'') is a
sequential data structure which allows elements to be added or removed
from either end in amortized O(1) time. It is a generalization of both a
queue and a stack, and can be used as either by disregarding the
irrelevant procedures.

This SRFI describes immutable deques, or \emph{ideques}. Immutable
structures are sometimes called \emph{persistent} and are closely
related to \emph{pure functional} (a.k.a. \emph{pure}) structures. The
availability of immutable data structures facilitates writing efficient
programs in the pure-functional style. Immutable deques can also be seen
as a bidirectional generalization of immutable lists, and some of the
procedures documented below are most useful in that context. Unlike the
immutable lists of
\href{http://srfi.schemers.org/srfi-116/srfi-116.html}{SRFI 116}, it is
efficient to produce modified versions of an ideque; unlike the list
queues of \href{http://srfi.schemers.org/srfi-117/srfi-117.html}{SRFI
117}, it is possible to efficiently return an updated version of an
ideque without mutating any earlier versions of it.

The specification was designed jointly by Kevin Wortman and John Cowan.
John Cowan is the editor and shepherd. The two-list implementation was
written by John Cowan.

\section{Procedure index}\label{procedure-index}

\protect\hyperlink{Constructors}{Constructors}:
\texttt{ideque,\ ideque-tabulate,\ ideque-unfold,\ ideque-unfold-right}

\protect\hyperlink{Predicates}{Predicates}:
\texttt{ideque?,\ ideque-empty?,\ ideque=,\ ideque-any,\ ideque-every}

\protect\hyperlink{Queueoperations}{Queue operations}:
\texttt{ideque-front,\ ideque-back,\ ideque-remove-front,\ ideque-remove-back,\ ideque-add-front,\ ideque-add-back}

\protect\hyperlink{Otheraccessors}{Other accessors}:
\texttt{ideque-ref,\ ideque-take,\ ideque-take-right,\ ideque-drop,\ ideque-drop-right,\ ideque-split-at}

\protect\hyperlink{Thewholedeque}{The whole ideque}:
\texttt{ideque-length,\ ideque-append,\ ideque-reverse,\ ideque-count,\ ideque-zip}

\protect\hyperlink{Mapping}{Mapping}:
\texttt{ideque-map,\ ideque-filter-map,\ ideque-for-each,\ ideque-for-each-right,\ ideque-fold,\ ideque-fold-right,\ ideque-append-map}

\protect\hyperlink{Filtering}{Filtering}:
\texttt{ideque-filter,\ ideque-remove,\ ideque-partition}

\protect\hyperlink{Searching}{Searching}:
\texttt{ideque-find,\ ideque-find-right,\ ideque-take-while,\ ideque-take-while-right,\ ideque-drop-while,\ ideque-drop-while-right,\ ideque-span,\ ideque-break}

\protect\hyperlink{Conversion}{Conversion}:
\texttt{list-\textgreater{}ideque,\ ideque-\textgreater{}list,\ generator-\textgreater{}ideque,\ ideque-\textgreater{}generator}

\section{Specification}\label{specification}

We specify required time efficiency upper bounds using big-O notation.
We note when, in some cases, there is ``slack'' between the required
bound and the theoretically optimal bound for an operation.
Implementations may use data structures with amortized time bounds, but
should document which bounds hold in only an amortized sense.

Deques are disjoint from all other Scheme types.

\hypertarget{Constructors}{\subsection{Constructors}\label{Constructors}}

\texttt{(ideque\ } \emph{element} \ldots{}\texttt{)}

Returns an ideque containing the \emph{elements}. The first element (if
any) will be at the front of the ideque and the last element (if any)
will be at the back. Takes O(n) time, where \emph{n} is the number of
elements.

\texttt{(ideque-tabulate\ }\emph{n proc}\texttt{)}

Invokes the predicate \emph{proc} on every exact integer from 0
(inclusive) to \emph{n} (exclusive). Returns an ideque containing the
results in order of generation. Takes O(n) time.

\texttt{(ideque-unfold\ }\emph{stop? mapper successor seed}\texttt{)}

Invokes the predicate \emph{stop?} on \emph{seed}. If it returns false,
generate the next result by applying \emph{mapper} to \emph{seed},
generate the next seed by applying \emph{successor} to \emph{seed}, and
repeat this algorithm with the new seed. If \emph{stop?} returns true,
return an ideque containing the results in order of accumulation. Takes
O(n) time.

\texttt{(ideque-unfold-right\ }\emph{stop? mapper successor
seed}\texttt{)}

Invokes the predicate \emph{stop?} on \emph{seed}. If it returns false,
generate the next result by applying \emph{mapper} to \emph{seed},
generate the next seed by applying \emph{successor} to \emph{seed}, and
repeat the algorithm with the new seed. If \emph{stop?} returns true,
return an ideque containing the results in reverse order of
accumulation. Takes O(n) time.

\hypertarget{Predicates}{\subsection{Predicates}\label{Predicates}}

\texttt{(ideque?\ }\emph{x}\texttt{)}

Returns \texttt{\#t} if \emph{x} is an ideque, and \texttt{\#f}
otherwise. Takes O(1) time.

\texttt{(ideque-empty?\ }\emph{ideque}\texttt{)}

Returns \texttt{\#t} if \emph{ideque} contains zero elements, and
\texttt{\#f} otherwise. Takes O(1) time.

\texttt{(ideque=\ }\emph{elt= ideque \ldots{}}\texttt{)}

Determines ideque equality, given an element-equality procedure. Ideque
A equals ideque B if they are of the same length, and their
corresponding elements are equal, as determined by \emph{elt=}. If the
element-comparison procedure's first argument is from
\emph{ideque\textsubscript{i}}, then its second argument is from
\emph{ideque\textsubscript{i+1}}, i.e. it is always called as
\texttt{(}\emph{elt= a b}\texttt{)} for \emph{a} an element of ideque A,
and \emph{b} an element of ideque B.

In the n-ary case, every \emph{ideque\textsubscript{i}} is compared to
\emph{ideque\textsubscript{i+1}} (as opposed, for example, to comparing
\emph{ideque\textsubscript{1}} to every \emph{ideque\textsubscript{i}},
for i \textgreater{} 1). If there are zero or one ideque arguments,
\texttt{ideque=} simply returns true. The name does not end in a
question mark for compatibility with the SRFI-1 procedure
\texttt{list=}.

Note that the dynamic order in which the \emph{elt=} procedure is
applied to pairs of elements is not specified. For example, if
\texttt{ideque=} is applied to three ideques, A, B, and C, it may first
completely compare A to B, then compare B to C, or it may compare the
first elements of A and B, then the first elements of B and C, then the
second elements of A and B, and so forth.

The equality procedure must be consistent with \texttt{eq?}. Note that
this implies that two ideques which are \texttt{eq?} are always
\texttt{ideque=}, as well; implementations may exploit this fact to
``short-cut'' the element-by-element comparisons.

\texttt{(ideque-any\ }\emph{pred ideque}\texttt{)}

\texttt{(ideque-every\ }\emph{pred ideque}\texttt{)}

Invokes \emph{pred} on the elements of the \emph{ideque} in order until
one call returns a true/false value, which is then returned. If there
are no elements, returns \texttt{\#f}/\texttt{\#t}. Takes O(n) time.

\hypertarget{Queueoperations}{\subsection{Queue
operations}\label{Queueoperations}}

\texttt{(ideque-front\ }\emph{ideque}\texttt{)}

\texttt{(ideque-back\ }\emph{ideque}\texttt{)}

Returns the front/back element of \emph{ideque}. It is an error for
\emph{ideque} to be empty. Takes O(1) time.

\texttt{(ideque-remove-front\ }\emph{ideque}\texttt{)}

\texttt{(ideque-remove-back\ }\emph{ideque}\texttt{)}

Returns an ideque with the front/back element of \emph{ideque} removed.
It is an error for \emph{ideque} to be empty. Takes O(1) time.

\texttt{(ideque-add-front\ }\emph{ideque obj}\texttt{)}

\texttt{(ideque-add-back\ }\emph{ideque obj}\texttt{)}

Returns an ideque with \emph{obj} pushed to the front/back of
\emph{ideque}. Takes O(1) time.

\hypertarget{Otheraccessors}{\subsection{Other
accessors}\label{Otheraccessors}}

\texttt{(ideque-ref\ }\emph{ideque n}\texttt{)}

Returns the \emph{nth} element of \emph{ideque}. It is an error unless
\emph{n} is less than the length of \emph{ideque}. Takes O(n) time.

\texttt{(ideque-take\ }\emph{ideque n}\texttt{)}

\texttt{(ideque-take-right\ }\emph{ideque n}\texttt{)}

Returns an ideque containing the first/last \emph{n} elements of
\emph{ideque}. It is an error if \emph{n} is greater than the length of
\emph{ideque}. Takes O(n) time.

\texttt{(ideque-drop\ }\emph{ideque n}\texttt{)}

\texttt{(ideque-drop-right\ }\emph{ideque n}\texttt{)}

Returns an ideque containing all but the first/last \emph{n} elements of
\emph{ideque}. It is an error if \emph{n} is greater than the length of
\emph{ideque}. Takes O(n) time.

\texttt{(ideque-split-at\ }\emph{ideque n}\texttt{)}

Returns two values, the results of \texttt{(ideque-take\ }\emph{ideque
n}\texttt{)} and \texttt{(ideque-drop\ }\emph{ideque n}\texttt{)}
respectively, but may be more efficient. Takes O(n) time.

\hypertarget{Thewholedeque}{\subsection{The whole
ideque}\label{Thewholedeque}}

\texttt{(ideque-length\ }\emph{ideque}\texttt{)}

Returns the length of \emph{ideque} as an exact integer. May take O(n)
time, though O(1) is optimal.

\texttt{(ideque-append\ }\emph{ideque} \ldots{}\texttt{)}

Returns an ideque with the contents of the \emph{ideque} followed by the
others, or an empty ideque if there are none. Takes O(kn) time, where k
is the number of ideques and n is the number of elements involved,
though O(k log n) is possible.

\texttt{(ideque-reverse\ }\emph{ideque}\texttt{)}

Returns an ideque containing the elements of \emph{ideque} in reverse
order. Takes O(1) time.

\texttt{(ideque-count\ }\emph{pred ideque}\texttt{)}

\emph{Pred} is a procedure taking a single value and returning a single
value. It is applied element-wise to the elements of ideque, and a count
is tallied of the number of elements that produce a true value. This
count is returned. Takes O(n) time. The dynamic order of calls to pred
is unspecified.

\texttt{(ideque-zip\ }\emph{ideque\textsubscript{1}
ideque\textsubscript{2}} \ldots{}\texttt{)}

Returns an ideque of lists (not ideques) each of which contains the
corresponding elements of ideques in the order specified. Terminates
when all the elements of any of the ideques have been processed. Takes
O(kn) time, where \emph{k} is the number of ideques and \emph{n} is the
number of elements in the shortest ideque.

\hypertarget{Mapping}{\subsection{Mapping}\label{Mapping}}

\texttt{(ideque-map\ }\emph{proc ideque}\texttt{)}

Applies \emph{proc} to the elements of \emph{ideque} and returns an
ideque containing the results in order. The dynamic order of calls to
\emph{proc} is unspecified. Takes O(n) time.

\texttt{(ideque-filter-map\ }\emph{proc ideque}\texttt{)}

Applies \emph{proc} to the elements of \emph{ideque} and returns an
ideque containing the true (i.e. non-\texttt{\#f}) results in order. The
dynamic order of calls to \emph{proc} is unspecified. Takes O(n) time.

\texttt{(ideque-for-each\ }\emph{proc ideque}\texttt{)}

\texttt{(ideque-for-each-right\ }\emph{proc ideque}\texttt{)}

Applies \emph{proc} to the elements of \emph{ideque} in forward/reverse
order and returns an unspecified result. Takes O(n) time.

\texttt{(ideque-fold\ }\emph{proc nil ideque}\texttt{)}

\texttt{(ideque-fold-right\ }\emph{proc nil ideque}\texttt{)}

Invokes \emph{proc} on the elements of \emph{ideque} in forward/reverse
order, passing the result of the previous invocation as a second
argument. For the first invocation, \emph{nil} is used as the second
argument. Returns the result of the last invocation, or \emph{nil} if
there was no invocation. Takes O(n) time.

\texttt{(ideque-append-map\ }\emph{proc ideque}\texttt{)}

Applies \emph{proc} to the elements of \emph{ideque}. It is an error if
the result is not a list. Returns an ideque containing the elements of
the lists in order. Takes O(n) time, where n is the number of elements
in all the lists returned.

\hypertarget{Filtering}{\subsection{Filtering}\label{Filtering}}

\texttt{(ideque-filter\ }\emph{pred ideque}\texttt{)}

\texttt{(ideque-remove\ }\emph{pred ideque}\texttt{)}

Returns an ideque containing the elements of \emph{ideque} that do/do
not satisfy \emph{pred}. Takes O(n) time.

\texttt{(ideque-partition\ }\emph{proc ideque}\texttt{)}

Returns two values, the results of \texttt{(ideque-filter\ }\emph{pred
ideque}\texttt{)} and \texttt{(ideque-remove\ }\emph{pred
ideque}\texttt{)} respectively, but may be more efficient. Takes O(n)
time.

\hypertarget{Searching}{\subsection{Searching}\label{Searching}}

\texttt{(ideque-find\ }\emph{pred ideque} {[} \emph{failure}
{]}\texttt{)}

\texttt{(ideque-find-right\ }\emph{pred ideque} {[} \emph{failure}
{]}\texttt{)}

Returns the first/last element of \emph{ideque} that satisfies
\emph{pred}. If there is no such element, returns the result of invoking
the thunk \emph{failure}; the default thunk is
\texttt{(lambda\ ()\ \#f)}. Takes O(n) time.

\texttt{(ideque-take-while\ }\emph{pred ideque}\texttt{)}

\texttt{(ideque-take-while-right\ }\emph{pred ideque}\texttt{)}

Returns an ideque containing the longest initial/final prefix of
elements in \emph{ideque} all of which satisfy \emph{pred}. Takes O(n)
time.

\texttt{(ideque-drop-while\ }\emph{pred ideque}\texttt{)}

\texttt{(ideque-drop-while-right\ }\emph{pred ideque}\texttt{)}

Returns an ideque which omits the longest initial/final prefix of
elements in \emph{ideque} all of which satisfy \emph{pred}, but includes
all other elements of \emph{ideque}. Takes O(n) time.

\texttt{(ideque-span\ }\emph{pred ideque}\texttt{)}

\texttt{(ideque-break\ }\emph{pred ideque}\texttt{)}

Returns two values, the initial prefix of the elements of \emph{ideque}
which do/do not satisfy \emph{pred}, and the remaining elements. Takes
O(n) time.

\hypertarget{Conversion}{\subsection{Conversion}\label{Conversion}}

\texttt{(list-\textgreater{}ideque\ }\emph{list}\texttt{)}

\texttt{(ideque-\textgreater{}list\ }\emph{ideque}\texttt{)}

Conversion between ideque and list structures. FIFO order is preserved,
so the front of a list corresponds to the front of an ideque. Each
operation takes O(n) time.

\texttt{(generator-\textgreater{}ideque\ }\emph{generator}\texttt{)}

\texttt{(ideque-\textgreater{}generator\ }\emph{ideque}\texttt{)}

Conversion between
\href{http://srfi.schemers.org/srfi-121/srfi-121.html}{SRFI 121}
generators and ideques. Each operation takes O(n) time. A generator is a
procedure that is called repeatedly with no arguments to generate
consecutive values, and returns an end-of-file object when it has no
more values to return.

\section{Implementation}\label{implementation}

The sample implementation is in the in the \texttt{ideque-2list}
subdirectory of the repository for this SRFI. It is an implementation
based on two Scheme lists, one for the front and another (in reverse
element order) for the back. The cost of \texttt{ideque-front},
\texttt{ideque-back}, \texttt{ideque-remove-front}, and
\texttt{ideque-remove-back} are amortized O(1). An implementation based
on finger trees may be added later.

\section{Copyright}\label{copyright}

Copyright (C) John Cowan, Kevin Wortman (2015). All Rights Reserved.

Permission is hereby granted, free of charge, to any person obtaining a
copy of this software and associated documentation files (the
``Software''), to deal in the Software without restriction, including
without limitation the rights to use, copy, modify, merge, publish,
distribute, sublicense, and/or sell copies of the Software, and to
permit persons to whom the Software is furnished to do so, subject to
the following conditions:

The above copyright notice and this permission notice shall be included
in all copies or substantial portions of the Software.

THE SOFTWARE IS PROVIDED ``AS IS'', WITHOUT WARRANTY OF ANY KIND,
EXPRESS OR IMPLIED, INCLUDING BUT NOT LIMITED TO THE WARRANTIES OF
MERCHANTABILITY, FITNESS FOR A PARTICULAR PURPOSE AND NONINFRINGEMENT.
IN NO EVENT SHALL THE AUTHORS OR COPYRIGHT HOLDERS BE LIABLE FOR ANY
CLAIM, DAMAGES OR OTHER LIABILITY, WHETHER IN AN ACTION OF CONTRACT,
TORT OR OTHERWISE, ARISING FROM, OUT OF OR IN CONNECTION WITH THE
SOFTWARE OR THE USE OR OTHER DEALINGS IN THE SOFTWARE.

\begin{center}\rule{0.5\linewidth}{\linethickness}\end{center}

Editor: \href{mailto:srfi-editors+at+srfi+dot+schemers+dot+org}{Arthur
A. Gleckler}
