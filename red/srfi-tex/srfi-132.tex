\section{Title}\label{title}

Sort Libraries

\section{Author}\label{author}

John Cowan (based on SRFI 32 by Olin Shivers)

\section{Status}\label{status}

This SRFI is currently in \emph{final} status. Here is
\href{http://srfi.schemers.org/srfi-process.html}{an explanation} of
each status that a SRFI can hold. To provide input on this SRFI, please
send email to \texttt{srfi-132@nospamsrfi.schemers.org}. To subscribe to
the list, follow
\href{http://srfi.schemers.org/srfi-list-subscribe.html}{these
instructions}. You can access previous messages via the mailing list
\href{http://srfi-email.schemers.org/srfi-132}{archive}.

\begin{itemize}
\tightlist
\item
  Received: 2015/12/9
\item
  60-day deadline: 2016/2/7
\item
  Draft \#1 published: 2015/12/9
\item
  Draft \#2 published: 2015/12/17
\item
  Draft \#3 published: 2016/1/20
\item
  Draft \#4 published: 2016/1/24 (code changes only)
\item
  Draft \#5 published: 2016/2/1
\item
  Draft \#6 published: 2016/3/13
\item
  Draft \#7 published: 2016/3/21 (code changes only)
\item
  Finalized: 2016/4/20
\end{itemize}

\section{Abstract}\label{abstract}

This SRFI describes the API for a full-featured sort toolkit.

\section{Issues}\label{issues}

None at present.

\section{Rationale}\label{rationale}

\subsection{100 SRFIs later\ldots{}}\label{srfis-later}

This SRFI is based on
\href{http://srfi.schemers.org/srfi-32/srfi-32.txt}{SRFI 32} by Olin
Shivers, which was withdrawn twelve years ago.

There are two backward incompatible changes to the API. The most
important one is that the comparison predicate now precedes the data
rather than following it. Most of the SRFI 32 commenters thought that
this is the better order, despite the then-widespread precedent. The
other change is that the algorithm-specific procedures have been removed
from the SRFI, though not necessarily from the sample implementation.
For documentation on them, see SRFI 32 or the source code.

Editorially, the text has been reorganized, removing the massive
redundancies. As in R5RS and R7RS, procedures that return nothing of
interest, and in SRFI 32 returned zero or more unspecified values, are
now specified to return one otherwise unspecified value. The multiple
packages, most with only a few procedures, into which SRFI 32 is divided
have been consolidated. References to
\href{http://srfi.schemers.org/srfi-95/srfi-95.html}{SRFI 95} and
\href{http://www.r6rs.org/final/html/r6rs-lib/r6rs-lib-Z-H-5.html\#node_chap_4}{R6RS}
equivalents have been added.

\subsection{What vs. how}\label{Whatvs.how}

In SRFI 32 there were two different interfaces: ``what'' (simple) and
``how'' (detailed).

\begin{itemize}
\tightlist
\item
  Simple: you specified semantics: datatype (list or vector),
  mutability, and stability.
\end{itemize}

\begin{itemize}
\tightlist
\item
  Detailed: you specified the datatype and the actual algorithm (quick,
  heap, insert, merge).
\end{itemize}

However, the difficulty with exposing specific algorithms by name is
that the local optima in the search space of algorithms changes over
time. For example, the ``qsort'' in the musl C library is actually
smooth sort, not quick sort; Python and Java have switched from quick
sort to timsort; some implementations of the C++ STL use introsort
rather than quick sort. Having procedure names that imply e.g. quick
sort, that do not actually implement quick sort, is confusing.

Therefore, only the ``what'' interface has been retained. The sample
implementation uses some of the original SRFI 32 algorithms, but this is
not a requirement of this SRFI.

\subsection{Consistency across procedure
signatures}\label{Consistencyacrossproceduresignatures}

Procedures share common signatures wherever possible, to facilitate
switching a given call from one procedure to another.

\subsection{Ordering parameter before data
parameter}\label{Orderingparameterbeforedataparameter}

These procedures uniformly observe the following parameter order: the
ordering, equality, or comparison argument precedes the argument(s)
denoting the data to be sorted. In SRFI 32, the data-first convention
was used, because it was consistent with all the implementations
examined in 2002, with the sole exception of Chez Scheme. However,
\href{http://www.r6rs.org/final/html/r6rs-lib/r6rs-lib-Z-H-5.html\#node_chap_4}{R6RS}
adopted the procedure-first convention, which is more consistent with
other Scheme libraries that put the procedure first --- the ``operation
currying'' convention, as in \texttt{for-each} or \texttt{map} or
\texttt{find}. This SRFI uses the R6RS convention throughout.

\subsection{Stability}\label{Stability}

A ``stable'' sort is one that preserves the pre-existing order of equal
elements. Suppose, for example, that we sort a list of numbers by
comparing their absolute values using \texttt{abs\textless{}} (defined
below). If we sort a list that contains both 3 and -3, then a stable
sort is an algorithm that will not swap the order of these two elements,
that is, the answer will look like \texttt{...\ 3\ -3\ ...}, not
\texttt{...\ -3\ 3\ ...}.

A ``stable'' merge is one that reliably favors one of its data sets when
equal items appear in both data sets. All merge operations specified in
this SRFI are stable, breaking ties between data sets in favor of the
first data set --- elements of the first list come before equal elements
in the second list.

So, if we are merging two lists of numbers ordered by absolute value
using the stable merge operation \texttt{list-merge} and
\texttt{abs\textless{}} compares numbers by their absolute values, then

\begin{verbatim}
(list-merge abs< '(0 -2 4 8 -10) '(-1 3 -4 7))
\end{verbatim}

reliably places the 4 of the first list before the equal-comparing -4 of
the second list: \texttt{(0\ -1\ -2\ 4\ -4\ 7\ 8\ -10)}.

Here's a definition of \texttt{abs\textless{}}, introduced just for the
examples as it is not part of this SRFI:

\begin{verbatim}
(define (abs< x y) (< (abs x) (abs y))))
\end{verbatim}

\subsection{All vector operations accept optional subrange
parameters}\label{Allvectoroperationsacceptoptionalsubrangeparameters}

The vector operations specified below all take optional \emph{start} and
\emph{end} arguments indicating a selected subrange of a vector's
elements. This compensates for the absence in most Schemes of shared
subvectors.

\subsection{Required vs. allowed
side-effects}\label{Requiredvs.allowedside-effects}

The \texttt{list-sort!} and \texttt{list-stable-sort!} procedures are
allowed, but not required, to alter their arguments' cons cells to
construct the result list.

The other procedures ending in \texttt{!}, on the other hand, explicitly
commit to the use of side-effects on their input lists in order to
guarantee their key algorithmic properties (e.g., linear-time operation,
constant-space stack use).

\subsection{Sorting lists}\label{Sortinglists}

Note that sorting lists involves chasing pointers through memory, which
can be a loser on modern machine architectures because of poor cache and
page locality. Pointer \emph{writing}, which is what the
\texttt{set-cdr!}s of a destructive list-sort algorithm do, is even
worse, especially if your Scheme has a generational GC --- the writes
will thrash the write-barrier. Sorting vectors has inherently better
locality.

In particular, all complexity guarantees assume that the basic accessors
and mutators of standard Scheme have O(1) space and time complexity.

The reference implementation's destructive list merge and merge sort
implementations are opportunistic --- they avoid redundant
\texttt{set-cdr!}s, and try to take long already-ordered runs of list
structure as-is when doing the merges.

\section{Specification}\label{specification}

\subsection{Procedure naming and
functionality}\label{Procedurenamingandfunctionality}

Most of the procedures described below are variants of two basic
operations: sorting and merging. These procedures are consistently named
by composing a set of basic lexemes to indicate what they do.

\begin{longtable}[]{@{}ll@{}}
\toprule
Lexeme & Meaning\tabularnewline
\texttt{vector} & The procedure operates upon vectors.\tabularnewline
\texttt{list} & The procedure operates upon lists.\tabularnewline
\texttt{stable} & This lexeme indicates that the sort is a stable
one.\tabularnewline
\texttt{sort} & The procedure sorts its input data set by some ordering
function.\tabularnewline
\texttt{merge} & The procedure merges two ordered data sets into a
single ordered result.\tabularnewline
\texttt{!} & Procedures that end in \texttt{!} are allowed, and
sometimes required, to reuse their input storage to construct their
answer.\tabularnewline
\bottomrule
\end{longtable}

\subsection{Types of parameters and return
values}\label{Typesofparametersandreturnvalues}

In the procedures specified below:

\begin{itemize}
\tightlist
\item
  A \emph{lis} parameter is a list.
\end{itemize}

\begin{itemize}
\tightlist
\item
  A \emph{v} parameter is a vector.
\end{itemize}

\begin{itemize}
\tightlist
\item
  An \emph{=} parameter is an equality predicate. See
  \href{http://srfi.schemers.org/srfi-128/srfi-128.html}{SRFI 128} for
  the requirements on equality predicates. Note that neither this SRFI
  nor its sample implementation depend on SRFI 128.
\end{itemize}

\begin{itemize}
\tightlist
\item
  A \emph{\textless{}} parameter is an ordering predicate. See
  \href{http://srfi.schemers.org/srfi-128/srfi-128.html}{SRFI 128} for
  the requirements on ordering predicates.
\end{itemize}

\begin{itemize}
\tightlist
\item
  A \emph{start} parameter or \emph{start} and \emph{end} parameter pair
  are exact non-negative integers such that 0 \textless{}= \emph{start}
  \textless{}= \emph{end} \textless{}=
  \texttt{(vector-length\ }\emph{v}\texttt{)}, where \emph{v} is the
  related vector parameter. If not specified, they default to 0 and the
  length of the vector, respectively. They are interpreted to select the
  range {[}\emph{start}, \emph{end}), that is, all elements from index
  \emph{start} (inclusive) up to, but not including, index \emph{end}.
\end{itemize}

Passing values to procedures with these parameters that do not satisfy
these constraints is an error.

If a procedure is said to return ``an unspecified value'', this means
that nothing at all is said about what the procedure returns, except
that it returns one value.

\subsection{Predicates}\label{Predicates}

\texttt{(list-sorted?\ }\emph{\textless{} lis}\texttt{)}

\texttt{(vector-sorted?\ }\emph{\textless{} v} {[}\emph{start} {[}
\emph{end} {]} {]}

These procedures return true iff their input list or vector is in sorted
order, as determined by \emph{\textless{}}. Specifically, they return
\texttt{\#f} iff there is an adjacent pair \ldots{} X Y \ldots{} in the
input list or vector such that Y \textless{} X in the sense of
\emph{\textless{}}. The optional \emph{start} and \emph{end} range
arguments restrict \texttt{vector-sorted?} to examining the indicated
subvector.

These procedures are equivalent to the SRFI 95 \texttt{sorted?}
procedure when applied to lists or vectors respectively, except that
they do not accept a key procedure.

\subsection{General sort procedures}\label{Generalsortprocedures}

These procedures provide basic sorting and merging functionality
suitable for general programming. The procedures are named by their
semantic properties, i.e., what they do to the data (sort, stable sort,
and so forth).

\texttt{(list-sort\ }\emph{\textless{} lis}\texttt{)}

\texttt{(list-stable-sort\ }\emph{\textless{} lis}\texttt{)}

These procedures do not alter their inputs, but are allowed to return a
value that shares a common tail with a list argument.

The \texttt{list-stable-sort} procedure is equivalent to the R6RS
\texttt{list-sort} procedure. It is also equivalent to the SRFI 95
\texttt{sort} procedure when applied to lists, except that it does not
accept a key procedure.

\texttt{(list-sort!\ }\emph{\textless{} lis}\texttt{)}

\texttt{(list-stable-sort!\ }\emph{\textless{} lis}\texttt{)}

These procedures are linear update operators --- they are allowed, but
not required, to alter the cons cells of their arguments to produce
their results. They return a sorted list containing the same elements as
\emph{lis}.

The \texttt{list-stable-sort!} procedure is equivalent to the SRFI 95
\texttt{sort!} procedure when applied to lists, except that it does not
accept a key procedure.

\texttt{(vector-sort\ }\emph{\textless{} v} {[} \emph{start} {[}
\emph{end} {]} {]}\texttt{)}

\texttt{(vector-stable-sort\ }\emph{\textless{} v} {[} \emph{start} {[}
\emph{end} {]} {]}\texttt{)}

These procedures do not alter their inputs, but allocate a fresh vector
as their result, of length \emph{end} - \emph{start}. The
\texttt{vector-stable-sort} procedure with no optional arguments is
equivalent to the R6RS \texttt{vector-sort} procedure. It is also
equivalent to the SRFI 95 \texttt{sort} procedure when applied to
vectors, except that it does not accept a key procedure.

\texttt{(vector-sort!\ }\emph{\textless{} v} {[} \emph{start} {[}
\emph{end} {]} {]}\texttt{)}

\texttt{(vector-stable-sort!\ }\emph{\textless{} v} {[} \emph{start} {[}
\emph{end} {]} {]}\texttt{)}

These procedures sort their data in-place. (But note that
\texttt{vector-stable-sort!} may allocate temporary storage proportional
to the size of the input --- there are no known O(n lg n) stable vector
sorting algorithms that run in constant space.) They return an
unspecified value.

The \texttt{vector-sort!} procedure with no optional arguments is
equivalent to the R6RS \texttt{vector-sort!} procedure.

\subsection{Merge procedures}\label{Mergeprocedures}

All four merge operations are stable: an element of the initial list
\emph{lis\textsubscript{1}} or vector \emph{v\textsubscript{1}} will
come before an equal-comparing element in the second list
\emph{lis\textsubscript{2}} or vector \emph{v\textsubscript{2}} in the
result.

\texttt{(list-merge\ }\emph{\textless{} lis\textsubscript{1}
lis\textsubscript{2}}\texttt{)}

This procedure does not alter its inputs, and is allowed to return a
value that shares a common tail with a list argument.

This procedure is equivalent to the SRFI 95 \texttt{merge} procedure
when applied to lists, except that it does not accept a key procedure.

\texttt{(list-merge!\ }\emph{\textless{} lis\textsubscript{1}
lis\textsubscript{2}}\texttt{)}

This procedure makes only a single, iterative, linear-time pass over its
argument lists, using \texttt{set-cdr!}s to rearrange the cells of the
lists into the list that is returned --- it works ``in place.'' Hence,
any cons cell appearing in the result must have originally appeared in
an input. It returns the sorted input.

Additionally, \texttt{list-merge!} is iterative, not recursive --- it
can operate on arguments of arbitrary size without requiring an
unbounded amount of stack space. The intent of this iterative-algorithm
commitment is to allow the programmer to be sure that if, for example,
\texttt{list-merge!} is asked to merge two ten-million-element lists,
the operation will complete without performing some extremely (possibly
twenty-million) deep recursion.

This procedure is equivalent to the SRFI 95 \texttt{merge!} procedure
when applied to lists, except that it does not accept a key procedure.

\texttt{(vector-merge} \emph{\textless{} v\textsubscript{1}
v\textsubscript{2}} {[} \emph{start\textsubscript{1}} {[}
\emph{end\textsubscript{1}} {[} \emph{start\textsubscript{2}} {[}
\emph{end\textsubscript{2}} {]} {]} {]} {]}\texttt{)}

This procedure does not alter its inputs, and returns a newly allocated
vector of length (\emph{end\textsubscript{1}} -
\emph{start\textsubscript{1}}) + (\emph{end\textsubscript{2}} -
\emph{start\textsubscript{2}}\texttt{)}.

This procedure is equivalent to the SRFI 95 \texttt{merge} procedure
when applied to vectors, except that it does not accept a key procedure.

\texttt{(vector-merge!\ }\emph{\textless{} to from\textsubscript{1}
from\textsubscript{2}} {[} \emph{start} {[}
\emph{start\textsubscript{1}} {[} \emph{end\textsubscript{1}} {[}
\emph{start\textsubscript{2}} {[} \emph{end\textsubscript{2}} {]} {]}
{]} {]} {]}\texttt{)}

This procedure writes its result into vector \emph{to}, beginning at
index \emph{start}, for indices less than \emph{end}, which is defined
as \emph{start} + (\emph{end\textsubscript{1}} -
\emph{start\textsubscript{1}}) + (\emph{end\textsubscript{2}} -
\emph{start\textsubscript{2}}\texttt{)}. The target subvector
\emph{to}{[}\emph{start}, \emph{end}) may not overlap either of the
source subvectors
\emph{from\textsubscript{1}}{[}\emph{start\textsubscript{1}},
\emph{end\textsubscript{1}}{]} and
\emph{from\textsubscript{2}}{[}\emph{start\textsubscript{2}},
\emph{end\textsubscript{2}}{]}. It returns an unspecified value.

This procedure is equivalent to the SRFI 95 \texttt{merge!} procedure
when applied to lists, except that it does not accept a key procedure.

\subsection{Deleting duplicate
neighbors}\label{Deletingduplicateneighbors}

These procedures delete adjacent duplicate elements from a list or a
vector, using a given element-equality procedure. The first/leftmost
element of a run of equal elements is the one that survives. The list or
vector is not otherwise disordered.

These procedures are linear time --- much faster than the
O(n\textsuperscript{2}) general duplicate-element deletion procedures
that do not assume any ``bunching'' of elements provided by
\href{http://srfi.schemers.org/srfi-1/srfi-1.html}{SRFI 1}. If you want
to delete duplicate elements from a large list or vector, sort the
elements to bring equal items together, then use one of these
procedures, for a total time of O(n lg n).

The equality procedure is always invoked as \texttt{(=\ x\ y)}, where
\emph{x} comes before \emph{y} in the containing list or vector.

\texttt{(list-delete-neighbor-dups\ }\emph{= lis}\texttt{)}

This procedure does not alter its input list, but its result may share
storage with the input list.

\texttt{(list-delete-neighbor-dups!\ }\emph{= lis}\texttt{)}

This procedure mutates its input list in order to construct its result.
It makes only a single, iterative, linear-time pass over its argument,
using \texttt{set-cdr!}s to rearrange the cells of the list into the
final result --- it works ``in place.'' Hence, any cons cell appearing
in the result must have originally appeared in the input.

\texttt{(vector-delete-neighbor-dups\ }\emph{= v} {[} \emph{start} {[}
\emph{end} {]} {]}\texttt{)}

This procedure does not alter its input vector, but rather newly
allocates and returns a vector to hold the result.

\texttt{(vector-delete-neighbor-dups!\ }\emph{= v} {[} \emph{start} {[}
\emph{end} {]} {]}\texttt{)}

This procedure reuses its input vector to hold the answer, packing it
into the index range {[}\emph{start, newend}), where \emph{newend} is
the non-negative exact integer that is returned as its value. The vector
is not altered outside the range {[}\emph{start, newend}).

Examples:

\begin{verbatim}
  (list-delete-neighbor-dups = '(1 1 2 7 7 7 0 -2 -2))
               => (1 2 7 0 -2)

    (vector-delete-neighbor-dups = '#(1 1 2 7 7 7 0 -2 -2))
               => #(1 2 7 0 -2)

    (vector-delete-neighbor-dups < '#(1 1 2 7 7 7 0 -2 -2) 3 7))
               => #(7 0 -2)

;; Result left in v[3,9):
(let ((v (vector 0 0 0 1 1 2 2 3 3 4 4 5 5 6 6)))
  (cons (vector-delete-neighbor-dups! < v 3)
        v))
              => (9 . #(0 0 0 1 2 3 4 5 6 4 4 5 5 6 6))
\end{verbatim}

\subsection{Finding the median}\label{Findingthemedian}

These procedures do not have SRFI 32 counterparts. They find the median
element of a vector after sorting it in accordance with an ordering
procedure. If the number of elements in \texttt{v} is odd, the
middlemost element of the sorted result is returned. If the number of
elements is zero, \texttt{knil} is returned. Otherwise, \texttt{mean} is
applied to the two middlemost elements in the order in which they appear
in \texttt{v}, and whatever it returns is returned. If \texttt{mean} is
omitted, then the default mean procedure is
\texttt{(lambda\ (a\ b)\ (/\ (+\ a\ b)\ 2)}, but this procedure is
applicable to non-numeric values as well.

\texttt{(vector-find-median\ }\emph{\textless{} v knil} {[} \emph{mean}
{]}\texttt{)}

This procedure does not alter its input vector, but rather newly
allocates a vector to hold the intermediate result. Runs in O(n) time.

\texttt{(vector-find-median!\ }\emph{\textless{} v knil} {[} \emph{mean}
{]}\texttt{)}

This procedure reuses its input vector to hold the intermediate result,
leaving it sorted, but is otherwise the same as
\texttt{vector-find-median}. Runs in O(n ln n) time.

\subsection{Selection}\label{Selection}

These procedures do not have SRFI 32 counterparts.

\texttt{(vector-select!\ }\emph{\textless{} v k} {[} \emph{start} {[}
\emph{end} {]} {]} \texttt{)}

This procedure returns the \emph{k}th smallest element (in the sense of
the \textless{} argument) of the region of a vector between \emph{start}
and \emph{end}. Elements within the range may be reordered, whereas
those outside the range are left alone. Runs in O(n) time.

\texttt{(vector-separate!\ }\emph{\textless{} v k} {[} \emph{start} {[}
\emph{end} {]} {]} \texttt{)}

This procedure places the smallest \emph{k} elements (in the sense of
the \textless{} argument) of the region of a vector between \emph{start}
and \emph{end} into the first \emph{k} positions of that range, and the
remaining elements into the remaining positions. Otherwise, the elements
are not in any particular order. Elements outside the range are left
alone. Runs in O(n) time. Returns an unspecified value.

\section{Implementation}\label{implementation}

The sample implementation is a modified version of the Scheme 48
implementation of the \texttt{sorting} structure, and is found in the
repository of this SRFI. It will use the R6RS sorting library if it is
available, but does not depend on it. This is close to the original SRFI
32 reference implementation, but includes some bug fixes and switches
the \textless{} and = arguments to the initial position. It also adds
implementations for the median and selection procedures. The code is
very portable and freely reusable. It is tightly bummed, as far as could
be done in portable Scheme, and is commented in Olin's usual voluminous
style, including notes on porting and implementation-specific
optimizations. The median and selection code is specific to this SRFI.

The only non-R4RS features in the code are the use of R5RS/R6RS/R7RS
multiple-value return, with \texttt{values} and
\texttt{call-with-values} procedures, and the use of R7RS-style
\texttt{error} to report an assertion violation.

You could speed up the vector code a lot by error-checking the procedure
parameters and then shifting over to fixnum-specific arithmetic and
dangerous vector-indexing and vector-setting primitives. The comments in
the code indicate where the initial error checks would have to be added.
There are several \texttt{(quotient\ }\emph{n}\texttt{\ 2)} calls that
could be changed to a fixnum right-shift, as well, in both the list and
vector code. The code is designed to enable this --- each file usually
exports one or two ``safe'' procedures that end up calling an internal
``dangerous'' primitive. The little exported cover procedures are where
you move the error checks.

This should provide \emph{big} speedups. In fact, all the code bumming
in the source pretty much disappears in the noise unless you have a good
compiler and also can dump the vector-index checks and generic
arithmetic --- so it's really set up for optimization rather than fully
optimized.

The optional-arg parsing, defaulting, and error checking is done with a
portable \texttt{syntax-rules} macro. But if the target Scheme has a
faster mechanism (e.g., Chez), it's definitely better to switch to using
it. Note that argument defaulting and error-checking are interleaved ---
there's no need to error-check defaulted \emph{start} and \emph{end}
args to see if they are fixnums that are legal vector indices for the
corresponding vector, etc.

\subsection{Files}\label{Files}

\texttt{delndups.scm} - the \texttt{delete-neighbor-dups} procedures\\
\texttt{lmsort.scm} - list merge sort\\
\texttt{median.scm} - the \texttt{find-median} procedures\\
\texttt{selection.scm} - the \texttt{selection} procedure\\
\texttt{sort.scm} - generic sort and merge procedures\\
\texttt{sorting-test.scm} - test file\\
\texttt{sortp.scm} - sort predicates\\
\texttt{srfi-132.scm} - a Chicken library providing this SRFI\\
\texttt{srfi-132.sld} - an R7RS counterpart of \texttt{srfi-132.scm}\\
\texttt{vector-util.scm} - vector utilities\\
\texttt{vhsort.scm} - vector heap sort\\
\texttt{visort.scm} - vector insert sort\\
\texttt{vmsort.scm} - vector merge sort\\
\texttt{vqsort2.scm} - vector quick sort\\

\section{Acknowledgements}\label{acknowledgements}

Olin thanked the authors of the open source consulted when designing
this library, particularly Richard O'Keefe, Donovan Kolbly and the MIT
Scheme Team. John thanks Will Clinger for his detailed comments, and
both Will Clinger and Alex Shinn for their implementation efforts.

\section{Copyright}\label{copyright}

\subsection{SRFI text copyright}\label{SRFItextcopyright}

This document is copyright (C) Olin Shivers (1998, 1999). All Rights
Reserved.

Permission is hereby granted, free of charge, to any person obtaining a
copy of this software and associated documentation files (the
``Software''), to deal in the Software without restriction, including
without limitation the rights to use, copy, modify, merge, publish,
distribute, sublicense, and/or sell copies of the Software, and to
permit persons to whom the Software is furnished to do so, subject to
the following conditions:

The above copyright notice and this permission notice shall be included
in all copies or substantial portions of the Software.

THE SOFTWARE IS PROVIDED ``AS IS'', WITHOUT WARRANTY OF ANY KIND,
EXPRESS OR IMPLIED, INCLUDING BUT NOT LIMITED TO THE WARRANTIES OF
MERCHANTABILITY, FITNESS FOR A PARTICULAR PURPOSE AND NONINFRINGEMENT.
IN NO EVENT SHALL THE AUTHORS OR COPYRIGHT HOLDERS BE LIABLE FOR ANY
CLAIM, DAMAGES OR OTHER LIABILITY, WHETHER IN AN ACTION OF CONTRACT,
TORT OR OTHERWISE, ARISING FROM, OUT OF OR IN CONNECTION WITH THE
SOFTWARE OR THE USE OR OTHER DEALINGS IN THE SOFTWARE.

\subsection{Sample implementation
copyright}\label{Sampleimplementationcopyright}

Short summary: no restrictions.

While Olin wrote all of this code himself, he read a lot of code before
he began writing. However, all such code is, itself, either open source
or public domain, rendering irrelevant any issue of ``copyright taint.''

Hence the sample implementation is Copyright © 1998 by Olin Shivers and
made available under the same copyright as the SRFI text (see above).

\begin{center}\rule{0.5\linewidth}{\linethickness}\end{center}

Editor:
\href{mailto:srfi-editors\%20at\%20srfi\%20dot\%20schemers\%20dot\%20org}{Arthur
A. Gleckler}
