\section{Title}\label{title}

Generators

\section{Author}\label{author}

Shiro Kawai, John Cowan, Thomas Gilray

This SRFI is currently in ``final'' status. To see an explanation of
each status that a SRFI can hold, see
\href{http://srfi.schemers.org/srfi-process.html}{here}. To provide
input on this SRFI, please
\href{mailto:srfi\%20minus\%20121\%20at\%20srfi\%20dot\%20schemers\%20dot\%20org}{mail
to
\texttt{\textless{}srfi\ minus\ 121\ at\ srfi\ dot\ schemers\ dot\ org\textgreater{}}}.
See
\href{http://srfi.schemers.org/srfi-list-subscribe.html}{instructions
here} to subscribe to the list. You can access previous messages via
\href{http://srfi.schemers.org/srfi-121/mail-archive/maillist.html}{the
archive of the mailing list}.

\begin{itemize}
\tightlist
\item
  Received:
  \href{http://srfi.schemers.org/srfi-121/srfi-121-1.1.html}{2015/01/27}
\item
  Draft \#2 published:
  \href{http://srfi.schemers.org/srfi-121/srfi-121-1.2.html}{2015/06/10}
\item
  Draft \#3 published: 2015/09/03
\item
  Draft \#4 published: 2015/10/14
\item
  Draft \#5 published: 2015/10/19
\item
  Draft \#6 published: 2015/10/25
\item
  Draft \#7 published: 2015/11/2
\item
  Draft \#8 published: 2015/11/8
\item
  Draft \#9 published: 2015/12/25
\item
  Draft \#10 published: 2016/1/4
\item
  Draft \#11 published: 2016/1/18
\item
  Finalized: 2016/1/18
\item
  Revised to fix errata: 2016/1/20
\end{itemize}

\section{Abstract}\label{abstract}

This SRFI defines utility procedures that create, transform, and consume
generators. A generator is simply a procedure with no arguments that
works as a source of a series of values. Every time it is called, it
yields a value. Generators may be finite or infinite; a finite generator
returns an end-of-file object to indicate that it is exhausted. For
example, \texttt{read-char}, \texttt{read-line}, and \texttt{read} are
generators that generate characters, lines, and objects from the current
input port. Generators provide lightweight laziness.

\section{Issues}\label{issues}

None at present.

\section{Rationale}\label{rationale}

The main purpose of generators is high performance. Although
\href{http://srfi.schemers.org/srfi-41/srfi-41.html}{SRFI 41} streams
can do everything generators can do and more, SRFI 41 uses lazy pairs
that require making a thunk for every item. Generators can generate
items without consing, so they are very lightweight and are useful for
implementing simple on-demand calculations.

Existing examples of generators are readers from the current input port
and \href{http://srfi.schemers.org/srfi-27/srfi-27.html}{SRFI 27} random
numbers. If Scheme had streams as one of its built-in abstractions,
these would have been naturally represented by lazy streams. But Scheme
usually does not expose this kind of API using lazy streams.
Generator-like interfaces are common, so it seems worthwhile to have
some common idioms extracted into a library.

Calling a generator is a side-effecting construct; you can't safely
backtrack, for example, as you can with streams. Persistent lazy
sequences based on generators and ordinary Scheme pairs (which are
heavier weight than generators, but lighter weight than lazy pairs) is
the subject of SRFI 127. Of course the efficiency of streams depends on
the implementation. Some implementations may have have super-light thunk
creation. But in most, thunk creation is probably slower than simple
consing.

The generators of this SRFI don't belong to a disjoint type. They are
just procedures that conform to a calling convention, so you can
construct a generator with \texttt{lambda}. The constructors of this
SRFI are provided for convenience. Any procedure that can be called with
no arguments can serve as a generator.

Using an end-of-file object to indicate that there is no more input
makes it impossible to include such an object in the stream of generated
values. However, it is compatible with the existing design of input
ports, and it makes for more compact code than returning a
user-specified termination object (as in Common Lisp) or returning
multiple values. (Note that some generators are infinite in length, and
never return an end-of-file object.)

The combination of \texttt{make-for-each-generator} and
\texttt{generator-unfold} makes it possible to convert any collection
that has a for-each procedure into any collection that has an unfold
constructor. This generalizes such conversion procedures as
\texttt{\&list-\textgreater{}vector} and
\texttt{string-\textgreater{}list}.

These procedures are drawn from the Gauche core and the Gauche module
\texttt{gauche.generator}, with some renaming to make them more
systematic, and with a few additions from the Python library
\href{https://docs.python.org/3/library/itertools.html}{\texttt{itertools}}.
Consequently, Shiro Kawai, the author of Gauche and its specifications,
is listed as first author of this SRFI. John Cowan served as editor and
shepherd. Thomas Gilray provided the sample implementation and a
valuable critique of the SRFI. Special acknowledgements to Kragen Javier
Sitaker for his extensive review.

\section{Specification}\label{specification}

Generators can be divided into two classes, finite and infinite. Both
kinds of generators can be invoked an indefinite number of times. After
a finite generator has generated all its values, it will return an
end-of-file object for all subsequent calls. A generator is said to be
\emph{exhausted} if calling it will return an end-of-file object. By
definition, infinite generators can never be exhausted.

A generator is said to be in an \emph{undefined state} if it cannot be
determined exactly how many values it has generated. This arises because
it is impossible to tell by inspecting a generator whether it is
exhausted or not. For example,
\texttt{(generator-fold\ +\ 0\ (generator\ 1\ 2\ 3)\ (generator\ 1\ 2))}
will compute 0 + 1 + 1 + 2 + 2 = 6, at which time the second generator
will be exhausted. If the first generator is invoked, however, it may
return either 3 or an end-of-file object, depending on whether the
implementation of \texttt{generator-fold} has invoked it or not.
Therefore, the first generator is said to be in an undefined state.

\subsection{Generator constructors}\label{generator-constructors}

The result of a generator constructor is just a procedure, so printing
it doesn't show much. In the examples in this section we use
\texttt{generator-\textgreater{}list} to convert the generator to a
list.

These procedures have names ending with \texttt{generator}.

\begin{description}
\item[\texttt{generator} \emph{arg \ldots{}}]
The simplest finite generator. Generates each of its arguments in turn.
When no arguments are provided, it returns an empty generator that
generates no values.
\end{description}

\begin{description}
\item[\texttt{make-iota-generator} \emph{count {[} start {[} step {]}
{]}}]
Creates a finite generator of a sequence of count numbers. The sequence
begins with start (which defaults to 0) and increases by step (which
defaults to 1). If both start and step are exact, it generates exact
numbers; otherwise it generates inexact numbers. The exactness of count
doesn't affect the exactness of the results.

\begin{longtable}[]{@{}ll@{}}
\toprule
\begin{minipage}[t]{0.47\columnwidth}\raggedright\strut
~\strut
\end{minipage} & \begin{minipage}[t]{0.47\columnwidth}\raggedright\strut
\begin{verbatim}
(generator->list (make-iota-generator 3 8))
  ⇒ (8 9 10)
\end{verbatim}
\strut
\end{minipage}\tabularnewline
\begin{minipage}[t]{0.47\columnwidth}\raggedright\strut
~\strut
\end{minipage} & \begin{minipage}[t]{0.47\columnwidth}\raggedright\strut
\begin{verbatim}
(generator->list (make-iota-generator 3 8 2))
  ⇒ (8 10 12)
\end{verbatim}
\strut
\end{minipage}\tabularnewline
\bottomrule
\end{longtable}
\end{description}

\begin{description}
\item[\texttt{make-range-generator} \emph{start {[} end {[} step {]}
{]}}]
Creates a generator of a sequence of numbers. The sequence begins with
start, increases by step (default 1), and continues while the number is
less than end, or forever if end is omitted. If both start and step are
exact, it generates exact numbers; otherwise it generates inexact
numbers. The exactness of end doesn't affect the exactness of the
results.

\begin{longtable}[]{@{}ll@{}}
\toprule
\begin{minipage}[t]{0.47\columnwidth}\raggedright\strut
~\strut
\end{minipage} & \begin{minipage}[t]{0.47\columnwidth}\raggedright\strut
\begin{verbatim}
(generator->list (make-range-generator 3) 4)
  ⇒ (3 4 5 6)
\end{verbatim}
\strut
\end{minipage}\tabularnewline
\begin{minipage}[t]{0.47\columnwidth}\raggedright\strut
~\strut
\end{minipage} & \begin{minipage}[t]{0.47\columnwidth}\raggedright\strut
\begin{verbatim}
(generator->list (make-range-generator 3 8))
  ⇒ (3 4 5 6 7)
\end{verbatim}
\strut
\end{minipage}\tabularnewline
\begin{minipage}[t]{0.47\columnwidth}\raggedright\strut
~\strut
\end{minipage} & \begin{minipage}[t]{0.47\columnwidth}\raggedright\strut
\begin{verbatim}
(generator->list (make-range-generator 3 8 2))
  ⇒ (3 5 7)
\end{verbatim}
\strut
\end{minipage}\tabularnewline
\bottomrule
\end{longtable}
\end{description}

\begin{description}
\item[\texttt{make-coroutine-generator} \emph{proc}]
Creates a generator from a coroutine.

The proc argument is a procedure that takes one argument, yield. When
called, \texttt{make-coroutine-generator} immediately returns a
generator g. When g is called, proc runs until it calls yield. Calling
yield causes the execution of proc to be suspended, and g returns the
value passed to yield.

Whether this generator is finite or infinite depends on the behavior of
proc. If proc returns, it is the end of the sequence --- g returns an
end-of-file object from then on. The return value of proc is ignored.

The following code creates a generator that produces a series 0, 1, and
2 (effectively the same as \texttt{(make-range-generator\ 0\ 3)} and
binds it to \texttt{g}.

\begin{longtable}[]{@{}ll@{}}
\toprule
\begin{minipage}[t]{0.47\columnwidth}\raggedright\strut
~\strut
\end{minipage} & \begin{minipage}[t]{0.47\columnwidth}\raggedright\strut
\begin{verbatim}
(define g
  (make-coroutine-generator
   (lambda (yield) (let loop ((i 0))
               (when (< i 3) (yield i) (loop (+ i 1)))))))

(generator->list g) ⇒ (0 1 2)
\end{verbatim}
\strut
\end{minipage}\tabularnewline
\bottomrule
\end{longtable}
\end{description}

\begin{description}
\item[\texttt{list-\textgreater{}generator} \emph{lis}\\
\texttt{vector-\textgreater{}generator} \emph{vec {[} start {[} end {]}
{]}}\\
\texttt{reverse-vector-\textgreater{}generator} \emph{vec {[} start {[}
end {]} {]}}\\
\texttt{string-\textgreater{}generator} \emph{str {[} start {[} end {]}
{]}}\\
\texttt{bytevector-\textgreater{}generator} \emph{bytevector {[} start
{[} end {]} {]}}]
These procedures return generators that yield each element of the given
argument. Mutating the underlying object will affect the results of the
generator.

\begin{longtable}[]{@{}ll@{}}
\toprule
\begin{minipage}[t]{0.47\columnwidth}\raggedright\strut
~\strut
\end{minipage} & \begin{minipage}[t]{0.47\columnwidth}\raggedright\strut
\begin{verbatim}
(generator->list (list->generator '(1 2 3 4 5)))
  ⇒ (1 2 3 4 5)
(generator->list (vector->generator '#(1 2 3 4 5)))
  ⇒ (1 2 3 4 5)
(generator->list (reverse-vector->generator '#(1 2 3 4 5)))
  ⇒ (5 4 3 2 1)
(generator->list (string->generator "abcde"))
  ⇒ (#\a #\b #\c #\d #\e)
\end{verbatim}
\strut
\end{minipage}\tabularnewline
\bottomrule
\end{longtable}

The generators returned by the constructors are exhausted once all
elements are retrieved; the optional start-th and end-th arguments can
limit the range the generator walks across.

For \texttt{reverse-vector-\textgreater{}generator}, the first value is
the element right before the end-th element, and the last value is the
start-th element. For all the other constructors, the first value the
generator yields is the start-th element, and it ends right before the
end-th element.

\begin{longtable}[]{@{}ll@{}}
\toprule
\begin{minipage}[t]{0.47\columnwidth}\raggedright\strut
~\strut
\end{minipage} & \begin{minipage}[t]{0.47\columnwidth}\raggedright\strut
\begin{verbatim}
(generator->list (vector->generator '#(a b c d e) 2))
  ⇒ (c d e)
(generator->list (vector->generator '#(a b c d e) 2 4))
  ⇒ (c d)
(generator->list (reverse-vector->generator '#(a b c d e) 2))
  ⇒ (e d c)
(generator->list (reverse-vector->generator '#(a b c d e) 2 4))
  ⇒ (d c)
(generator->list (reverse-vector->generator '#(a b c d e) 0 2))
  ⇒ (b a)
\end{verbatim}
\strut
\end{minipage}\tabularnewline
\bottomrule
\end{longtable}
\end{description}

\begin{description}
\item[\texttt{make-for-each-generator} \emph{for-each obj}]
A generator constructor that converts any collection obj to a generator
that returns its elements using a for-each procedure appropriate for
obj. This must be a procedure that when called as (for-each proc obj)
calls proc on each element of obj. Examples of such procedures are
\texttt{for-each}, \texttt{string-for-each}, and
\texttt{vector-for-each} from R7RS. The value returned by for-each is
ignored. The generator is finite if the collection is finite, which
would typically be the case.

The collections need not be conventional ones (lists, strings, etc.) as
long as \emph{for-each} can invoke a procedure on everything that counts
as a member. For example, the following procedure allows
\texttt{for-each-generator} to generate the digits of an integer from
least to most significant:

\begin{longtable}[]{@{}ll@{}}
\toprule
\begin{minipage}[t]{0.47\columnwidth}\raggedright\strut
~\strut
\end{minipage} & \begin{minipage}[t]{0.47\columnwidth}\raggedright\strut
\begin{verbatim}
(define (for-each-digit proc n)
  (when (> n 0)
    (let-values (((div rem) (truncate/ n 10)))
      (proc rem)
      (for-each-digit proc div))))
\end{verbatim}
\strut
\end{minipage}\tabularnewline
\bottomrule
\end{longtable}
\end{description}

\begin{description}
\item[\texttt{make-unfold-generator} \emph{stop? mapper successor seed}]
A generator constructor similar to
\href{http://srfi.schemers.org/srfi-1/srfi-1.html}{SRFI 1's}
\texttt{unfold}.

The stop? predicate takes a seed value and determines whether to stop.
The mapper procedure calculates a value to be returned by the generator
from a seed value. The successor procedure calculates the next seed
value from the current seed value.

For each call of the resulting generator, stop? is called with the
current seed value. If it returns true, then the generator returns an
end-of-file object. Otherwise, it applies mapper to the current seed
value to get the value to return, and uses successor to update the seed
value.

This generator is finite unless stop? never returns true.

\begin{longtable}[]{@{}ll@{}}
\toprule
\begin{minipage}[t]{0.47\columnwidth}\raggedright\strut
~\strut
\end{minipage} & \begin{minipage}[t]{0.47\columnwidth}\raggedright\strut
\begin{verbatim}
(generator->list (make-unfold-generator
                      (lambda (s) (> s 5))
                      (lambda (s) (* s 2))
                      (lambda (s) (+ s 1))
                      0))
  ⇒ (0 2 4 6 8 10)
\end{verbatim}
\strut
\end{minipage}\tabularnewline
\bottomrule
\end{longtable}
\end{description}

\subsection{Generator operations}\label{generator-operations}

The following procedures accept one or more generators and return a new
generator without consuming any elements from the source generator(s).
In general, the result will be a finite generator if the arguments are.

The names of these procedures are prefixed with \texttt{g}.

\begin{description}
\item[\texttt{gcons*} \emph{item \ldots{} gen}]
Returns a generator that adds items in front of gen. Once the items have
been consumed, the generator is guaranteed to tail-call gen.

\begin{longtable}[]{@{}ll@{}}
\toprule
\begin{minipage}[t]{0.47\columnwidth}\raggedright\strut
~\strut
\end{minipage} & \begin{minipage}[t]{0.47\columnwidth}\raggedright\strut
\begin{verbatim}
(generator->list (gcons* 'a 'b (make-range-generator 0 2)))
 ⇒ (a b 0 1)
\end{verbatim}
\strut
\end{minipage}\tabularnewline
\bottomrule
\end{longtable}
\end{description}

\begin{description}
\item[\texttt{gappend} \emph{gen \ldots{}}]
Returns a generator that yields the items from the first given
generator, and once it is exhausted, from the second generator, and so
on.

\begin{longtable}[]{@{}ll@{}}
\toprule
\begin{minipage}[t]{0.47\columnwidth}\raggedright\strut
~\strut
\end{minipage} & \begin{minipage}[t]{0.47\columnwidth}\raggedright\strut
\begin{verbatim}
(generator->list (gappend (make-range-generator 0 3) (make-range-generator 0 2)))
 ⇒ (0 1 2 0 1)

(generator->list (gappend))
 ⇒ ()
\end{verbatim}
\strut
\end{minipage}\tabularnewline
\bottomrule
\end{longtable}
\end{description}

\begin{description}
\item[\texttt{gcombine} \emph{proc seed gen gen\textsubscript{2}
\ldots{}}]
A generator for mapping with state. It yields a sequence of sub-folds
over proc.

The proc argument is a procedure that takes as many arguments as the
input generators plus one. It is called as \texttt{(}proc
v\textsubscript{1} v\textsubscript{2} \ldots{} seed), where
v\textsubscript{1}, v\textsubscript{2}, \ldots{} are the values yielded
from the input generators, and seed is the current seed value. It must
return two values, the yielding value and the next seed. The result
generator is exhausted when any of the \emph{gen\textsubscript{n}}
generators is exhausted, at which time all the others are in an
undefined state.
\end{description}

\begin{description}
\item[\texttt{gfilter} \emph{pred gen}\\
\texttt{gremove} \emph{pred gen}]
Return generators that yield the items from the source generator, except
those on which pred answers false or true respectively.
\end{description}

\begin{description}
\item[\texttt{gtake} \emph{gen k {[} padding {]}}\\
\texttt{gdrop} \emph{gen k}]
These are generator analogues of SRFI 1 \texttt{take} and \texttt{drop}.
\texttt{Gtake} returns a generator that yields (at most) the first k
items of the source generator, while \texttt{gdrop} returns a generator
that skips the first k items of the source generator.

These won't complain if the source generator is exhausted before
generating k items. By default, the generator returned by \texttt{gtake}
terminates when the source generator does, but if you provide the
padding argument, then the returned generator will yield exactly k
items, using the padding value as needed to provide sufficient
additional values.
\end{description}

\begin{description}
\item[\texttt{gtake-while} \emph{pred gen}\\
\texttt{gdrop-while} \emph{pred gen}]
The generator analogues of SRFI-1 \texttt{take-while} and
\texttt{drop-while}. The generator returned from \texttt{gtake-while}
yields items from the source generator as long as pred returns true for
each. The generator returned from \texttt{gdrop-while} first reads and
discards values from the source generator while pred returns true for
them, then starts yielding items returned by the source.
\end{description}

\begin{description}
\item[\texttt{gdelete} \emph{item gen {[} = {]}}]
Creates a generator that returns whatever gen returns, except for any
items that are the same as item in the sense of =, which defaults to
\texttt{equal?}. The = predicate is passed exactly two arguments, of
which the first was generated by gen before the second.

\begin{longtable}[]{@{}ll@{}}
\toprule
\begin{minipage}[t]{0.47\columnwidth}\raggedright\strut
~\strut
\end{minipage} & \begin{minipage}[t]{0.47\columnwidth}\raggedright\strut
\begin{verbatim}
(generator->list (gdelete 3 (generator 1 2 3 4 5 3 6 7)))
  ⇒ (1 2 4 5 6 7)
\end{verbatim}
\strut
\end{minipage}\tabularnewline
\bottomrule
\end{longtable}
\end{description}

\begin{description}
\item[\texttt{gdelete-neighbor-dups} \emph{gen {[} = {]}}]
Creates a generator that returns whatever gen returns, except for any
items that are equal to the preceding item in the sense of =, which
defaults to \texttt{equal?}. The = predicate is passed exactly two
arguments, of which the first was generated by gen before the second.

\begin{longtable}[]{@{}ll@{}}
\toprule
\begin{minipage}[t]{0.47\columnwidth}\raggedright\strut
~\strut
\end{minipage} & \begin{minipage}[t]{0.47\columnwidth}\raggedright\strut
\begin{verbatim}
(generator->list (gdelete-neighbor-dups (list->generator '(a a b c a a a d c))))
  ⇒ (a b c a d c)
\end{verbatim}
\strut
\end{minipage}\tabularnewline
\bottomrule
\end{longtable}
\end{description}

\begin{description}
\item[\texttt{gindex} \emph{value-gen index-gen}]
Creates a generator that returns elements of value-gen specified by the
indices (non-negative exact integers) generated by index-gen. It is an
error if the indices are not strictly increasing, or if any index
exceeds the number of elements generated by value-gen. The result
generator is exhausted when either generator is exhausted, at which time
the other is in an undefined state.

\begin{longtable}[]{@{}ll@{}}
\toprule
\begin{minipage}[t]{0.47\columnwidth}\raggedright\strut
~\strut
\end{minipage} & \begin{minipage}[t]{0.47\columnwidth}\raggedright\strut
\begin{verbatim}
(generator->list (gindex (list->generator '(a b c d e f))
                         (list->generator '(0 2 4))))
  ⇒ (a c e)
\end{verbatim}
\strut
\end{minipage}\tabularnewline
\bottomrule
\end{longtable}
\end{description}

\begin{description}
\item[\texttt{gselect} \emph{value-gen truth-gen}]
Creates a generator that returns elements of value-gen that correspond
to the values generated by truth-gen. If the current value of truth-gen
is true, the current value of value-gen is generated, but otherwise not.
The result generator is exhausted when either generator is exhausted, at
which time the other is in an undefined state.

\begin{longtable}[]{@{}ll@{}}
\toprule
\begin{minipage}[t]{0.47\columnwidth}\raggedright\strut
~\strut
\end{minipage} & \begin{minipage}[t]{0.47\columnwidth}\raggedright\strut
\begin{verbatim}
(generator->list (gselect (list->generator '(a b c d e f))
                          (list->generator '(#t #f #f #t #t #f))))
  ⇒ (a d e)
\end{verbatim}
\strut
\end{minipage}\tabularnewline
\bottomrule
\end{longtable}
\end{description}

\subsection{Consuming generated
values}\label{consuming-generated-values}

Unless otherwise noted, these procedures consume all the values
available from the generator that is passed to them, and therefore will
not return if one or more generator arguments are infinite. They have
names prefixed with \texttt{generator}.

\begin{description}
\item[\texttt{generator-\textgreater{}list} \emph{generator {[} k {]}}]
Reads items from generator and returns a newly allocated list of them.
By default, it reads until the generator is exhausted.

If an optional argument k is given, it must be a non-negative integer,
and the list ends when either k items are consumed, or \emph{generator}
is exhausted; therefore \emph{generator} can be infinite in this case.
\end{description}

\begin{description}
\item[\texttt{generator-\textgreater{}reverse-list} \emph{generator {[}
k {]}}]
Reads items from generator and returns a newly allocated list of them in
reverse order. By default, this reads until the generator is exhausted.

If an optional argument k is given, it must be a non-negative integer,
and the list ends when either k items are read, or \emph{generator} is
exhausted; therefore \emph{generator} can be infinite in this case.
\end{description}

\begin{description}
\item[\texttt{generator-\textgreater{}vector} \emph{generator {[} k
{]}}]
Reads items from generator and returns a newly allocated vector of them.
By default, it reads until the generator is exhausted.

If an optional argument k is given, it must be a non-negative integer,
and the list ends when either k items are consumed, or \emph{generator}
is exhausted; therefore \emph{generator} can be infinite in this case.
\end{description}

\begin{description}
\item[\texttt{generator-\textgreater{}vector!} \emph{vector at
generator}]
Reads items from generator and puts them into vector starting at index
at, until vector is full or generator is exhausted. \emph{Generator} can
be infinite. The number of elements generated is returned.
\end{description}

\begin{description}
\item[\texttt{generator-\textgreater{}string} \emph{generator {[} k
{]}}]
Reads items from generator and returns a newly allocated string of them.
It is an error if the items are not characters. By default, it reads
until the generator is exhausted.

If an optional argument k is given, it must be a non-negative integer,
and the string ends when either k items are consumed, or
\emph{generator} is exhausted; therefore \emph{generator} can be
infinite in this case.
\end{description}

\begin{description}
\item[\texttt{generator-fold} \emph{proc seed gen\textsubscript{1}
gen\textsubscript{2} \ldots{}}]
Works like SRFI 1 \texttt{fold} on the values generated by the generator
arguments.

When one generator is given, for each value v generated by gen, proc is
called as \texttt{(proc\ v\ r)}, where r is the current accumulated
result; the initial value of the accumulated result is seed, and the
return value from proc becomes the next accumulated result. When gen is
exhausted, the accumulated result at that time is returned from
\texttt{generator-fold}.

When more than one generator is given, proc is invoked on the values
returned by all the generator arguments followed by the current
accumulated result. The procedure terminates when any of the
\emph{gen\textsubscript{n}} generators is exhausted, at which time all
the others are in an undefined state.

\begin{longtable}[]{@{}ll@{}}
\toprule
\begin{minipage}[t]{0.47\columnwidth}\raggedright\strut
~\strut
\end{minipage} & \begin{minipage}[t]{0.47\columnwidth}\raggedright\strut
\begin{verbatim}
(with-input-from-string "a b c d e"
  (lambda () (generator-fold cons 'z read)))
  ⇒ (e d c b a . z)
\end{verbatim}
\strut
\end{minipage}\tabularnewline
\bottomrule
\end{longtable}
\end{description}

\begin{description}
\item[\texttt{generator-for-each} \emph{proc gen gen\textsubscript{2}
\ldots{}}]
A generator analogue of \texttt{for-each} that consumes generated values
using side effects. Repeatedly applies proc on the values yielded by
gen, gen\textsubscript{2} \ldots{} until any one of the generators is
exhausted. The values returned from proc are discarded. The procedure
terminates when any of the \emph{gen\textsubscript{n}} generators is
exhausted, at which time all the others are in an undefined state.
Returns an unspecified value.
\end{description}

\begin{description}
\item[\texttt{generator-find} \emph{pred gen}]
Returns the first item from the generator gen that satisfies the
predicate pred, or \texttt{\#f} if no such item is found before gen is
exhausted. If gen is infinite, \texttt{generator-find} will not return
if it cannot find an appropriate item.
\end{description}

\begin{description}
\item[\texttt{generator-count} \emph{pred gen}]
Returns the number of items available from the generator gen that
satisfy the predicate pred.
\end{description}

\begin{description}
\item[\texttt{generator-any} \emph{pred gen}]
Applies \emph{pred} to each item from \emph{gen}. As soon as it yields a
true value, the value is returned without consuming the rest of
\emph{gen}. If \emph{gen} is exhausted, returns \texttt{\#f}.
\end{description}

\begin{description}
\item[\texttt{generator-every} \emph{pred gen}]
Applies \emph{pred} to each item from \emph{gen}. As soon as it yields a
false value, the value is returned without consuming the rest of
\emph{gen}. If \emph{gen} is exhausted, returns the last value returned
by \emph{pred}, or \texttt{\#t} if \emph{pred} was never called.
\end{description}

\begin{description}
\item[\texttt{generator-unfold} \emph{gen unfold arg \ldots{}}]
Equivalent to \texttt{(}unfold
\texttt{eof-object?\ (lambda\ (x)\ x)\ (lambda\ (x)\ (}gen\texttt{))\ (}gen\texttt{)}
arg \ldots{}\texttt{)}. The values of gen are unfolded into the
collection that unfold creates.

The signature of the unfold procedure is \texttt{(}unfold stop? mapper
successor seed args \ldots{}\texttt{)}. Note that the
\texttt{vector-unfold} and \texttt{vector-unfold-right} of
\href{http://srfi.schemers.org/srfi-43/srfi-43.html}{SRFI 43} and
\href{http://srfi.schemers.org/srfi-133/srfi-133.html}{SRFI 133} do not
have this signature and cannot be used with this procedure. To unfold
into a vector, use SRFI 1's \texttt{unfold} and then apply
\texttt{list-\textgreater{}vector} to the result.

\begin{longtable}[]{@{}ll@{}}
\toprule
\begin{minipage}[t]{0.47\columnwidth}\raggedright\strut
~\strut
\end{minipage} & \begin{minipage}[t]{0.47\columnwidth}\raggedright\strut
\begin{verbatim}
;; Iterates over string and unfolds into a list using SRFI 1 unfold
(generator-unfold (make-for-each-generator string-for-each "abc") unfold)
  ⇒ (#\a #\b #\c)
\end{verbatim}
\strut
\end{minipage}\tabularnewline
\bottomrule
\end{longtable}
\end{description}

\section{Implementation}\label{implementation}

The sample implementation is in the SRFI 121 repository. It contains the
following files:

\begin{itemize}
\tightlist
\item
  \texttt{generators-impl.scm} - implementation of generators
\item
  \texttt{r7rs-shim.scm} - supplementary code for non-R7RS systems
\item
  \texttt{generators.sld} - R7RS library
\item
  \texttt{generators.scm} - Chicken library
\item
  \texttt{generators-test.scm} - Chicken test-egg test file
\end{itemize}

\section{Copyright}\label{copyright}

Copyright (C) Shiro Kawai, John Cowan, Thomas Gilray (2015). All Rights
Reserved.

Permission is hereby granted, free of charge, to any person obtaining a
copy of this software and associated documentation files (the
``Software''), to deal in the Software without restriction, including
without limitation the rights to use, copy, modify, merge, publish,
distribute, sublicense, and/or sell copies of the Software, and to
permit persons to whom the Software is furnished to do so, subject to
the following conditions:

The above copyright notice and this permission notice shall be included
in all copies or substantial portions of the Software.

THE SOFTWARE IS PROVIDED ``AS IS'', WITHOUT WARRANTY OF ANY KIND,
EXPRESS OR IMPLIED, INCLUDING BUT NOT LIMITED TO THE WARRANTIES OF
MERCHANTABILITY, FITNESS FOR A PARTICULAR PURPOSE AND NONINFRINGEMENT.
IN NO EVENT SHALL THE AUTHORS OR COPYRIGHT HOLDERS BE LIABLE FOR ANY
CLAIM, DAMAGES OR OTHER LIABILITY, WHETHER IN AN ACTION OF CONTRACT,
TORT OR OTHERWISE, ARISING FROM, OUT OF OR IN CONNECTION WITH THE
SOFTWARE OR THE USE OR OTHER DEALINGS IN THE SOFTWARE.

\begin{center}\rule{0.5\linewidth}{\linethickness}\end{center}

Editor:
\href{mailto:srfi-editors\%20at\%20srfi\%20dot\%20schemers\%20\%20\%20\%20dot\%20org}{Arthur
Gleckler}

Last modified: Wed Jun 10 08:57:14 MST 2015
