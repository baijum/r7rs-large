\section{Title}\label{title}

Comparators (reduced)

\section{Author}\label{author}

John Cowan

\section{Status}\label{status}

This SRFI is currently in \emph{final} status. Here is
\href{http://srfi.schemers.org/srfi-process.html}{an explanation} of
each status that a SRFI can hold. To provide input on this SRFI, please
send email to \texttt{srfi-128@nospamsrfi.schemers.org}. To subscribe to
the list, follow
\href{http://srfi.schemers.org/srfi-list-subscribe.html}{these
instructions}. You can access previous messages via the mailing list
\href{http://srfi-email.schemers.org/srfi-128}{archive}.

\begin{itemize}
\tightlist
\item
  Received: 2015/10/26
\item
  60-day deadline: 2015/12/25
\item
  Draft \#1 published: 2015/10/26
\item
  Draft \#2 published: 2015/10/26
\item
  Draft \#3 published: 2015/10/29
\item
  Draft \#4 published: 2015/11/2
\item
  Draft \#5 published: 2015/11/8
\item
  Draft \#6 published: 2015/11/10
\item
  Draft \#7 published: 2015/11/23
\item
  Draft \#8 published: 2015/12/17
\item
  Draft \#9 published: 2015/12/25
\item
  Draft \#10 published: 2016/1/18
\item
  Draft \#11 published: 2016/2/6
\item
  Draft \#12 published: 2016/2/14
\item
  Finalized: 2016/2/14
\item
  Revised to fix errata: 2016/5/14
\end{itemize}

\textbf{Post-finalization note}: Because of the extremely high cost of
conforming to the first and third conditions of \texttt{default-hash},
implementers may disregard those conditions and examine only a bounded
portion of the argument.

\section{Abstract}\label{abstract}

This SRFI provides \emph{comparators}, which bundle a type test
predicate, an equality predicate, an ordering predicate, and a hash
function (the last two are optional) into a single Scheme object. By
packaging these procedures together, they can be treated as a single
item for use in the implementation of data structures.

\section{Issues}\label{issues}

None at present.

\section{Rationale}\label{rationale}

The four procedures above have complex dependencies on one another, and
it is inconvenient to have to pass them individually to other procedures
that might or might not make use of all of them. For example, a set
implementation by its nature requires only an equality predicate, but if
it is implemented using a hash table, an appropriate hash function is
also required if the implementation does not provide one; alternatively,
if it is implemented using a tree, procedures specifying a total order
are required. By passing a comparator rather than a bare equality
predicate, the set implementation can make use of whatever procedures
are available and useful to it.

This SRFI is a simplified and enhanced rewrite of
\href{http://srfi.schemers.org/srfi-114/srfi-114.html}{SRFI 114}, and
shares some of its design rationale and all of its acknowledgements. The
largest change is the replacement of the comparison procedure with the
ordering procedure. This allowed most of the special-purpose comparators
to be removed. In addition, many of the more specialized procedures, as
well as all but one of the syntax forms, have been removed as
unnecessary.

Special thanks to Taylan Ulrich Bayırlı/Kammer, whose insistence that
SRFI 114 was unacceptable inspired this redesign. Jörg Wittenberger
added Chicken-specific type declarations, which I have moved to
\texttt{comparators.scm}, as it is a Chicken-specific library. He also
provided Chicken-specific metadata and setup commands. Comments from
Shiro Kawai, Alex Shinn, and Kevin Wortman guided me to the current
design for bounds and salt.

\section{Specification}\label{specification}

The procedures in this SRFI are in the \texttt{(srfi\ 128)} library (or
\texttt{(srfi\ :128)} on R6RS), but the sample implementation currently
places them in the \texttt{(comparators)} library. This means it can't
be used alongside SRFI 114, but there's no reason for anyone to do that.

\subsubsection{Definitions}\label{definitions}

A \emph{comparator} is an object of a disjoint type. It is a bundle of
procedures that are useful for comparing two objects in a total order.
It is an error if any of the procedures have side effects. There are
four procedures in the bundle:

\begin{itemize}
\item
  The \emph{type test predicate} returns \texttt{\#t} if its argument
  has the correct type to be passed as an argument to the other three
  procedures, and \texttt{\#f} otherwise.
\item
  The \emph{equality predicate} returns \texttt{\#t} if the two objects
  are the same in the sense of the comparator, and \texttt{\#f}
  otherwise. It is the programmer's responsibility to ensure that it is
  reflexive, symmetric, transitive, and can handle any arguments that
  satisfy the type test predicate.
\item
  The \emph{ordering predicate} returns \texttt{\#t} if the first object
  precedes the second in a total order, and \texttt{\#f} otherwise. Note
  that if it is true, the equality predicate must be false. It is the
  programmer's responsibility to ensure that it is irreflexive,
  antisymmetric, transitive, and can handle any arguments that satisfy
  the type test predicate.
\item
  The \emph{hash function} takes an object and returns an exact
  non-negative integer. It is the programmer's responsibility to ensure
  that it can handle any argument that satisfies the type test
  predicate, and that it returns the same value on two objects if the
  equality predicate says they are the same (but not necessarily the
  converse).
\end{itemize}

It is also the programmer's responsibility to ensure that all four
procedures provide the same result whenever they are applied to the same
object(s) (in the sense of \texttt{eqv?}), unless the object(s) have
been mutated since the last invocation. In particular, they must not
depend in any way on memory addresses in implementations where the
garbage collector can move objects in memory.

\subsubsection{Limitations}\label{limitations}

The comparator objects defined in this SRFI are not applicable to
circular structure or to NaNs, or to objects containing any of these.
Attempts to pass any such objects to any procedure defined here, or to
any procedure that is part of a comparator defined here, is an error
except as otherwise noted.

\subsubsection{Index}\label{index}

\begin{itemize}
\item
  \protect\hyperlink{Predicates}{Predicates}:
  \texttt{comparator?\ comparator-ordered?\ comparator-hashable?}
\item
  \protect\hyperlink{Constructors}{Constructors}:
  \texttt{make-comparator\ \ make-pair-comparator\ make-list-comparator\ make-vector-comparator\ make-eq-comparator\ make-eqv-comparator\ make-equal-comparator}
\item
  \protect\hyperlink{Hashfunctions}{Standard hash functions}:
  \texttt{boolean-hash\ char-hash\ char-ci-hash\ string-hash\ string-ci-hash\ symbol-hash\ number-hash}
\item
  \protect\hyperlink{Boundsandsalt}{Bounds and salt}:
  \texttt{hash-bound\ hash-salt}
\item
  \protect\hyperlink{Defaultcomparators}{Default comparators}:
  \texttt{make-default-comparator\ default-hash\ comparator-register-default!}
\item
  \protect\hyperlink{Accessorsandinvokers}{Accessors and invokers}:
  \texttt{comparator-type-test-predicate\ comparator-equality-predicate\ comparator-ordering-predicate\ comparator-hash-function\ comparator-test-type\ comparator-check-type\ comparator-hash}
\item
  \protect\hyperlink{Comparisonpredicates}{Comparison predicates}:
  \texttt{=?\ \textless{}?\ \textgreater{}?\ \textless{}=?\ \textgreater{}=?}
\item
  \protect\hyperlink{Syntax}{Syntax}:
  \texttt{comparator-if\textless{}=\textgreater{}}
\end{itemize}

\subsubsection{Predicates}\label{predicates}

\texttt{(comparator?\ }\emph{obj}\texttt{)}

Returns \texttt{\#t} if \emph{obj} is a comparator, and \texttt{\#f}
otherwise.

\texttt{(comparator-ordered?\ }\emph{comparator}\texttt{)}

Returns \texttt{\#t} if \emph{comparator} has a supplied ordering
predicate, and \texttt{\#f} otherwise.

\texttt{(comparator-hashable?\ }\emph{comparator}\texttt{)}

Returns \texttt{\#t} if \emph{comparator} has a supplied hash function,
and \texttt{\#f} otherwise.

\hypertarget{Constructors}{\subsubsection{Constructors}\label{Constructors}}

The following comparator constructors all supply appropriate type test
predicates, equality predicates, ordering predicates, and hash functions
based on the supplied arguments. They are allowed to cache their
results: they need not return a newly allocated object, since
comparators are pure and functional. In addition, the procedures in a
comparator are likewise pure and functional.

\texttt{(make-comparator\ }\emph{type-test equality ordering
hash}\texttt{)}

Returns a comparator which bundles the \emph{type-test, equality,
ordering}, and \emph{hash} procedures provided. However, if
\emph{ordering} or \emph{hash} is \texttt{\#f}, a procedure is provided
that signals an error on application. The predicates
\texttt{comparator-ordered?} and/or \texttt{comparator-hashable?},
respectively, will return \texttt{\#f} in these cases.

Here are calls on \texttt{make-comparator} that will return useful
comparators for standard Scheme types:

\begin{itemize}
\item
  \texttt{(make-comparator\ boolean?\ boolean=?\ (lambda\ (x\ y)\ (and\ (not\ x)\ y))\ boolean-hash)}
  will return a comparator for booleans, expressing the ordering
  \texttt{\#f} \textless{} \texttt{\#t} and the standard hash function
  for booleans.
\item
  \texttt{(make-comparator\ real?\ =\ \textless{}\ (lambda\ (x)\ (exact\ (abs\ x))))}
  will return a comparator expressing the natural ordering of real
  numbers and a plausible (but not optimal) hash function.
\item
  \texttt{(make-comparator\ string?\ string=?\ string\textless{}?\ string-hash)}
  will return a comparator expressing the implementation's ordering of
  strings and the standard hash function.
\item
  \texttt{(make-comparator\ string?\ string-ci=?\ string-ci\textless{}?\ string-ci-hash}
  will return a comparator expressing the implementation's
  case-insensitive ordering of strings and the standard case-insensitive
  hash function.
\end{itemize}

\texttt{(make-pair-comparator\ } car-comparator cdr-comparator\texttt{)}

This procedure returns comparators whose functions behave as follows.

\begin{itemize}
\item
  The type test returns \texttt{\#t} if its argument is a pair, if the
  car satisfies the type test predicate of car-comparator, and the cdr
  satisfies the type test predicate of cdr-comparator.
\item
  The equality function returns \texttt{\#t} if the cars are equal
  according to car-comparator and the cdrs are equal according to
  cdr-comparator, and \texttt{\#f} otherwise.
\item
  The ordering function first compares the cars of its pairs using the
  equality predicate of car-comparator. If they are equal, then the
  ordering predicate of car-comparator is applied to the cars and its
  value is returned. Otherwise, the predicate compares the cdrs using
  the equality predicate of cdr-comparator. If they are not equal, then
  the ordering predicate of cdr-comparator is applied to the cdrs and
  its value is returned.
\item
  The hash function computes the hash values of the car and the cdr
  using the hash functions of car-comparator and cdr-comparator
  respectively and then hashes them together in an
  implementation-defined way.
\end{itemize}

\texttt{(make-list-comparator\ } \emph{element-comparator}
\emph{type-test empty? head tail}\texttt{)}

This procedure returns comparators whose functions behave as follows:

\begin{itemize}
\item
  The type test returns \texttt{\#t} if its argument satisfies type-test
  and the elements satisfy the type test predicate of
  element-comparator.
\item
  The total order defined by the equality and ordering functions is as
  follows (known as lexicographic order):

  \begin{itemize}
  \tightlist
  \item
    The empty sequence, as determined by calling empty?, compares equal
    to itself.
  \item
    The empty sequence compares less than any non-empty sequence.
  \item
    Two non-empty sequences are compared by calling the head procedure
    on each. If the heads are not equal when compared using
    element-comparator, the result is the result of that comparison.
    Otherwise, the results of calling the tail procedure are compared
    recursively.
  \end{itemize}
\item
  The hash function computes the hash values of the elements using the
  hash function of element-comparator and then hashes them together in
  an implementation-defined way.
\end{itemize}

\texttt{(make-vector-comparator\ } \emph{element-comparator}
\emph{type-test} \emph{length ref}\texttt{)}

This procedure returns comparators whose functions behave as follows:

\begin{itemize}
\item
  The type test returns \texttt{\#t} if its argument satisfies type-test
  and the elements satisfy the type test predicate of
  element-comparator.
\item
  The equality predicate returns \texttt{\#t} if both of the following
  tests are satisfied in order: the lengths of the vectors are the same
  in the sense of =, and the elements of the vectors are the same in the
  sense of the equality predicate of element-comparator.
\item
  The ordering predicate returns \texttt{\#t} if the results of applying
  length to the first vector is less than the result of applying length
  to the second vector. If the lengths are equal, then the elements are
  examined pairwise using the ordering predicate of element-comparator.
  If any pair of elements returns \texttt{\#t}, then that is the result
  of the list comparator's ordering predicate; otherwise the result is
  \texttt{\#f}
\item
  The hash function computes the hash values of the elements using the
  hash function of element-comparator and then hashes them together in
  an implementation-defined way.
\end{itemize}

Here is an example, which returns a comparator for byte vectors:

\begin{verbatim}
(make-vector-comparator
  (make-comparator exact-integer? = < number-hash)
  bytevector?
  bytevector-length
  bytevector-u8-ref)
\end{verbatim}

\texttt{(make-eq-comparator)}

\texttt{(make-eqv-comparator)}

\texttt{(make-equal-comparator)}

These procedures return comparators whose functions behave as follows:

\begin{itemize}
\item
  The type test returns \texttt{\#t} in all cases.
\item
  The equality functions are \texttt{eq?}, \texttt{eqv?}, and
  \texttt{equal?} respectively.
\item
  The ordering function is implementation-defined, except that it must
  conform to the rules for ordering functions. It may signal an error
  instead.
\item
  The hash function is \texttt{default-hash}.
\end{itemize}

These comparators accept circular structure (in the case of
\texttt{equal-comparator}, provided the implementation's \texttt{equal?}
predicate does so) and NaNs.

\hypertarget{Hashfunctions}{\subsubsection{Standard hash
functions}\label{Hashfunctions}}

These are hash functions for some standard Scheme types, suitable for
passing to \texttt{make-comparator}. Users may write their own hash
functions with the same signature. However, if programmers wish their
hash functions to be backward compatible with the reference
implementation of
\href{http://srfi.schemers.org/srfi-69/srfi-69.html}{SRFI 69}, they are
advised to write their hash functions to accept a second argument and
ignore it.

\texttt{(boolean-hash} obj\texttt{)}

\texttt{(char-hash} obj\texttt{)}

\texttt{(char-ci-hash} obj\texttt{)}

\texttt{(string-hash} obj\texttt{)}

\texttt{(string-ci-hash} obj\texttt{)}

\texttt{(symbol-hash} obj\texttt{)}

\texttt{(number-hash} obj\texttt{)}

These are suitable hash functions for the specified types. The hash
functions \texttt{char-ci-hash} and \texttt{string-ci-hash} treat their
argument case-insensitively. Note that while \texttt{symbol-hash} may
return the hashed value of applying \texttt{symbol-\textgreater{}string}
and then \texttt{string-hash} to the symbol, this is not a requirement.

\hypertarget{Boundsandsalt}{\subsubsection{Bounds and
salt}\label{Boundsandsalt}}

The following macros allow the callers of hash functions to affect their
behavior without interfering with the calling signature of a hash
function, which accepts a single argument (the object to be hashed) and
returns its hash value. They are provided as macros so that they may be
implemented in different ways: as a global variable, a SRFI 39 or R7RS
parameter, or an ordinary procedure, whatever is most efficient in a
particular implementation.

\texttt{(hash-bound)} {[}syntax{]}

Hash functions should be written so as to return a number between 0 and
the largest reasonable number of elements (such as hash buckets) a data
structure in the implementation might have. What that value is depends
on the implementation. This value provides the current bound as a
positive exact integer, typically for use by user-written hash
functions. However, they are not required to bound their results in this
way.

\texttt{(hash-salt)} {[}syntax{]}

A salt is random data in the form of a non-negative exact integer used
as an additional input to a hash function in order to defend against
dictionary attacks, or (when used in hash tables) against
denial-of-service attacks that overcrowd certain hash buckets,
increasing the amortized O(1) lookup time to O(n). Salt can also be used
to specify which of a family of hash functions should be used for
purposes such as cuckoo hashing. This macro provides the current value
of the salt, typically for use by user-written hash functions. However,
they are not required to make use of the current salt.

The initial value is implementation-dependent, but must be less than the
value of \texttt{(hash-bound)}, and should be distinct for distinct runs
of a program unless otherwise specified by the implementation.
Implementations may provide a means to specify the salt value to be used
by a particular invocation of a hash function.

\hypertarget{Defaultcomparators}{\subsubsection{Default
comparators}\label{Defaultcomparators}}

\texttt{(make-default-comparator)}

Returns a comparator known as a default comparator that accepts Scheme
values and orders them in some implementation-defined way, subject to
the following conditions:

\begin{itemize}
\item
  Given disjoint types \emph{a} and \emph{b}, one of three conditions
  must hold:

  \begin{itemize}
  \tightlist
  \item
    All objects of type \emph{a} compare less than all objects of type
    \emph{b}.
  \item
    All objects of type \emph{a} compare greater than all objects of
    type \emph{b}.
  \item
    All objects of both type \emph{a} and type \emph{b} compare equal to
    each other. This is not permitted for any of the Scheme types
    mentioned below.
  \end{itemize}
\item
  The empty list must be ordered before all pairs.
\item
  When comparing booleans, it must use the total order \texttt{\#f}
  \textless{} \texttt{\#t}.
\item
  When comparing characters, it must use \texttt{char=?} and
  \texttt{char\textless{}?}.

  Note: In R5RS, this is an implementation-dependent order that is
  typically the same as Unicode codepoint order; in R6RS and R7RS, it is
  Unicode codepoint order.
\item
  When comparing pairs, it must behave the same as a comparator returned
  by \texttt{make-pair-comparator} with default comparators as
  arguments.
\item
  When comparing symbols, it must use an implementation-dependent total
  order. One possibility is to use the order obtained by applying
  \texttt{symbol-\textgreater{}string} to the symbols and comparing them
  using the total order implied by \texttt{string\textless{}?}.
\item
  When comparing bytevectors, it must behave the same as a comparator
  created by the expression
  \texttt{(make-vector-comparator\ (make-comparator\ \textless{}\ =\ number-hash)\ bytevector?\ bytevector-length\ bytevector-u8-ref)}.
\item
  When comparing numbers where either number is complex, since non-real
  numbers cannot be compared with \texttt{\textless{}}, the following
  least-surprising ordering is defined: If the real parts are
  \textless{} or \textgreater{}, so are the numbers; otherwise, the
  numbers are ordered by their imaginary parts. This can still produce
  somewhat surprising results if one real part is exact and the other is
  inexact.
\item
  When comparing real numbers, it must use \texttt{=} and
  \texttt{\textless{}}.
\item
  When comparing strings, it must use \texttt{string=?} and
  \texttt{string\textless{}?}.

  Note: In R5RS, this is lexicographic order on the
  implementation-dependent order defined by \texttt{char\textless{}?};
  in R6RS it is lexicographic order on Unicode codepoint order; in R7RS
  it is an implementation-defined order.
\item
  When comparing vectors, it must behave the same as a comparator
  returned by
  \texttt{(make-vector-comparator\ (make-default-comparator)\ vector?\ vector-length\ vector-ref)}.
\item
  When comparing members of types registered with
  \texttt{comparator-register-default!}, it must behave in the same way
  as the comparator registered using that function.
\end{itemize}

Default comparators use \texttt{default-hash} as their hash function.

\texttt{(default-hash} obj\texttt{)}

This is the hash function used by default comparators, which accepts a
Scheme value and hashes it in some implementation-defined way, subject
to the following conditions:

\begin{itemize}
\tightlist
\item
  When applied to a pair, it must return the result of hashing together
  the values returned by \texttt{default-hash} when applied to the car
  and the cdr.
\item
  When applied to a boolean, character, string, symbol, or number, it
  must return the same result as \texttt{boolean-hash},
  \texttt{char-hash}, \texttt{string-hash}, \texttt{symbol-hash}, or
  \texttt{number-hash} respectively.
\item
  When applied to a list or vector, it must return the result of hashing
  together the values returned by \texttt{default-hash} when applied to
  each of the elements.
\end{itemize}

\texttt{(comparator-register-default!\ }comparator\texttt{)}

Registers comparator for use by default comparators, such that if the
objects being compared both satisfy the type test predicate of
comparator, it will be employed by default comparators to compare them.
Returns an unspecified value. It is an error if any value satisfies both
the type test predicate of comparator and any of the following type test
predicates: \texttt{boolean?}, \texttt{char?}, \texttt{null?},
\texttt{pair?}, \texttt{symbol?}, \texttt{bytevector?},
\texttt{number?}, \texttt{string?}, \texttt{vector?}, or the type test
predicate of a comparator that has already been registered.

This procedure is intended only to extend default comparators into
territory that would otherwise be undefined, not to override their
existing behavior. In general, the ordering of calls to
\texttt{comparator-register-default!} should be irrelevant. However,
implementations that support inheritance of record types may wish to
ensure that default comparators always check subtypes before supertypes.

\hypertarget{Accessorsandinvokers}{\subsubsection{Accessors and
Invokers}\label{Accessorsandinvokers}}

\texttt{(comparator-type-test-predicate\ }\emph{comparator}\texttt{)}

\texttt{(comparator-equality-predicate\ }\emph{comparator}\texttt{)}

\texttt{(comparator-ordering-predicate\ }\emph{comparator}\texttt{)}

\texttt{(comparator-hash-function\ }\emph{comparator}\texttt{)}

Return the four procedures of \emph{comparator}.

\texttt{(comparator-test-type\ }\emph{comparator obj}\texttt{)}

Invokes the type test predicate of \emph{comparator} on \emph{obj} and
returns what it returns. More convenient than
\texttt{comparator-type-test-predicate}, but less efficient when the
predicate is called repeatedly.

\texttt{(comparator-check-type\ }\emph{comparator obj}\texttt{)}

Invokes the type test predicate of \emph{comparator} on \emph{obj} and
returns true if it returns true, but signals an error otherwise. More
convenient than \texttt{comparator-type-test-predicate}, but less
efficient when the predicate is called repeatedly.

\texttt{(comparator-hash\ }\emph{comparator obj}\texttt{)}

Invokes the hash function of \emph{comparator} on obj and returns what
it returns. More convenient than \texttt{comparator-hash-function}, but
less efficient when the function is called repeatedly.

Note: No invokers are required for the equality and ordering predicates,
because \texttt{=?} and \texttt{\textless{}?} serve this function.

\hypertarget{Comparisonpredicates}{\subsubsection{Comparison
predicates}\label{Comparisonpredicates}}

\texttt{(=?\ }\emph{comparator} \emph{object\textsubscript{1}
object\textsubscript{2} object\textsubscript{3}} \ldots{}\texttt{)}

\texttt{(\textless{}?\ }\emph{comparator} \emph{object\textsubscript{1}
object\textsubscript{2} object\textsubscript{3}} \ldots{}\texttt{)}

\texttt{(\textgreater{}?\ }\emph{comparator}
\emph{object\textsubscript{1} object\textsubscript{2}
object\textsubscript{3}} \ldots{}\texttt{)}

\texttt{(\textless{}=?\ }\emph{comparator} \emph{object\textsubscript{1}
object\textsubscript{2} object\textsubscript{3}} \ldots{}\texttt{)}

\texttt{(\textgreater{}=?\ }\emph{comparator}
\emph{object\textsubscript{1} object\textsubscript{2}
object\textsubscript{3}} \ldots{}\texttt{)}

These procedures are analogous to the number, character, and string
comparison predicates of Scheme. They allow the convenient use of
comparators to handle variable data types.

These procedures apply the equality and ordering predicates of
\emph{comparator} to the \emph{objects} as follows. If the specified
relation returns \texttt{\#t} for all \emph{object\textsubscript{i}} and
\emph{object\textsubscript{j}} where \emph{n} is the number of objects
and 1 \textless{}= \emph{i \textless{} j \textless{}= n}, then the
procedures return \texttt{\#t}, but otherwise \texttt{\#f}. Because the
relations are transitive, it suffices to compare each object with its
successor. The order in which the values are compared is unspecified.

\hypertarget{Syntax}{\subsubsection{Syntax}\label{Syntax}}

\texttt{(comparator-if\textless{}=\textgreater{}\ }{[}
\textless{}comparator\textgreater{} {]}
\textless{}object\textsubscript{1}\textgreater{}
\textless{}object\textsubscript{2}\textgreater{}
\textless{}less-than\textgreater{} \textless{}equal-to\textgreater{}
\textless{}greater-than\textgreater{}\texttt{)}

It is an error unless \textless{}comparator\textgreater{} evaluates to a
comparator and \textless{}object\textsubscript{1}\textgreater{} and
\textless{}object\textsubscript{2}\textgreater{} evaluate to objects
that the comparator can handle. If the ordering predicate returns true
when applied to the values of
\textless{}object\textsubscript{1}\textgreater{} and
\textless{}object\textsubscript{2}\textgreater{} in that order, then
\textless{}less-than\textgreater{} is evaluated and its value returned.
If the equality predicate returns true when applied in the same way,
then \textless{}equal-to\textgreater{} is evaluated and its value
returned. If neither returns true, \textless{}greater-than\textgreater{}
is evaluated and its value returned.

If \textless{}comparator\textgreater{} is omitted, a default comparator
is used.

\section{Implementation}\label{implementation}

The \href{comparators.tar.gz}{sample implementation} is found in the
repository of this SRFI. It contains the following files.

\begin{itemize}
\tightlist
\item
  \texttt{comparators-impl.scm} - the record type definition and most of
  the procedures
\item
  \texttt{default.scm} - a simple implementation of the default
  constructor, which should be improved by implementers to handle
  records and implementation-specific types
\item
  \texttt{r7rs-shim.scm} - procedures for R7RS compatibility, including
  a trivial implementation of bytevectors on top of
  \href{http://srfi.schemers.org/srfi-4/srfi-4.html}{SRFI 4} u8vectors
\item
  \texttt{complex-shim.scm} - a trivial implementation of
  \texttt{real-part} and \texttt{imag-part} for Schemes that don't have
  complex numbers
\item
  \texttt{comparators.meta} - Chicken-specific metadata
\item
  \texttt{comparators.setup} - Chicken-specific executable setup
\item
  \texttt{comparators.sld} - an R7RS library
\item
  \texttt{comparators.scm} - a Chicken library
\item
  \texttt{comparators-test.scm} - a test file using the Chicken
  \texttt{test} egg
\end{itemize}

\section{Copyright}\label{copyright}

Copyright (C) John Cowan (2015). All Rights Reserved.

Permission is hereby granted, free of charge, to any person obtaining a
copy of this software and associated documentation files (the
``Software''), to deal in the Software without restriction, including
without limitation the rights to use, copy, modify, merge, publish,
distribute, sublicense, and/or sell copies of the Software, and to
permit persons to whom the Software is furnished to do so, subject to
the following conditions:

The above copyright notice and this permission notice shall be included
in all copies or substantial portions of the Software.

THE SOFTWARE IS PROVIDED ``AS IS'', WITHOUT WARRANTY OF ANY KIND,
EXPRESS OR IMPLIED, INCLUDING BUT NOT LIMITED TO THE WARRANTIES OF
MERCHANTABILITY, FITNESS FOR A PARTICULAR PURPOSE AND NONINFRINGEMENT.
IN NO EVENT SHALL THE AUTHORS OR COPYRIGHT HOLDERS BE LIABLE FOR ANY
CLAIM, DAMAGES OR OTHER LIABILITY, WHETHER IN AN ACTION OF CONTRACT,
TORT OR OTHERWISE, ARISING FROM, OUT OF OR IN CONNECTION WITH THE
SOFTWARE OR THE USE OR OTHER DEALINGS IN THE SOFTWARE.

\begin{center}\rule{0.5\linewidth}{\linethickness}\end{center}

Editor: \href{mailto:srfi-editors+at+srfi+dot+schemers+dot+org}{Arthur
A. Gleckler}
