\section{Title}\label{title}

Lazy Sequences

\section{Author}\label{author}

John Cowan

\section{Status}\label{status}

This SRFI is currently in \emph{final} status. Here is
\href{http://srfi.schemers.org/srfi-process.html}{an explanation} of
each status that a SRFI can hold. To provide input on this SRFI, please
send email to \texttt{srfi-127@nospamsrfi.schemers.org}. To subscribe to
the list, follow
\href{http://srfi.schemers.org/srfi-list-subscribe.html}{these
instructions}. You can access previous messages via the mailing list
\href{http://srfi-email.schemers.org/srfi-127}{archive}.

\begin{itemize}
\tightlist
\item
  Received: 2015/10/20
\item
  60-day deadline: 2015/12/19
\item
  Draft \#1 published: 2015/10/20
\item
  Draft \#2 published: 2015/10/21
\item
  Draft \#3 published: 2015/10/25
\item
  Draft \#4 published: 2015/10/29
\item
  Draft \#5 published: 2015/12/16
\item
  Finalized: 2016/1/18
\item
  Revised to fix errata: 2016/1/20, 2016/8/8
\end{itemize}

\section{Abstract}\label{abstract}

Lazy sequences (or lseqs, pronounced ``ell-seeks'') are a generalization
of lists. In particular, an lseq is either a proper list or a dotted
list whose last cdr is a
\href{http://srfi.schemers.org/srfi-121/srfi-121.html}{SRFI 121}
generator. A generator is a procedure that can be invoked with no
arguments in order to lazily supply additional elements of the lseq.
When a generator has no more elements to return, it returns an
end-of-file object. Consequently, lazy sequences cannot reliably contain
end-of-file objects.

This SRFI provides a set of procedures suitable for operating on lazy
sequences based on
\href{http://srfi.schemers.org/srfi-1/srfi-1.html}{SRFI 1}.

\section{Issues}\label{issues}

None at present.

\section{\texorpdfstring{\href{}{Rationale}}{Rationale}}\label{rationale}

Lazy sequences are more heavyweight than generators, on which they are
based, but they are more lightweight than
\href{http://srfi.schemers.org/srfi-41/srfi-41.html}{SRFI 41} streams.
However, streams are \emph{even}, as explained in the SRFI 41 rationale;
that is, the initial state of a stream does not have any elements that
have already been realized. By contrast, lazy sequences are \emph{odd},
meaning that at least one element is realized at all times unless the
lseq is empty. Therefore, when constructing an lseq in an iterative lazy
algorithm, only the cdr side of the lazy pair is lazily evaluated; the
car side is evaluated immediately, even if it is never used.

In most cases this doesn't matter, because calculating one additional
item is a negligible overhead. However, when you create a
self-referential lazy structure, in which the earlier elements of a
sequence are used to calculate the later elements of itself, a bit of
caution is needed; code that is valid for circular streams may not
terminate if it is mechanically converted to use lazy sequences. This
eagerness is also visible when side effects are involved; for example, a
lazy character sequence reading from a port may read one character
ahead.

This SRFI is less comprehensive than SRFI 1, because it omits many
procedures that process every element of their list arguments (at least,
when used in the absence of \texttt{call/cc}). Lseqs are meant to be
used with ordinary Scheme functions, which are strict, so neither left
nor right folds are provided: it is just as time-efficient to use
\texttt{lseq-realize} and then SRFI 1 \texttt{fold} or
\texttt{fold-right}, and just as space-efficient to use
\texttt{lseq-\textgreater{}generator} and SRFI 121's
\texttt{generator-fold}. Exceptions are \texttt{lseq-length} and
\texttt{lseq-for-each}, which are very common operations on lists and so
are provided.

Lazy alists have been omitted, as the point of an alist is to express a
function whose results are not algorithmically deducible from its
arguments. As a result, there is little point in using a generator
procedure to produce key-value mappings when an ordinary procedure can
compute the value from the key directly. The linear-update procedures of
SRFI 1 are also left out, as lazy sequences are not intended to be
mutated.

\subsection{Table of contents}\label{table-of-contents}

\protect\hyperlink{ProcedureIndex}{Procedure index}

\protect\hyperlink{Specification}{Specification}

\begin{itemize}
\tightlist
\item
  \protect\hyperlink{Constructors}{Constructors}
\item
  \protect\hyperlink{Predicates}{Predicates}
\item
  \protect\hyperlink{Selectors}{Selectors}
\item
  \protect\hyperlink{Whole}{The whole lazy sequence}
\item
  \protect\hyperlink{MappingFiltering}{Mapping and filtering}
\item
  \protect\hyperlink{Searching}{Searching}
\end{itemize}

\protect\hyperlink{SampleImplementation}{Sample Implementation}

\protect\hyperlink{Acknowledgements}{Acknowledgements}

\protect\hyperlink{ReferencesLinks}{References \& links}

\protect\hyperlink{Copyright}{Copyright}

\section{\texorpdfstring{\href{}{Procedure
Index}}{Procedure Index}}\label{procedure-index}

Here is a short list of the procedures provided by this SRFI.

\begin{description}
\item[ Constructors ]
\begin{verbatim}
generator->lseq 
\end{verbatim}
\item[ Predicates ]
\begin{verbatim}
lseq?         lseq=?
\end{verbatim}
\item[ Selectors ]
\begin{verbatim}
lseq-car     lseq-cdr
lseq-first   lseq-rest lseq-ref
lseq-take    lseq-drop   
\end{verbatim}
\item[ The whole lazy sequence ]
\begin{verbatim}
lseq-realize lseq->generator
lseq-length
lseq-append  lseq-zip
\end{verbatim}
\item[ Mapping and filtering ]
\begin{verbatim}
lseq-map        lseq-for-each
lseq-filter     lseq-remove
\end{verbatim}
\item[ Searching ]
\begin{verbatim}
lseq-find         lseq-find-tail 
lseq-any          lseq-every
lseq-index
lseq-take-while   lseq-drop-while
lseq-member       lseq-memq     lseq-memv
\end{verbatim}
\end{description}

\section{\texorpdfstring{\href{}{Specification}}{Specification}}\label{specification}

The templates given below obey the following conventions for procedure
formals:

\begin{longtable}[]{@{}ll@{}}
\toprule
lseq & A lazy sequence\tabularnewline
x, y, a, b & Any value\tabularnewline
object, value & Any value\tabularnewline
n, i & A natural number (an integer \textgreater{}= 0)\tabularnewline
proc & A procedure\tabularnewline
pred & A procedure whose return value is treated as a
boolean\tabularnewline
generator & A procedure with no arguments that returns a sequence of
values\tabularnewline
= & A boolean procedure taking two arguments\tabularnewline
\bottomrule
\end{longtable}

To interpret the examples, pretend that they are executed on a Scheme
that prints lazy sequences with the syntax of lists, truncating them
when they get too long.

\subsection{\texorpdfstring{\href{}{Constructors}}{Constructors}}\label{constructors}

Every list constructor procedure is also an lseq constructor procedure.
The procedure \texttt{generator-\textgreater{}lseq} constructs an lseq
based on the values of a generator. In order to prepend a value to an
lseq, simply use \texttt{cons}; to prepend more than one value, use SRFI
1's \texttt{cons*}.

\href{}{}

\texttt{generator-\textgreater{}lseq} generator -\textgreater{} lseq

Returns an lseq whose elements are the values generated by generator.
The exact behavior is as follows:

\begin{itemize}
\tightlist
\item
  Generator is invoked with no arguments to produce an object obj.
\item
  If obj is an end-of-file object, the empty list is returned.
\item
  Otherwise, a newly allocated pair whose car is obj and whose cdr is
  generator is returned.
\end{itemize}

\begin{verbatim}
(generator->lseq (make-iota-generator +inf.0 1)) => (1 2 3 ...)
\end{verbatim}

\subsection{\texorpdfstring{\href{}{Predicates}}{Predicates}}\label{predicates}

\begin{description}
\item[ \texttt{lseq?} x -\textgreater{} boolean \href{}{} ]
Returns \texttt{\#t} if x is an lseq. This procedure may also return
\texttt{\#t} if x is an improper list whose last cdr is a procedure that
requires arguments, since there is no portable way to examine a
procedure to determine how many arguments it requires. Otherwise it
returns \texttt{\#f}.
\item[ \href{}{} \texttt{lseq=?} elt=? lseq\textsubscript{1}
lseq\textsubscript{2} -\textgreater{} boolean ]
Determines lseq equality, given an element-equality procedure. Two lseqs
are equal if they are of the same length, and their corresponding
elements are equal, as determined by elt=?. When elt=? is called, its
first argument is always from lseq\textsubscript{1} and its second
argument is from lseq\textsubscript{2}.

The dynamic order in which the elt=? procedure is applied to pairs of
elements is not specified.

The elt=? procedure must be consistent with \texttt{eq?}. This implies
that two lseqs which are \texttt{eq?} are always \texttt{lseq=?}, as
well; implementations may exploit this fact to ``short-cut'' the
element-by-element equality tests.
\end{description}

\subsection{\texorpdfstring{\href{}{Selectors}}{Selectors}}\label{selectors}

\href{}{} \texttt{lseq-car\ \ \ }lseq -\textgreater{} object

\href{}{} \texttt{lseq-first\ \ }lseq -\textgreater{} object

These procedures are synonymous. They return the first element of lseq.
They are included for completeness, as they are the same as
\texttt{car}. It is an error to apply them to an empty lseq.

\href{}{} \texttt{lseq-cdr\ \ \ }lseq -\textgreater{} lseq

\href{}{} \texttt{lseq-rest\ \ }lseq -\textgreater{} lseq

These procedures are synonymous. They return an lseq with the contents
of lseq except for the first element. The exact behavior is as follows:

If lseq is a pair whose cdr is a procedure, then the procedure is
invoked with no arguments to produce an object obj.

\begin{itemize}
\tightlist
\item
  If obj is an end-of-file object, then the cdr of lseq is set to the
  empty list, which is returned.
\item
  If obj is any other object, then a new pair is allocated whose car is
  \emph{obj} and whose cdr is the cdr of lseq (i.e. the procedure). The
  cdr of lseq is set to the newly allocated pair, which is returned.
\end{itemize}

If lseq is a pair whose cdr is not a procedure, then the cdr is
returned.

If lseq is not a pair, it is an error.

\texttt{lseq-ref} lseq i -\textgreater{} value

Returns the ith element of lseq. (This is the same as
\texttt{(lseq-first\ (lseq-drop\ lseq\ i))}.) It is an error if i
\textgreater{}= n, where n is the length of lseq.

\begin{verbatim}
    
(lseq-ref '(a b c d) 2) => c
\end{verbatim}

\href{}{} \texttt{lseq-take} lseq i -\textgreater{} lseq

\href{}{} \texttt{lseq-drop} lseq i -\textgreater{} lseq

\texttt{lseq-take} lazily returns the first i elements of lseq.
\texttt{lseq-drop} returns all but the first i elements of lseq.

\begin{verbatim}
(lseq-take '(a b c d e)  2) => (a b)
(lseq-drop '(a b c d e)  2) => (c d e)
\end{verbatim}

\texttt{lseq-drop} is exactly equivalent to performing i
\texttt{lseq-rest} operations on lseq.

\subsection{\texorpdfstring{\href{}{The whole lazy
sequence}}{The whole lazy sequence}}\label{the-whole-lazy-sequence}

\begin{description}
\item[ \href{}{} \texttt{lseq-realize} lseq -\textgreater{} list ]
Repeatedly applies \texttt{lseq-cdr} to lseq until its generator (if
there is one) has been exhausted, and returns lseq, which is now
guaranteed to be a proper list. This procedure can be called on an
arbitrary lseq before passing it to a procedure which only accepts
lists. However, if the generator never returns an end-of-file object,
\texttt{lseq-realize} will never return.
\end{description}

\begin{description}
\item[ \href{}{} \texttt{lseq-\textgreater{}generator} lseq
-\textgreater{} generator ]
Returns a generator which when invoked will return all the elements of
lseq, including any that have not yet been realized.
\end{description}

\href{}{} \texttt{lseq-length\ \ }lseq -\textgreater{} integer

Returns the length of its argument, which is the non-negative integer n
such that \texttt{lseq-rest} applied n times to the lseq produces an
empty lseq. lseq must be finite, or this procedure will not return.

\begin{description}
\item[\texttt{lseq-append} \emph{lseq \ldots{}}]
Returns an lseq that lazily contains all the elements of all the lseqs
in order.
\end{description}

\href{}{}

\texttt{lseq-zip} lseq\textsubscript{1} lseq\textsubscript{2} \ldots{}
-\textgreater{} lseq

If \texttt{lseq-zip} is passed n lseqs, it lazily returns an lseq each
element of which is an n-element list comprised of the corresponding
elements from the lseqs. If any of the lseqs are finite in length, the
result is as long as the shortest lseq.

\begin{verbatim}
(lseq-zip '(one two three) 
     (generator->lseq (make-iota-generator +inf.0 1 1))
     (generator->lseq (make-repeating-generator) '(odd even))))
    => ((one 1 odd) (two 2 even) (three 3 odd))

(lseq-zip '(1 2 3)) => ((1) (2) (3))
\end{verbatim}

\subsection{\texorpdfstring{\href{}{Mapping and
filtering}}{Mapping and filtering}}\label{mapping-and-filtering}

\href{}{} \texttt{lseq-map} proc lseq\textsubscript{1}
lseq\textsubscript{2} \ldots{} -\textgreater{} lseq

The \texttt{lseq-map} procedure lazily applies proc element-wise to the
corresponding elements of the lseqs, where proc is a procedure taking as
many arguments as there are lseqs and returning a single value, and
returns an lseq of the results in order. The dynamic order in which proc
is applied to the elements of the lseqs is unspecified.

\begin{verbatim}
(lseq-map
  (lambda (x) (lseq-car (lseq-cdr x)))
  '((a b) (d e) (g h))) =>  (b e h)

(lseq-map (lambda (n) (expt n n))
     (make-iota-generator +inf.0 1 1)
    =>  (1 4 27 256 3125 ...)

(lseq-map + '(1 2 3) '(4 5 6)) =>  (5 7 9)

(let ((count 0))
  (lseq-map (lambda (ignored)
         (set! count (+ count 1))
         count)
       '(a b))) =>  (1 2) or (2 1)
\end{verbatim}

\href{}{} \texttt{lseq-for-each} proc lseq\textsubscript{1}
lseq\textsubscript{2} \ldots{} -\textgreater{} unspecified

The arguments to \texttt{lseq-for-each} are like the arguments to
\texttt{lseq-map}, but \texttt{lseq-for-each} calls proc for its side
effects rather than for its values. Unlike \texttt{lseq-map},
\texttt{lseq-for-each} is guaranteed to call proc on the elements of the
lseqs in order from the first element(s) to the last, and the value
returned by \texttt{lseq-for-each} is unspecified.

If none of the lseqs are finite, \texttt{lseq-for-each} never returns.

\begin{verbatim}
(let ((v (make-vector 5)))
  (lseq-for-each (let ((count 0))
                   (lambda (i)
                     (vector-set! v count (* i i))
                     (set! count (+ count 1))))
                 '(0 1 2 3 4))
  v)
  =>  (#0 1 2 3 4)
\end{verbatim}

\href{}{} \texttt{lseq-filter} pred lseq -\textgreater{} lseq

\href{}{} \texttt{lseq-remove} pred lseq -\textgreater{} lseq

The procedure \texttt{lseq-filter} lazily returns an lseq that contains
only the elements of lseq that satisfy pred.

The procedure \texttt{lseq-remove} is the same as \texttt{lseq-filter},
except that it returns elements that do not satisfy pred. These
procedures are guaranteed to call pred on the elements of the lseqs in
sequence order.

\begin{verbatim}
(lseq-filter odd? (generator->lseq (make-range-generator 1 5)))
  =>  (1 3)
\end{verbatim}

\begin{verbatim}
(lseq-remove odd? (generator->lseq (make-range-generator 1 5)))
  =>  (2 4)
\end{verbatim}

\subsection{\texorpdfstring{\href{}{Searching}}{Searching}}\label{searching}

The following procedures all search lseqs for the leftmost element
satisfying some criterion.

\href{}{} \texttt{lseq-find} pred lseq -\textgreater{} value

Return the first element of lseq that satisfies predicate pred, or
\texttt{\#f} if no element does. It cannot reliably be applied to lseqs
that include \texttt{\#f} as an element; use \texttt{lseq-find-tail}
instead. The predicate is guaranteed to be evaluated on the elements of
lseq in sequence order, and only as often as necessary.

\begin{verbatim}
(lseq-find even? '(3 1 4 1 5 9 2 6)) => 4
\end{verbatim}

\href{}{} \texttt{lseq-find-tail} pred lseq -\textgreater{} lseq or
\texttt{\#f}

Returns the longest tail of lseq whose first element satisfies pred, or
\texttt{\#f} if no element does. The predicate is guaranteed to be
evaluated on the elements of lseq in sequence order, and only as often
as necessary.

\texttt{lseq-find-tail} can be viewed as a general-predicate variant of
the \texttt{lseq-member} function.

Examples:

\begin{verbatim}
(lseq-find-tail even? '(3 1 37 -8 -5 0 0)) => (-8 -5 0 0)
(lseq-find-tail even? '(3 1 37 -5)) => #f

;; equivalent to (lseq-member elt lseq)
(lseq-find-tail (lambda (elt) (equal? x elt)) lseq)
\end{verbatim}

\href{}{} \texttt{lseq-take-while\ } pred lseq -\textgreater{} lseq

Lazily returns the longest initial prefix of lseq whose elements all
satisfy the predicate pred.

\begin{verbatim}
(lseq-take-while even? '(2 18 3 10 22 9)) => (2 18)
\end{verbatim}

\href{}{} \texttt{lseq-drop-while} pred lseq -\textgreater{} lseq

Drops the longest initial prefix of lseq whose elements all satisfy the
predicate pred, and returns the rest of the lseq.

\begin{verbatim}
(lseq-drop-while even? '(2 18 3 10 22 9)) => (3 10 22 9)
\end{verbatim}

Note that \texttt{lseq-drop-while} is essentially
\texttt{lseq-find-tail} where the sense of the predicate is inverted:
\texttt{lseq-find-tail} searches until it finds an element satisfying
the predicate; \texttt{lseq-drop-while} searches until it finds an
element that \emph{doesn't} satisfy the predicate.

\href{}{} \texttt{lseq-any} pred lseq\textsubscript{1}
lseq\textsubscript{2} \ldots{} -\textgreater{} value

Applies pred to successive elements of the lseqs, returning true if pred
returns true on any application. If an application returns a true value,
\texttt{lseq-any} immediately returns that value. Otherwise, it iterates
until a true value is produced or one of the lseqs runs out of values;
in the latter case, \texttt{lseq-any} returns \texttt{\#f}. It is an
error if \emph{pred} does not accept the same number of arguments as
there are \emph{lseqs} and return a boolean result.

Note the difference between \texttt{lseq-find} and \texttt{lseq-any} ---
\texttt{lseq-find} returns the element that satisfied the predicate;
\texttt{lseq-any} returns the true value that the predicate produced.

Like \texttt{lseq-every}, \texttt{lseq-any}'s name does not end with a
question mark --- this is to indicate that it does not return a simple
boolean (\texttt{\#t} or \texttt{\#f}), but a general value.

\begin{verbatim}
(lseq-any integer? '(a 3 b 2.7))   => #t
(lseq-any integer? '(a 3.1 b 2.7)) => #f
(lseq-any < '(3 1 4 1 5)
       '(2 7 1 8 2)) => #t
\end{verbatim}

\begin{verbatim}
(define (factorial n)
  (cond
    ((< n 0) #f)
    ((= n 0) 1)
    (else (* n (factorial (- n 1))))))
(lseq-any factorial '(-1 -2 3 4))
        => 6
\end{verbatim}

\href{}{} \texttt{lseq-every} pred lseq\textsubscript{1}
lseq\textsubscript{2} \ldots{} -\textgreater{} value

Applies pred to successive elements of the lseqs, returning true if the
predicate returns true on every application. If an application returns a
false value, \texttt{lseq-every} immediately returns that value.
Otherwise, it iterates until a false value is produced or one of the
lseqs runs out of values; in the latter case, \texttt{lseq-every}
returns the last value returned by pred, or \texttt{\#t} if pred was
never invoked. It is an error if \emph{pred} does not accept the same
number of arguments as there are \emph{lseqs} and return a boolean
result.

Like \texttt{lseq-any}, \texttt{lseq-every}'s name does not end with a
question mark --- this is to indicate that it does not return a simple
boolean (\texttt{\#t} or \texttt{\#f}), but a general value.

\begin{verbatim}
(lseq-every factorial '(1 2 3 4))
        => 24
\end{verbatim}

\href{}{} \texttt{lseq-index} pred lseq\textsubscript{1}
lseq\textsubscript{2} \ldots{} -\textgreater{} integer or \texttt{\#f}

Return the index of the leftmost element that satisfies pred.

Applies pred to successive elements of the lseqs, returning an index
usable with \texttt{lseq-ref} if the predicate returns true. Otherwise,
it iterates until one of the lseqs runs out of values, in which case
\texttt{\#f} is returned.

It is an error if \emph{pred} does not accept the same number of
arguments as there are \emph{lseqs} and return a boolean result.

The iteration stops when one of the lseqs runs out of values; in this
case, \texttt{lseq-index} returns \texttt{\#f}.

\begin{verbatim}
(lseq-index even? '(3 1 4 1 5 9)) => 2
(lseq-index < '(3 1 4 1 5 9 2 5 6) '(2 7 1 8 2)) => 1
(lseq-index = '(3 1 4 1 5 9 2 5 6) '(2 7 1 8 2)) => #f
\end{verbatim}

\href{}{} \texttt{lseq-member} x lseq {[} pred {]} -\textgreater{} lseq

\href{}{} \texttt{lseq-memq} x lseq -\textgreater{} lseq

\href{}{} \texttt{lseq-memv} x lseq -\textgreater{} lseq

These procedures return the longest tail of lseq whose first element is
x, where the tails of lseq are the non-empty lseqs returned by
\texttt{(lseq-drop\ lseq\ i)} for i less than the length of lseq. If x
does not occur in lseq, then \texttt{\#f} is returned.
\texttt{lseq-memq} uses \texttt{eq?} to compare x with the elements of
lseq, while \texttt{lseq-memv} uses \texttt{eqv?}, and
\texttt{lseq-member} uses pred, which defaults to \texttt{equal?}.

\begin{verbatim}
    (lseq-memq 'a '(a b c))           =>  (a b c)
    (lseq-memq 'b '(a b c))           =>  (b c)
    (lseq-memq 'a '(b c d))           =>  #f
    (lseq-memq (list 'a) '(b (a) c)) =>  #f
    (lseq-member (list 'a)
            '(b (a) c))           =>  ((a) c)
    (lseq-memq 101 '(100 101 102))    =>  *unspecified*
    (lseq-memv 101 '(100 101 102))    =>  (101 102)
\end{verbatim}

The equality procedure is used to compare the elements
e\textsubscript{i} of lseq to the key x in this way: the first argument
is always x, and the second argument is one of the lseq elements. Thus
one can reliably find the first element of lseq that is greater than
five with \texttt{(lseq-member\ 5\ lseq\ \textless{})}

Note that fully general lseq searching may be performed with the
\texttt{lseq-find-tail} procedure, \emph{e.g.}

\begin{verbatim}
(lseq-find-tail even? lseq) ; Find the first elt with an even key.
\end{verbatim}

\section{\texorpdfstring{\href{}{Sample
Implementation}}{Sample Implementation}}\label{sample-implementation}

The files in the implementation are in the SRFI 127 repository, and are
as follows:

\begin{itemize}
\tightlist
\item
  \texttt{lseq.sld} - an R7RS library
\item
  \texttt{lseq.scm} - a Chicken library
\item
  \texttt{lseq-impl.scm} - portable implementation code
\item
  \texttt{lseq-test.scm} - a Chicken test-egg file
\end{itemize}

\section{\texorpdfstring{\href{}{Acknowledgements}}{Acknowledgements}}\label{acknowledgements}

Without the work of Olin Shivers on
\href{http://srfi.schemers.org/srfi-1/srfi-1.html}{SRFI 1}, this SRFI
would not exist. Everyone acknowledged there is transitively
acknowledged here. This is not to imply that either Olin or anyone else
necessarily endorses the final results, of course.

Special thanks to Shiro Kawai, whose Gauche implementation of lazy
sequences inspired this one, and to Kragen Javier Sitaker, who did a
thorough review.

\section{Copyright}\label{copyright}

Copyright (C) John Cowan (2015). All Rights Reserved.

Permission is hereby granted, free of charge, to any person obtaining a
copy of this software and associated documentation files (the
``Software''), to deal in the Software without restriction, including
without limitation the rights to use, copy, modify, merge, publish,
distribute, sublicense, and/or sell copies of the Software, and to
permit persons to whom the Software is furnished to do so, subject to
the following conditions:

The above copyright notice and this permission notice shall be included
in all copies or substantial portions of the Software.

THE SOFTWARE IS PROVIDED ``AS IS'', WITHOUT WARRANTY OF ANY KIND,
EXPRESS OR IMPLIED, INCLUDING BUT NOT LIMITED TO THE WARRANTIES OF
MERCHANTABILITY, FITNESS FOR A PARTICULAR PURPOSE AND NONINFRINGEMENT.
IN NO EVENT SHALL THE AUTHORS OR COPYRIGHT HOLDERS BE LIABLE FOR ANY
CLAIM, DAMAGES OR OTHER LIABILITY, WHETHER IN AN ACTION OF CONTRACT,
TORT OR OTHERWISE, ARISING FROM, OUT OF OR IN CONNECTION WITH THE
SOFTWARE OR THE USE OR OTHER DEALINGS IN THE SOFTWARE.

\begin{center}\rule{0.5\linewidth}{\linethickness}\end{center}
